\documentclass[t]{beamer}
\usepackage{lmodern} % Need to use this package for proper font encoding.
\usepackage[T1]{fontenc}
\usepackage{amsmath}
\usepackage{graphicx}
\usepackage{multicol}
\usepackage{booktabs}
\usepackage{bookmark}
\usepackage[table]{xcolor}
% Title slide information
\title{Thesis Defence: Evaluation of Gene Finding Tools When Applied to \textit{Trichoderma} Genomes}
\date{January, 2026} 
\author{Connor Burbridge}
\titlegraphic{UsaskAerial.jpg} 
\institute{Department of Computer Science} 
\usetheme{usask}

\begin{document}

\begin{frame}
	\titlepage
\end{frame}

%%%%%%%%%%%%%%%%%
\begin{frame}
	\frametitle{Outline} 

	\begin{itemize}
		\item \textit{Trichoderma} fungi.
		\item Novel \textit{Trichoderma} genomes.
		\item Motivation.
		\item Research objectives.
		\item Workflow overview.
		\item Results.
		\item Conclusions and recommendations.
		\item Questions.
	\end{itemize}
\end{frame}

%%%%%%%%%%%%%%%%


\begin{frame}
	\frametitle{What is \textit{Trichoderma}?}
	\begin{itemize}
		\item \textit{Trichoderma} is a genus of filamentous fungi that is ubiquitous in soil and plays a significant role in nutrient cycling.
		\item They are also used in biocontrol and as biofertilizers.
		\item Known for their production of plant cell wall degrading enzymes, and \textbf{secondary metabolites}.
		\item Further genomic studies can help in understanding their biology and potential applications.
	\end{itemize}
	\vspace{0.05cm}
	\centering
	%\includegraphics[width=0.2\textwidth]{./Trichoderma_harzianum.jpg}

	\begin{figure}
		\centering
		\begin{minipage}{0.5\textwidth}
			\centering
			\includegraphics[width=0.35\linewidth]{./Trichoderma_harzianum.jpg}
			\caption{\textit{T. harzianum}}
		\end{minipage}\hfill
		\begin{minipage}{0.48\textwidth}
			\centering
			\includegraphics[width=0.6\linewidth]{./Trichoderma_petri.jpg}
			\caption{\textit{Trichoderma} colony}
		\end{minipage}
	\end{figure}
\end{frame}

\begin{frame}
	\frametitle{Novel \textit{Trichoderma} Genomes}
	\begin{itemize}
		\item \textit{Trichoderma} species are diverse, with many species not yet fully characterized.
		\item DC1 and Tsth20, shown to improve drought and salt tolerance when applied to crops, and have been shown to breakdown hydrocarbons in soils.
		\item Recent advances in sequencing technology have made it possible to generate high-quality genomes for these species.
	\end{itemize}
\end{frame}

\begin{frame}
	\frametitle{Motivation} 
	\begin{itemize}
		\item To understand \textit{Trichoderma} biology, we must identify genes within their genomes.
		\item Gene prediction tools vary significantly in implementation and performance.
		\item \textbf{Few comparative studies exist for these tools in fungi, particularly in \textit{Trichoderma}}.
		\item New high-quality \textit{Trichoderma} genomes (DC1 and Tsth20) enable comparative evaluation of gene finding tools. 
	\end{itemize}
\end{frame}

\begin{frame}
	\frametitle{Small but Significant Details}
	\begin{itemize}
		\item Before diving too deep, how can we confirm that gene finders predictions are different?
		\item Kolmogorov-Smirnov (KS) tests were performed on gene length distributions from preliminary gene predictions.
		\begin{center}
			\includegraphics[width=0.9\textwidth]{../../working-thesis/figures/t-reesei-cdf-lengths-log.pdf}
		\end{center}
	\end{itemize}	
\end{frame}

\begin{frame}
	\frametitle{Gene Finding is still Challenging}
	\begin{columns}
		\column{0.38\linewidth}
			\centering
			\includegraphics[height=4cm, width=5cm]{../../working-thesis/figures/gc-plot.pdf}
		\column{0.54\linewidth}
			We've been finding genes for decades. What's going on?
			Sure, exons and introns as well as start and stop positions make gene finding tricky.
			Inherent properties of genomes also complicate gene finding.

	\end{columns}
\end{frame}

\begin{frame}
	\frametitle{Research Objectives}
	\begin{itemize}
		\item Assemble and evaluate novel assemblies of DC1 and Tsth20.
		\item Apply gene finders Braker2 and GeneMark in DC1, Tsth20, and three other RefSeq assemblies.
		\item Compare gene finding tools based on relevant criteria, including:
		\begin{itemize}
			 \item Proportions of gene lengths predicted.
			 \item Presence of functional domains and closely related protein sequences.
			 \item Presence of genes in AT-rich sequence.
			 \item Agreement of gene finders on start and stop positions of a gene. 
		\end{itemize}
	\end{itemize}
\end{frame}



\begin{frame}
	\frametitle{Workflow Overview}
	\centering
	\includegraphics[width=\textwidth]{../../working-thesis/figures/workflow-simple.pdf}
\end{frame}

\begin{frame}
	\frametitle{Acknowledgements}
	\begin{itemize}
		\item My supervisors for their committed support through COVID and other challenges.
		\item The Global Institute for Food Security for providing the data for this project as well as a portion of my funding.
		\item My committee members for their feedback and support.
	\end{itemize}
\end{frame}

\begin{frame}
	\frametitle{Datasets and Tools}
	\begin{itemize}
		\item \textbf{Datasets:}
		\begin{itemize}
			\item Novel \textit{Trichoderma} genomes: DC1 and Tsth20.
			\item Reference genomes and annotations: \textit{T. reesei, T. harzianum, T.virens}.
			\item RNAseq training data from \textit{T. reesei}. 
			\item Benchmarking Universal Single-Copy Orthologs (BUSCO) fungal database.
			\item Protein sequence queries for tblastn from \textit{T. atroviride, Fusarium graminearum, and Saccharomyces cerevisiae}.	
		\end{itemize}
		\item \textbf{Tools:}
		\begin{itemize}
			\item Sequence processing: FastQC, Trimmomatic, Hisat2.
			\item Genome assembly: NextDenovo and NextPolish.
			\item Gene finding tools: Braker2 and GeneMark-ES.
			\item Evaluation tools: BUSCO, tblastn, InterProScan, and custom scripts.
		\end{itemize}
	\end{itemize}
\end{frame}

\begin{frame}
	\vspace{0.75cm}
	\centering
	\includegraphics[width=0.5\textwidth]{../../working-thesis/figures/conclusion-snip.png}
\end{frame}

\begin{frame}
	\frametitle{Image Credits}
	\begin{itemize}
		\item \textit{T. harzianum} image: \url{https://en.wikipedia.org/wiki/Trichoderma}
		\item \textit{Trichoderma colony} image:\url{https://biocontrol.entomology.cornell.edu/pathogens/trichoderma.php}
	\end{itemize}
\end{frame}

\end{document}