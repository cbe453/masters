\section{BUSCO Results}

The results of BUSCO analysis using the fungal subset provided by
BUSCO are presented in table \ref{table:busco}. Results from BUSCO
indicate that all gene sets considered in this analysis have a BUSCO
completeness of 99.2\% or higher, with a maximum completeness of
99.9\% for some gene sets. In general, Braker2 and RefSeq have the
most BUSCO complete sets of gene predictions of the three tools
considered. Interestingly, Braker2 produces far more duplicated BUSCO
matches than both GeneMark and RefSeq. Examining the BUSCO output
logs, this appears to be due to Braker2 producing more than one RNA
for some genes predictions, resulting in multiple similar proteins. In
general, all gene finders perform exceptionally well in regards to
BUSCO performance. While these results do not capture the entire set
of genes possibly present in these \textit{Trichoderma} assemblies,
they do confirm that the gene finders are at minimum predicting
evolutionarily conserved fungal genes.

\begin{table}
  \begin{center}
    \begin{subtable}{\textwidth}
      \centering
      \begin{tabular}{|c|c|c|c|c|c|c|}
        \hline
        Strain & Complete & Single & Duplicated & Fragmented & Missing & No. markers \\ \hline
        DC1 & 99.5 & 80.2 & 19.3 & 0.1 & 0.4 & 758 \\ \hline
        Tsth20 & 99.9 & 81.7 & 18.2 & 0.0 & 0.1 & 758 \\ \hline
        \textit{T. harzianum} & 99.7 & 80.2 & 19.5 & 0.0 & 0.3 & 758 \\ \hline
        \textit{T. virens} & 99.8 & 79.0 & 20.8 & 0.1 & 0.1 & 758 \\ \hline
        \textit{T. reesei} & 99.9 & 85.5 & 14.4 & 0.1 & 0.0 & 758 \\ \hline
      \end{tabular}
      \caption{Braker2}
    \end{subtable}
    \begin{subtable}{\textwidth}
      \centering
      \begin{tabular}{|c|c|c|c|c|c|c|}
        \hline
        Strain & Complete & Single & Duplicated & Fragmented & Missing & No. markers \\ \hline
        DC1 & 99.2 & 98.8 & 0.4 & 0.3 & 0.5 & 758 \\ \hline
        Tsth20 & 99.8 & 99.1 & 0.7 & 0.0 & 0.2 & 758 \\ \hline
        \textit{T. harzianum} & 99.6 & 98.9 & 0.7 & 0.0 & 0.4 & 758 \\ \hline
        \textit{T. virens} & 99.7 & 99.2 & 0.5 & 0.1 & 0.2 & 758 \\ \hline
        \textit{T. reesei} & 99.6 & 99.5 & 0.1 & 0.0 & 0.4 & 758 \\ \hline
      \end{tabular}
      \caption{GeneMark}
    \end{subtable}
    \begin{subtable}{\textwidth}
      \centering
      \begin{tabular}{|c|c|c|c|c|c|c|}
        \hline
        Strain & Complete & Single & Duplicated & Fragmented & Missing & No. markers \\ \hline
        \textit{T. harzianum} & 99.9 & 99.2 & 0.7 & 0.0 & 0.1 & 758 \\ \hline
        \textit{T. virens} & 99.5 & 98.8 & 0.7 & 0.3 & 0.2 & 758 \\ \hline
        \textit{T. reesei} & 99.8 & 99.5 & 0.3 & 0.0 & 0.2 & 758 \\ \hline
      \end{tabular}
      \caption{RefSeq}
    \end{subtable}
  \end{center}
  \caption{Results from BUSCO using the fungal analysis option
    organized by gene finding tool. For more information on the
    categories assigned by BUSCO, please refer to the documentation.}
  \label{table:busco}
\end{table}

While BUSCO matches are a good metric for general performance of gene
finders, it is also important to investigate BUSCO proteins without
matching gene predictions. Tables \ref{table:braker-busco},
\ref{table:genemark-busco} and \ref{table:refseq-busco}, show
breakdowns of genes missed by each gene finder across the
\textit{Trichoderma} assemblies. Braker2 misses four unique proteins
across the five \textit{Trichoderma} assemblies, with only one protein
missing in more than one assembly. This protein represents a formyl
transferase protein. GeneMark predictions miss six unique BUSCO
proteins, with two proteins missing in more than one assembly. These
proteins are the same formyl transferase missed by Braker2,
which was missed in four of the five assemblies, and a
ubiquitin-conjugating enzyme, which was missed in all five
assemblies. RefSeq, being the gene finder with the most BUSCO complete
set of gene predictions, misses only three unique BUSCO proteins in
the three assemblies. Of those three proteins, two of them are missed
twice. Those missing proteins represent a YEATS protein domain and
Midasin protein.

\begin{table}
  \centering
  \begin{tabular}{|c|c|c|c|c|c|c|}
    \hline
    BUSCO ID & Annotation & DC1 & Tsth20 & \textit{T. reesei} & \textit{T. harzianum} & \textit{T. reesei} \\ \hline
    195619at4751 & \makecell{Pyridoxal phosphate-dependent \\ transferase} & X & \checkmark & \checkmark & \checkmark & \checkmark \\ \hline 
    285254at4751 & Aminoacyl-tRNA synthetase & \checkmark & \checkmark & \checkmark & X & \checkmark \\ \hline
    348020at4751 & Formyl transferase & X & X & \checkmark & X & \checkmark \\ \hline
    497024at4751 & Zinc finger C2H2-type & X & \checkmark & \checkmark & \checkmark & \checkmark \\ \hline 
  \end{tabular}
  \caption[Braker2 missed BUSCO proteins]{The presence (\checkmark) or
    absence (X) of all BUSCO IDs missed by Braker2 in each
    \textit{Trichoderma} assembly.}
  \label{table:braker-busco}
\end{table}

\begin{table}
  \centering
  \begin{tabular}{|c|c|c|c|c|c|c|}
    \hline
    BUSCO ID & Annotation & DC1 & Tsth20 & \textit{T. reesei} & \textit{T. harzianum} & \textit{T. reesei} \\ \hline
    195619at4751 & \makecell{Pyridoxal phosphate-dependent \\ transferase} & X & \checkmark & \checkmark & \checkmark & \checkmark \\ \hline
    285254at4751 & Aminoacyl-tRNA synthetase & \checkmark & \checkmark & \checkmark & X & \checkmark \\ \hline
    348020at4751 & Formyl transferase & X & X & X & X & \checkmark \\ \hline 
    438731at4751 & LSM domain & \checkmark & \checkmark & X & \checkmark & \checkmark  \\ \hline
    470813at4751 & Ubiquitin-conjugating enzyme & X & X & X & X & X \\ \hline
    497024at4751 & Zinc finger C2H2-type & X & \checkmark & \checkmark & \checkmark & \checkmark \\ \hline
  \end{tabular}
  \caption[GeneMark missed BUSCO proteins]{The presence (\checkmark)
    or absence (X) of all BUSCO IDs missed by GeneMark in each
    \textit{Trichoderma} assembly.}
  \label{table:genemark-busco}
\end{table}

\begin{table}
  \centering
  \begin{tabular}{|c|c|c|c|c|c|c|}
    \hline
    BUSCO ID & Annotation & DC1 & Tsth20 & \textit{T. reesei} & \textit{T. harzianum} & \textit{T. reesei} \\ \hline
    494at4751 & Midasin & N/A & N/A & X & X & \checkmark\\ \hline
    315802at4751 & tRNA dimethylallyltransferase & N/A & N/A & \checkmark & \checkmark & X \\ \hline
    352224at4751 & YEATS & N/A & N/A & X & \checkmark & X \\ \hline
  \end{tabular}
  \caption[RefSeq missed BUSCO proteins]{The presence (\checkmark) or
    absence (X) of all BUSCO IDs missed by RefSeq in each
    \textit{Trichoderma} assembly.}
  \label{table:refseq-busco}
\end{table}

Braker2, GeneMark and RefSeq all demonstrate excellent coverage of the
BUSCO fungal protein set, indicating that these gene finders are
capable of predicting genes that are expected to be present in these
assemblies. From this we can say that the foundations of the
underlying gene models used by each gene finder are solid. Braker2
produces more duplicate matches than GeneMark and RefSeq, but this is
likely due to mutliple isoforms of possible genes being present in the
input data.Despite excellent coverage of the BUSCO fungal proteins,
all three gene finders miss some BUSCO proteins in their
predictions. GeneMark misses the most proteins and in paritcularly
struggles with predicting a formyl transferase and a
ubiquitin-conjugatin enzyme. Braker2 also appears to have difficulty
predicting a formyl transferase just as GeneMark did. RefSeq misses
the fewest BUSCO proteins and does not appear to systematically miss
certain proteins, although it is hard to draw a conclusion with only
three assemblies considered. It is also worth noting that RefSeq
misses completely different proteins than the other gene finders while
Braker2 and GeneMark do share some missed proteins.
