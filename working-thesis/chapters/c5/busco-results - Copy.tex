\section{BUSCO Results}
\label{section:busco}

The results of BUSCO analysis using the fungal subset provided by
BUSCO are presented in Table \ref{table:busco}. Results from BUSCO
indicate that all gene sets considered in this analysis have a BUSCO
completeness of 99.2\% or higher, with a maximum completeness of
99.9\% for some gene sets. In general, Braker2 and RefSeq have the
most BUSCO complete sets of gene predictions of the three tools
considered. Interestingly, Braker2 produces far more duplicated BUSCO
matches than both GeneMark and RefSeq. Examining the BUSCO output
logs, this appears to be due to Braker2 predicting more than one
coding sequence for some genes predictions, resulting in multiple
similar proteins. In general, all gene finders perform exceptionally
well in regards to BUSCO performance. While these results do not
capture the entire set of genes possibly present in these
\textit{Trichoderma} assemblies, they do confirm that the gene finders
are at minimum predicting many evolutionarily conserved fungal genes.

\begin{table}
  \begin{center}
    \begin{subtable}{\textwidth}
      \centering
      \begin{tabular}{|c|c|c|c|c|c|c|}
        \hline
        Strain & Complete & Single & Duplicated & Fragmented & Missing \\ \hline
        DC1 & 99.5 & 80.2 & 19.3 & 0.1 & 0.4 \\ \hline
        Tsth20 & 99.9 & 81.7 & 18.2 & 0.0 & 0.1 \\ \hline
        \textit{T. harzianum} & 99.7 & 80.2 & 19.5 & 0.0 & 0.3 \\ \hline
        \textit{T. virens} & 99.8 & 79.0 & 20.8 & 0.1 & 0.1 \\ \hline
        \textit{T. reesei} & 99.9 & 85.5 & 14.4 & 0.1 & 0.0 \\ \hline
      \end{tabular}
      \caption{Braker2}
    \end{subtable}
    \begin{subtable}{\textwidth}
      \centering
      \begin{tabular}{|c|c|c|c|c|c|c|}
        \hline
        Strain & Complete & Single & Duplicated & Fragmented & Missing \\ \hline
        DC1 & 99.2 & 98.8 & 0.4 & 0.3 & 0.5 \\ \hline
        Tsth20 & 99.8 & 99.1 & 0.7 & 0.0 & 0.2 \\ \hline
        \textit{T. harzianum} & 99.6 & 98.9 & 0.7 & 0.0 & 0.4 \\ \hline
        \textit{T. virens} & 99.7 & 99.2 & 0.5 & 0.1 & 0.2 \\ \hline
        \textit{T. reesei} & 99.6 & 99.5 & 0.1 & 0.0 & 0.4 \\ \hline
      \end{tabular}
      \caption{GeneMark}
    \end{subtable}
    \begin{subtable}{\textwidth}
      \centering
      \begin{tabular}{|c|c|c|c|c|c|c|}
        \hline
        Strain & Complete & Single & Duplicated & Fragmented & Missing \\ \hline
        \textit{T. harzianum} & 99.9 & 99.2 & 0.7 & 0.0 & 0.1 \\ \hline
        \textit{T. virens} & 99.5 & 98.8 & 0.7 & 0.3 & 0.2 \\ \hline
        \textit{T. reesei} & 99.8 & 99.5 & 0.3 & 0.0 & 0.2 \\ \hline
      \end{tabular}
      \caption{RefSeq}
    \end{subtable}
  \end{center}
  \caption{Results from BUSCO using the fungal analysis option
    organized by gene finding tool. The selected BUSCO dataset
    contains 758 markers. For more information on the categories
    assigned by BUSCO, please refer to the documentation.}
  \label{table:busco}
\end{table}

While BUSCO matches are a good metric for general performance of gene
finders, it is also important to investigate BUSCO proteins without
matching gene predictions. Table \ref{table:busco}, shows breakdowns
of genes missed by each gene finder across the \textit{Trichoderma}
assemblies. Braker2 misses four unique proteins across the five
\textit{Trichoderma} assemblies, with only one protein missing in more
than one assembly. This protein represents a formyl transferase
protein. GeneMark predictions miss six unique BUSCO proteins, with two
proteins missing in more than one assembly. These proteins are the
same formyl transferase missed by Braker2, which was missed in four of
the five assemblies, and a ubiquitin-conjugating enzyme, which was
missed in all five assemblies. RefSeq, being the gene finder with the
most BUSCO complete set of gene predictions, misses only three unique
BUSCO proteins in the three assemblies. Those missing proteins
represent a YEATS protein domain and Midasin protein. Of those three
proteins, the Midasin and YEATS proteins are missed in two of the
three assemblies from NCBI. Those missing proteins represent a YEATS
protein domain and Midasin protein.

%\begin{table}
%  \centering
%  \begin{tabular}{|c|c|c|c|c|c|c|}
%    \hline
%    BUSCO ID & Annotation & DC1 & Tsth20 & \textit{T. reesei} & \textit{T. harzianum} & \textit{T. reesei} \\ \hline
%    195619at4751 & \makecell{Pyridoxal phosphate-dependent \\ transferase} &  & \checkmark & \checkmark & \checkmark & \checkmark \\ \hline 
%    285254at4751 & Aminoacyl-tRNA synthetase & \checkmark & \checkmark & \checkmark &  & \checkmark \\ \hline
%    348020at4751 & Formyl transferase &  &  & \checkmark &  & \checkmark \\ \hline
%    497024at4751 & Zinc finger C2H2-type &  & \checkmark & \checkmark & \checkmark & \checkmark \\ \hline 
%  \end{tabular}
%  \caption[Braker2 missed BUSCO proteins]{The presence (\checkmark) or
%    absence (X) of all BUSCO IDs missed by Braker2 in each
%    \textit{Trichoderma} assembly.}
%  \label{table:braker-busco}
%\end{table}


\begin{center}
  \begin{table}
  \makebox[\textwidth]{
  \begin{tabular}{|c|c|c|c|c|c|c|c|}
    \hline
    Tool & BUSCO ID & Annotation & DC1 & Tsth20 & \textit{T. reesei} & \textit{T. harzianum} & \textit{T. virens} \\ \hline
    Braker2 & 195619at4751 & \makecell{Pyridoxal phosphate-dependent \\ transferase} & \  & \checkmark & \checkmark & \checkmark & \checkmark \\ \hline
    Braker2 & 285254at4751 & Aminoacyl-tRNA synthetase & \checkmark & \checkmark & \checkmark &  & \checkmark \\ \hline
    Braker2 & 348020at4751 & Formyl transferase &  &  & \checkmark &  & \checkmark \\ \hline
    Braker2 & 497024at4751 & Zinc finger C2H2-type &  & \checkmark & \checkmark & \checkmark & \checkmark \\ \hline
    GeneMark & 195619at4751 & \makecell{Pyridoxal phosphate-dependent \\ transferase} &  & \checkmark & \checkmark & \checkmark & \checkmark \\ \hline
    GeneMark & 285254at4751 & Aminoacyl-tRNA synthetase & \checkmark & \checkmark & \checkmark &  & \checkmark \\ \hline
    GeneMark & 348020at4751 & Formyl transferase &  &  &  &  & \checkmark \\ \hline 
    GeneMark & 438731at4751 & LSM domain & \checkmark & \checkmark &  & \checkmark & \checkmark  \\ \hline
    GeneMark & 470813at4751 & Ubiquitin-conjugating enzyme &  &  &  &  &  \\ \hline
    GeneMark & 497024at4751 & Zinc finger C2H2-type &  & \checkmark & \checkmark & \checkmark & \checkmark \\ \hline
    RefSeq & 494at4751 & Midasin & N/A & N/A &  &  & \checkmark\\ \hline
    RefSeq & 315802at4751 & tRNA dimethylallyltransferase & N/A & N/A & \checkmark & \checkmark &  \\ \hline
    RefSeq & 352224at4751 & YEATS & N/A & N/A &  & \checkmark &  \\ \hline
  \end{tabular}
  }
  \caption[GeneMark missed BUSCO proteins]{The presence (\checkmark)
    or absence of all BUSCO IDs missed by Braker2, GeneMark and RefSeq
    in each \textit{Trichoderma} assembly.}
  \label{table:genemark-busco}
\end{table}
\end{center}

%\begin{table}
%  \centering
%  \begin{tabular}{|c|c|c|c|c|c|c|}
%    \hline
%    BUSCO ID & Annotation & DC1 & Tsth20 & \textit{T. reesei} & \textit{T. harzianum} & \textit{T. reesei} \\ \hline
%    494at4751 & Midasin & N/A & N/A & X & X & \checkmark\\ \hline
%    315802at4751 & tRNA dimethylallyltransferase & N/A & N/A & \checkmark & \checkmark & X \\ \hline
%    352224at4751 & YEATS & N/A & N/A & X & \checkmark & X \\ \hline
%  \end{tabular}
%  \caption[RefSeq missed BUSCO proteins]{The presence (\checkmark) or
%    absence (X) of all BUSCO IDs missed by RefSeq in each
%    \textit{Trichoderma} assembly.}
%  \label{table:refseq-busco}
%\end{table}

Braker2, GeneMark and RefSeq all demonstrate excellent coverage of the
BUSCO fungal protein set, indicating that these gene finders are
capable of predicting genes that are expected to be present in these
assemblies. From this we can say that the foundations of the
underlying gene models used by each gene finder are solid. Braker2
produces more duplicate matches than GeneMark and RefSeq, but this is
likely due to multiple isoforms of possible genes being present in the
input data. Despite excellent coverage of the BUSCO fungal proteins,
all three gene finders miss some BUSCO proteins in their
predictions. GeneMark misses the most proteins and in particularly
struggles with predicting a formyl transferase and a
ubiquitin-conjugating enzyme. Braker2 also appears to have difficulty
predicting a formyl transferase just as GeneMark did. RefSeq misses
the fewest BUSCO proteins and does not appear to systematically miss
certain proteins, although it is hard to draw a conclusion with only
three assemblies considered. It is also worth noting that RefSeq
misses completely different proteins than the other gene finders while
Braker2 and GeneMark do share some missed proteins. Finally, we note
the possibility that human curation of RefSeq datasets is responsible
for these differences, but this requires further investigation.
