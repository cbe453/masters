\section{InterProScan as Supporting Evidence for Predicted Genes}

Pfam hits from InterProScan analysis are presented in table
~\ref{ips-pfam:table}.


\begin{table}
  \centering
  \begin{tabular}{|c|c|c|c|c|c|c|}
    \hline
    Assembly & \makecell{Braker \\ Proteins} & \makecell{Braker Pfam \\ Hits} & \makecell{GeneMark \\ Proteins} & \makecell{GeneMark Pfam \\Hits} & \makecell{RefSeq \\ Proteins} & \makecell{RefSeq Pfam \\ Hits}  \\ \hline
    DC1 & 14479 & 10676 & 11354 & 8416  & N/A & N/A \\ \hline
    Tsth20 & 15546 & 11389 & 12373 & 9168 & N/A & N/A \\ \hline
    \textit{T. reesei} & 11704 & 8471 & 9196 & 6990 & 9111 & 6964 \\ \hline
    \textit{T. harzianum} & 15408 & 11370 & 12164 & 9061 & 14065 & 9293 \\ \hline
    \textit{T. virens} & 15062 & 11249 & 11866 & 8871 & 12383 & 9062 \\ \hline
  \end{tabular}
  \caption[InterProScan Pfam Evidence]{Table with counts of predicted genes with Pfam annotations from InterProScan}
  \label{table:regioncounts}
\end{table}

