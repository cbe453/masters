\section{Initial Gene Finding Results} 
\label{section:gene-finding}

Prior to discussing gene finding results, we will first define the
terms gene and coding sequence in the context of this work. We refer
to a gene as a set of start and stop coordinates in a genomic
sequence. This definition of a gene simplifies processing, although
the definition could be expanded in the future to include introns,
exons and potential up and downstream sequences as well. Gene finders
may also predict isoforms, or alternative splice variants of a gene
based on evidence provided in training, resulting in multiple
potential coding sequences for a gene. In this work, we refer to
coding sequences as the set of all coding sequences predicted by a
gene finder.

Counts of genes and coding sequences predicted by Braker2, GeneMark
and RefSeq are shown in Table \ref{table:gene-counts}. Immediately we
see that Braker2 predicts far fewer genes in all assemblies, except in
the case of \textit{Trichoderma reesei.} This is possibly due to the
effects of training the Braker2 gene model using data from
\textit{Trichoderma reesei}, which has a significantly smaller genome
in comparison to other \textit{Trichoderma} assemblies, although
genome size is not always indicative of gene content. Regardless, we
observe a difference in the number of genes predicted by Braker2 in
comparison to GeneMark and RefSeq. The number of genes predicted by
GeneMark and RefSeq are similar, except in the case of
\textit{T. harzianum}, in which RefSeq predicts roughly 17\% more
genes than GeneMark. Braker2 consistently predicts more coding
sequences than GeneMark and RefSeq. RefSeq also appears to predict
multiple coding sequences for each gene but in fewer numbers than
Braker2. Coding sequence prediction counts in \textit{T. harzianum}
from RefSeq are also interesting, with RefSeq predicting fewer coding
sequences than genes. Why this occurs is unknown but may warrant
further investigation.

\begin{table}
  \centering
  \begin{tabular}{|c|c|c|c|c|c|c|}
    \hline
    Assembly & Braker2 & & GeneMark & & RefSeq & \\ \hline
     & Genes & CDS & Genes & CDS & Genes & CDS \\ \hline
    DC1 & 8546 & 8637 & 11353 & 11353 & N/A & N/A \\ \hline
    Tsth20 & 8784 & 8858 & 12362 & 12362 & N/A & N/A \\ \hline
    \textit{T. reesei} & 9659 & 10175 & 9196 & 9196 & 9109 & 9118 \\ \hline
    \textit{T. harzianum} & 8314 & 8385 & 12164 & 12164 & 14269 & 14090 \\ \hline
    \textit{T. virens} & 7801 & 7863 & 11866 & 11866 & 12405 & 12406 \\ \hline
  \end{tabular}
  \caption[Gene prediction counts]{Number of genes and coding sequences
    (isoforms) predicted by each gene finder for each
    \textit{Trichoderma} genome.}
  \label{table:gene-counts}
\end{table}


