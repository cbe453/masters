\section{BLAST Results}
Results from the T-BLAST-N runs are presented in table
\ref{table:tblastn}. Initial BLAST results appear promising for both
the \textit{T. atroviride} and \textit{Fusarium} datasets. All
assemblies considered contain at minimum 89\% of the reference protein
sequences in the case of \textit{T. atroviride} and a minimum of 75\%
in the case of \textit{Fusarium}. In the case of
\textit{S. cerevisiae}, a minimum of 57\% of reference proteins
matched. The decreasing percentage of hits reported conincides with
increasing distances in the evolutionary tree. These results provide
rough validation that the assemblies contain potential for protein
coding sequences.


\begin{table}
  \centering
  \begin{tabular}{|c|c|c|c|c|c|c|}
    \hline
    Reference & Ref. Proteins & DC1 & Tsth20 & \textit{T. reesei} & \textit{T. harzianum} & \textit{T. virens}  \\ \hline
    \textit{T. atroviride} & 11807 & 11552 & 11080 & 10601 & 11081 & 11078 \\ \hline 
    \textit{Fusarium} & 13312 & 10327 & 10429 & 10064 & 10434 & 10490 \\ \hline
    \textit{S. cerevisiae} & 6014 & 3537 & 3517 & 3445 & 3509 & 3500 \\ \hline
  \end{tabular}
  \caption{tBLASTn hits from reference protein sequences to selected
    assemblies of intereset. Hits are reported if the alignment length
    is greater than 30\% of the reference protein length and if 30\%
    of the aligned length have identical matches.}
  \label{table:tblastn}
\end{table}
