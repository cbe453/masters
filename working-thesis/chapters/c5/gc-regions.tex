\section{Genes in Regions of Anomalous GC Content}

Evaluating gene finder performance in regions of anomalous GC content
is one of the key topics of this research. One simple way to evaluate
performance is whether or not gene finding tools predict genes
uniformly throughout a given sequence. Biologically, we know that
regions of anomalous nucleotide composition are less likely to contain
coding sequences than typical genomic regions, leading us to the
problem of first identifying predicted genes in standard and anomalous
regions. After identifying low GC segments within each assembly, we
can include them in the region identification method. Counts from
region identification with the inclusion of anomolous GC content are
presented in table \ref{table:gc-regions}.

\begin{table}[h!]
  \begin{center}
    \begin{tabular}{|c|c|c|c|c|}
      \hline
      Assembly & Total Regions & Complete Agreement & Partial Agreement & Singletons \\ \hline
      DC1 & 11856 & 10 & N/A & 20  \\ \hline
      Tsth20 & 12507 & 2 & N/A & 9  \\ \hline
      \textit{T. reesei} & 10428 & 25 & 19 & 115  \\ \hline
      \textit{T. harzianum} & 14622 & 26 & 45 & 548  \\ \hline
      \textit{T. virens} & 12121 & 8 & 11 & 17  \\ \hline
    \end{tabular}
  \end{center}
  \caption[Agreement of predictions in anomalous GC regions.]{Total
    regions identified each assembly followed by counts of regions
    (both agreement and singletons) in regions of anomalous GC
    content.}
  \label{table:gc-regions}
\end{table}

To better understand the performance of each tool in identifying genes
in repetitive sequences, we will investigate the composition of the
partial and singleton regions. Table \ref{table:gc-tools} shows the
the number of genes identified by each tool in those regions.

%\begin{table}                                                                                                                                                                                                                                                                                                                                                            
%  \begin{center}                                                                                                                                                                                                                                                                                                                                                         

%  \end{center}                                                                                                                                                                                                                                                                                                                                                           
%\end{table}                                                                                                                                                                                                                                                                                                                                                              

From this result, we classified predicted genes into two classes;
genes in regions with normal GC content, and genes in regions with
anomalous content. In this case, anomalous content is defined as a
window of genomic sequence containing a percent GC composition of 28\%
or lower. This number was chosen based on the plots of GC content
presented in the assembly section of the results. After classifying
predicted genes, two-sided binomial tests were performed with the null
hypothesis being that predicted genes are distributed uniformly
throughout an assembly. Framed differently, we expect the sum of genes
predicted in both regular and irregular regions to be proportional to
the sum of lengths of those regions, respectively. This is not the
case as demonstrated in table~\ref{table:gc-binomial}.

\begin{table}[h!]
  \begin{center}
    \begin{tabular}{|c|c|c|c|c|c|}
      \hline
      Tool & DC1 & Tsth20 & \textit{T. reesei} & \textit{T. harzianum} & \textit{T. virens} \\ \hline
      Braker2 & $9.56^{-181}$ & $1.14^{-259}$ & $2.68^{-96}$ & $4.05^{-140}$ & $1.35^{-35}$ \\ \hline
      GeneMark & $5.12^{-216}$ & $0.0$ & $5.66^{-49}$ & $5.37^{-219}$ & $5.31^{-35}$ \\ \hline
      RefSeq & N/A & N/A & $1.29^{-49}$ & $2.44^{-205}$ & $7.40^{-33}$ \\ \hline
    \end{tabular}
  \end{center}
  \caption{\textit{p} values produced from a two-sided binomial test
    for each combination of tool and assembly.}
  \label{table:gc-binomial}
\end{table}
