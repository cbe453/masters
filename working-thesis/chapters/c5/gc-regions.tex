\section{Genes in Regions of Anomalous GC Content}
\label{section:gc-regions}

Table \ref{table:gc-regions} shows the results of applying the same
region finding process from earlier to segments of the genome
identified as AT-rich. AT-rich is defined as a region of genomic
sequence with percent GC composition less than 28\%. In comparison to
results from section \ref{section:regions}, we see that overall there
are far fewer regions with gene predictions in AT-rich genomic
segements. In DC1, Tsth20, there are very few regions with full
support from both Braker2 and GeneMark, but more singletons. Again, as
in section \ref{section:regions}, there are no regions with partial
support as only two gene finding tools were applied to those
assemblies. \textit{T. reesei} and \textit{T. harzianum} report more
regions in AT-rich genomic sequence than the other assemblies, but
with the majoirty of regions belonging to the singleton
category. \textit{T. virens} is an interesting case, reporting a
similar numbers of regions and genes in AT-rich genomic sequence as
DC1 and Tsth20. \textit{T. virens} is also the only assembly to report
zero singleton gene predictions. Why \textit{T. virens} differs from the
other RefSeq assemblies is unclear. In general, there are few regions
with gene predictions in AT-rich genomic segments, and within these
regions, the majority of cases are isolated singleton gene
predictions. These observations differ greatly from those made in
nucleotide composition agnostic approach in section
\ref{section:regions}.

\begin{table}
  \begin{center}
    \begin{tabular}{|c|c|c|c|c|}
      \hline
      Assembly & Full Support & Partial Support & Singletons & No. Genes \\ \hline
      DC1 & 11 & N/A & 20 & 42  \\ \hline
      Tsth20 & 2 & N/A & 9 & 13  \\ \hline
      \textit{T. reesei} & 25 & 18 & 54 & 194  \\ \hline
      \textit{T. harzianum} & 26 & 43 & 68 & 265  \\ \hline
      \textit{T. virens} & 8 & 11 & 0 & 49  \\ \hline
    \end{tabular}
  \end{center}
  \caption[Agreement of predictions in anomalous GC regions.]{Total
    number of regions identified in each assembly followed by counts
    of regions (both agreement and singletons) in regions of AT-rich
    genomic sequence.}
  \label{table:gc-regions}
\end{table}

In addition to a breakdown of regions in AT-rich genomic sequence,
understanding which gene finders predict more or fewer genes in these
regions may be of interest. Table \ref{table:gc-gene-counts} shows the
number of genes predicted by each gene finding tool in regions of
AT-rich genomic sequence. GeneMark appears to predict the fewest genes
in AT-rich regions, while RefSeq appears to predict the most. Braker2
lies somewhere in the middle. Again, \textit{T. virens} appears as an
odd case, with very few predictions from all gene finders. Why
\textit{T. virens} differs from the other RefSeq assemblies is
unclear.

\begin{table}
  \begin{center}
    \begin{tabular}{|c|c|c|c|}
      \hline
      Assembly & Braker2 & GeneMark & RefSeq \\ \hline
      DC1 & 31 & 11 & N/A \\ \hline
      Tsth20 & 11 & 2 & N/A \\ \hline
      \textit{T. reesei} & 39 & 48 & 107 \\ \hline
      \textit{T. harzianum} & 81 & 30 & 154 \\ \hline
      \textit{T.virens} & 21 & 8 & 20 \\ \hline
    \end{tabular}
  \end{center}
  \caption{Number of genes predicted by Braker2, GeneMark and RefSeq
    in AT-rich genomic sequence from each assembly.}
  \label{table:gc-gene-counts}
\end{table}

Finally, to test the probability of any given gene prediction falling
in an AT-rich genomic sequence, a two-sided binomial test was
performed to determine if the number of genes predicted in AT-rich
sequences is proportional to fraction of genomic sequence they
comprise. The results of the test are shown in table
\ref{table:gc-binomial}. In all cases, it appears that the gene
finding tools selected for this analysis do not predict the same
proportion of genes in AT-rich genomic sequence as in typical genomic
sequence.

\begin{table}
  \begin{center}
    \begin{tabular}{|c|c|c|c|c|c|}
      \hline
      Tool & DC1 & Tsth20 & \textit{T. reesei} & \textit{T. harzianum} & \textit{T. virens} \\ \hline
      Braker2 & $9.56^{-181}$ & $1.14^{-259}$ & $2.68^{-96}$ & $4.05^{-140}$ & $1.35^{-35}$ \\ \hline
      GeneMark & $5.12^{-216}$ & $0.0$ & $5.66^{-49}$ & $5.37^{-219}$ & $5.31^{-35}$ \\ \hline
      RefSeq & N/A & N/A & $1.29^{-49}$ & $2.44^{-205}$ & $7.40^{-33}$ \\ \hline
    \end{tabular}
  \end{center}
  \caption{\textit{p}-values produced from a two-sided binomial test
    for each combination of tool and assembly.}
  \label{table:gc-binomial}
\end{table}

In summary, very few regions with gene predictions are present in
AT-rich genomic sequence. Additionally, genes predicted in these
regions tend to be isolated and not supported by other gene finders,
although some agreement is observed. In terms of number of genes
predicted, RefSeq tends to predict the most genes in these AT-rich
regions while GeneMark predicts the fewest. Lastly, the selected gene
finding tools do not predict genes in AT-rich sequences in proportion
to the fraction of genomic sequence they comprise.
