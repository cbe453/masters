\section{Installation and Profiling Gene Finding Tools}

While Braker2, GeneMark and RefSeq all provide lists of possible genes
for a provided input, the implementation of each tool is different,
requiring more or less effort to install and run thn other tools. This
section discusses the implementations, installation procedures and
execution of GeneMark and Braker2 in more detail. For more information
on the versions and parameters used in this processing, please see
section \ref{met:braker2}. The computing platform used in this
research is hosted and managed by University of Saskatchewan services,
which is modelled around the HPC platform used by the Digital Research
Alliance of Canada and software is managed similarily.

GeneMark\cite{10.1093/nar/gki937} is a gene finding tool developed by
the Georgia Institute of Technology with packages prepared for Linux
and MacOS. It is provided as licensed product in the form of a package
which can be downloaded from their website after submitting a
form. Once the necessary information is submitted, the user is
provided with a key that must be placed in the appropriate location
once the software is downloaded and unpacked. The core controlling
methods of GeneMark are written in Perl, accompanied by several Python
scripts and compiled exectuables. GeneMark was tested by the
developers with Perl version 5.10, and Python 3.3. A number of Perl
dependencies are also required, which can be installed via CPAN. The
user will have to know which implementation of GeneMark they are
wanting to use for their application, as GeneMark has several
variations it can run depending on the desired application. In this
work, the GeneMark-ES variant of GeneMark was executed, as it is the
self-training \textit{ab initio} GeneMark method for eukaryotic
organisms. Options required by GeneMark at runtime are documented in
the help message, and simple enough that any user with familiarity of
bioinformatics tools should be able to run GeneMark, although
documentation for use is only provided by the help message when
running the program and not online. In regards to run-time, running
the GeneMark-ES pipeline on DC1 with 56 threads finished in 16
minutes. GeneMark's outputs consist of a GTF or GFF file of predicted
genes as well as a number of other outputs related to the run.

% CPAN here: https://www.cpan.org/modules/INSTALL.html

Braker2\cite{Bruna2021} is hosted on GitHub as a repository that
receives relatively frequent updates with Braker3 being released while
working before the completion of this work. Braker is maintained by
Katharina Hoff from the University of Greifswald and is available
under the Open Source Artistic License. Installation of the repository
is a straightforward pull from GitHub. As with GeneMark, Braker2 uses
a combination of Perl, Python and other exectuables in its regular
use. Downloading the repository itself is not enough for execution, as
Braker2 relies on a number of dependencies and bioinformatics tools
including Perl and (Perl dependencies), Augustus\cite{Stanke2006},
BamTools\cite{Barnett2011},
BedTools\cite{10.1093/bioinformatics/btq033},
GeneMark\cite{Borodovsky2011}, StringTie\cite{Pertea2015},
GFFRead\cite{Pertea2020} and several others. Manual installation of
these dependencies would be difficult, time consuming and in general
advised against. In this case, many of Braker2's requirements are
satisfied by modules already included in the environment, making
installation relatively simple if one is familiar with the Digital
Research Alliance of Canada's software stack. This case still required
installation of some Perl modules in addition to loading necessary
modules. Alternatively, one could use a package manager like Anaconda
or Minoconda to handle installation of packages. This is perfectly
reasonable, but also requires knowledge specific to Anaconda, which
can be complicated and frustrating for users with little software
management experience. Once installation is finished, the Braker2
pipeline is relatively straightforwad to run as well, with excellent
documentation included both online and through the built in help
message. The training, including RNAseq alignments, and gene finding
pipeline in this case took 1 hour and 17 minutes using 60
threads. Applying the Braker2 trained gene finding model to DC1 with
60 threads took 21 minutes to complete. Run times will of course vary
depending on processing power available to the end user, but in the
case of \textit{Trichoderma} genomes, users can expect quick results
with relatively little computing power. Once annotation is complete,
Braker2 produces a GFF file containing predicted genes along with CDS
sequences and amino acid sequences their protein products.

% Braker3 footnote...
% The newly released Braker3, also includes a
% containerized version of the software, allowing users to build and
% execute Braker3 with ease in its own environment. Can also touch on
% UTR stuff as well

In summation, Braker2 and GeneMark are not direct plug-and-play
software packages. Users should expect to encounter issues when
getting these programs running in addition to normal downloading and
unpacking of software packages so some expertise is
recommended. Neither Braker2 nor GeneMark require users to compile
software, however Braker2's dependencies may require additional
compilation and attention. Once installed, both tools are relatively
simple to use with documention available for both on the commandline
and excellent documentation available for Braker2 on their GitHub
page. Outputs from both tools are similar although Braker2 has the
ability to output coding and amino acid sequences for downstream
processing. Both tools run in reasonable amounts of time, where in the
case of smaller genomes such as \textit{Trichoderma}, users can expect
results within a few hours to a day depending on number of computing
cycles available to them.
