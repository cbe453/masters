\section{Relevant Literature}

\subsection{Gene Finding}
\subsubsection{Overview}
Gene finding (or gene annotation) has been a long standing
computational problem in bioinformatics, which concerns itself with
identifying potential genes within assemblies based on patterns or
pre-existing experimental evidence evidence considered by the gene
finding program. This process is critical for unraveling and
understanding the complex processes occurring in all forms of life
with applications in medical science, agriculture, biomanufacturing,
environmental studies and many others. In a general sense, gene
finding programs operate by searching for patters or indicators
showing that a gene of feature may be present. The most basic
indicators being start and stop codons, with introns and exons in
between should the sequence match the applied model. The results
produced by gene finding tools can vary considerably for a number of
reasons, including quality of the assembly, the intrinsic model used
by the gene finder, filtering criteria, and even the nature of the
organism and assembly itself. Given the broad applications, choice of
gene finding tools, and the variability of assemblies being
considered, it is important that we gain a deeper understanding of
these tools prior to putting them to use

\subsubsection{Gene Finding Methods}
There are two common methods for gene finding, those methods being
\textit{ab initio} methods, where programs search for patterns and
gene structures, and similarity or evidence-based searches, which use
prior information such as RNAseq data, expressed sequence tags and
expressd protein sequences to identify genes within a new
genome\cite{GeneFinding}. Complicating the process more is the
introduction of introns and alternative splicing in eukaryotes, making
it possible for one gene to have several possible transcripts at the
same locus. An example of an \textit{ab initio} method would be
GeneMark-ES\cite{GeneMarkES}, while an evidence based tool would be
Braker2\cite{Braker2}.

\textit{Ab initio} gene finders typically base predicted genes on a
Hidden Markov Model (HMM). 

There are also other aspects of gene finding tools that are important
to consider. These include features such as whether or not the gene
finders find non-coding RNAs, annotation of 5' and 3' UTR regions, and
in the case of ab-initio methods, the assumptions made by the
underlying models used for gene finding. These features and others can
influence a user's decision on which gene finding tool to consider and
will complicate comparative analysis of multiple gene finding tools.

\subsubsection{Trichoderma}

Crop resistance to environmental stressors is a necessity for crop
health and overall crop yields. Current popular methods for crop
protection involve the use of pesticides and genetically modified
organisms, which can be expensive and potentially politically dividing
in the case of GMOs\cite{GMO}. In addition, crops suffer when soils
are not sufficient for crop growth and health. Soil insufficiencies
can result in drought stress as well as nutrient stress, leading to
poor overall yields.

\textit{Trichoderma} is a type of fungi that can colonize the roots of
plants in a non-toxic, non-lethal, opportunistic symbiotic
relationship\cite{Trichoderma}. Many strains of \textit{Trichoderma}
have been shown to provide resistance to bacteria and other fungi in
soils through the use of polyketides, non-ribosomal peptide
synthetases and other antibiotic
products\cite{Trichoderma}\cite{Secretome}. Recently, two strains of
\textit{Trichoderma} have been identified in the prairie regions of
Alberta and Saskatchewan. These two strains, named Tsth20 and DC1,
have been found to have beneficial properties when used as an
inoculant for plants in the soils mentioned before. In addition to
these beneficial properties, the two strains mentioned previously
provide even further protection for plants in dry, salty soils and one
strain also has potential for use as a bioremediation tool in soils
contaminated with hydrocarbon content. Bioremediation and resistance
to drought tolerance has also been investigated in other strains of
\textit{Trichoderma} as well\cite{Drought}\cite{Kaminskyj}. However,
little is known about the mechanisms at work in these strains, so DC1
and Tsth20 were sequenced by the Global Institute for Food Security
(no publication yet) in an initial attempt to better understand the
details of these genomes. While this research does not directly
identify genomic elements related to the secretome of these genomes,
it may serve as a foundation for future research of
\textit{Trichoderma}.

\subsection{Genome Assembly}

Sequence assembly has been a long-standing application problem in the
field of bioinformatics\cite{assembly}. Determining the correct order
and combination of smaller subsequences into an accurate complete
sequence assembly is computationally difficult in terms of compute
resources such as memory, CPU cycles and storage required for input
sequences\cite{assembly}. In addition to these difficulties, there can
be other issues encountered during asssembly due to the nature of the
data or genomes themselves, such as low quality base calls for long
read data or the inherent content of genomes themselves using
repetitive regions as an example. Insufficient data used in an
assembly may result in short, fragmented assemblies, depending on the
size of the genomes, while sequence data that is not long enough can
fail to fully capture repetitive regions in an assembly. To solve this
problem, a wide range of assembly tools have been developed with their
own unique approaches to the genome assembly problem, so it is
important to use an appropriate assembler for the task at hand, and
also important to evaluate the assembly thoroughly. One approach to
aid in the previously mentioned issue of assembly correctness is to
use a combination of long and short reads in what is known as a hybrid
assembly. Combining both highly accurate short reads with deep
coverage along with less accurate but much longer reads can produce
high quality genome assemblies that capture long repetitive
regions. Hybrid assembly approaches have been shown to produce high
quality assemblies in a wide variety of organisms as the combine long
read data with short data to produce assemblies that properly
represent long repetetive regions with additionaly high quality
Illumina sequences for correction. Once assembled, the sequences must
also be evaluated with measures such as N50, L50, coverage, average
contig length and total assembled length to ensure that the genomes
are well assembled, at least based on these
metrics\cite{assembly}. Following appropriate assembly protocols is
essential to the further success of a project as downstream processing
such as annotation depends on a high-quality assembly.

\subsection{Repeat Identification/Masking and Identification of AT-rich Genomic regions}
Repeat identification within assembled genomes is a problem that needs
to be considered during the genome annotation process. Regions with
long repeats can have a significant impact on genome assembly as well
as gene finding due to the limitation of short reads used in some
assemblies\cite{Repeats}. Short reads may be unable to bridge or cover
entire repeat regions within a genome, so it is important to consider
the use of long reads from technologies such as Nanopore or PacBio to
provide a complete picture of these regions when pursuing a new genome
assembly project. It is also possible for repetitive regions to
contain genes as well, making for an interesting investigation in
regards to \textit{Trichoderma}, as fungal genomes have been shown to
contain many repeat regions with a high concentration of A and T
nucleotides\cite{fungalrepeats}. Once these repetitive regions have
been identified, the genome could be masked to exlude these regions in
downstream processing if desired, as these regions may be poorly
assembled and may result in found genes that do not truly exist in
those regions. However, this may not be as common today, as repetetive
regions have been shown to contain genes as well\cite{dontMask}. This
may affect the gene finding process described later and may be an
interesting topic to look into considering the large number of
available gene finding programs.
