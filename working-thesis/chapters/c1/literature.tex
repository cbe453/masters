\section{Genomics}
Genomics is a wide area of study focusing on the genomes of organisms
from all varieties of life. A genome is the fundamental set of 'rules'
used to create what we know as life. One can think of a genome as a
set of instructions that our cells use in order to complete the tasks
that make us function. A genome is comprised of tightly bundled
sequences of DNA, which are stored in the nucleus of cells. These
bundles of DNA contain sections known as genes, which can be thought
of as the tools described by the set of instructions. These tools
carry out a vast number of processes ranging from no known function at
all to genes that are key in protecting against diseases
cancer. Genomes can vary widely in size, ranging from small bacterial
genomes of roughly 4 Mb up to approximately 149000 Mb. Piecing
together genomes provides numerous opportunities to understand other
'omics' within cells, such as proteomics, metabolomics,
transcriptomics and epigenomics.

\subsection{Sequencing}
Sequencing data is a pivotal form of data used in nearly all
applications of Bioinformatics. To understand the processes used by
organisms for day to day survival or in unique circumstances, we must
have an initial set of data points to work with. These sequences,
referred to as reads after sequencing, are the foundation for solving
problems ranging from taxonomical classification to the understanding
or complex biological functions like signaling pathways.

\section{Whole Genome Shotgun Sequencing}
Whole Genome Shotgun sequencing (WGS), is a method to produce a large
number of genomic sequences from a sample of interest. This is form of
sequencing is quite common as it is has a wide variety of applications
in research. WGS involves slicing up genomic DNA into smaller
segments. These small segments are then processed further resulting in
a set of physical molecules that can be supplied to a compatible
sequencing platform of which there are a variety. Modern equencing
platforms are typically broken up into next generation sequencing
(NGS) and 3rd generation sequencing.

\section{Next Generation Sequencing - Illumina}
Illumina sequencing is one of the most popular NGS platforms currently
available. Illumina sequencing produces a very large number of high
quality short reads, typically between 75 and 250 base pairs in
length. Sequencing libraries can be prepared to produce reads solely
from one end of a sequence fragment (single-end) or both ends
(paired-end). Advantages of paired end sequences are the additional
context provided by the paired sequence on the opposite end of the
fragment. This context is leveraged by read processing tools to
identify features such as repetitive regions and genomic
rearrangments, which can be significant in downstream
analyses. Illumina sequence librairies are generated by first
fragmenting the DNA samples, amplifying them via PCR, ligating
adapters that allow the sequence to bind to the sequencing plate, and
finally identifying each fragment's sequence of nucleotides using
fluorescently-labeled nucleotides that bind to the fragments.

\section{3rd Generation Sequencing - Nanopore}

Nanopore sequencing data is relatively recent approach to sequencing
projects. While Illumina reads are considered to be short, Nanopore
reads are much larger, ranging from 10Kb to 300Kb depending on the
approach used. Long reads are beneficial due to their ability to
bridge the gaps between difficult to assemble regions when performing
sequence assembly. An example of a difficult to assemble region would
be a region with a high repeat content, where a large number of small
repeats may be collapsed during the assembly process, resulting in an
assembly that does not represent the true nature of the sequence being
studied. Whlie Nanopore was previously known for having lower quality
base calls when compared to Illumina, that is no longer the case at
this time. Nanopore sequencing works by passing long seqments of
genetic sequence through a membrane bound protein and measuring
changes in electrical current, which is characteristic of the
nucleotide at a given position.

\section{Trichoderma}

Crop resistance to environmental stressors is a necessity for crop
health and overall crop yields. Current popular methods for crop
protection involve the use of pesticides and genetically modified
organisms, which can be expensive and potentially politically dividing
in the case of GMOs\cite{doi:10.1080/10408390600762696}. In addition,
crops suffer when soils are not sufficient for crop growth and
health. Soil insufficiencies can result in drought stress as well as
nutrient stress, leading to poor overall yields.

\textit{Trichoderma} is a fungi that can both communicate with and
colonize the roots of plants in a non-toxic, non-lethal, opportunistic
symbiotic relationship\cite{Woo2023}. Many strains of
\textit{Trichoderma} have been shown to provide resistance to
pathogenic bacteria and other fungi in soils through the use of
polyketides, non-ribosomal peptide synthetases and other antibiotic
products\cite{Woo2023}. Recently, two strains of \textit{Trichoderma}
have been identified in the prairie regions of Alberta and
Saskatchewan. These two strains, named Tsth20 and DC1, have been found
to have beneficial properties when used as an inoculant for plants in
the soils mentioned before. In addition to these beneficial
properties, the two strains mentioned previously provide even further
protection for plants in dry, salty soils and one strain also has
potential for use as a bioremediation tool in soils contaminated with
hydrocarbon content. Bioremediation and resistance to drought
tolerance has also been investigated in other strains of
\textit{Trichoderma} as well\cite{10.3389/fpls.2023.1190304}. However,
little is known about the mechanisms at work in these strains, so DC1
and Tsth20 were sequenced by the Global Institute for Food Security
(no publication yet) in an initial attempt to better understand the
details of these genomes. While this research does not directly
identify genomic elements related to the secretome of these genomes,
it may serve as a foundation for future research of
\textit{Trichoderma}.

\section{Genome Assembly}

Sequence assembly has been a long-standing problem in the field of
bioinformatics\cite{Nagarajan2013}. Determining the correct order and
combination of smaller subsequences into an accurate complete sequence
assembly is computationally difficult in terms of compute resources
such as memory, CPU cycles and storage required for input
sequences\cite{Nagarajan2013}. In addition to these difficulties,
there can be other issues encountered during asssembly due to the
nature of the data or genomes themselves, such as low quality base
calls for long read data, which is not necessarily the case today, or
the inherent content of genomes themselves using repetitive regions as
an example. Insufficient data used in an assembly may result in short,
fragmented assemblies, depending on the size of the genomes, while
sequence data that is not long enough can fail to fully capture
repetitive regions in an assembly. To solve this problem, a wide range
of assembly tools have been developed with their own unique approaches
to the genome assembly problem, so it is important to use an
appropriate assembler for the task at hand, and also important to
evaluate the assembly thoroughly.

Genome assembly tools generally approach the assembly problem using a
graph-based approach. The most common graph-based approach is the de
Bruijn graph assembly. A graph in this context, is set of nodes
(\textit{k}-mers from sequences) connected by edges (overlaps between
\textit{k}-mers). Traversing through this graph results in longer
subsequences that ultimately result in a set of consensus sequences
and final assembly. In the early years of long read sequence data,
sequencing platforms encountered difficulties producing consistently
high scores for base calls when seuqncing. To combat this, some
assembly workflows may also include a polishing or correction step
once the initial assembly is completed in which high quality short
read sequences are supplied as supplemental information to correct low
quality base calls in the assembly. These low quality base calls are
typically not present in modern long read sequencing approaches as the
methodology and quality of calls have improved drastically. While the
polishing step is arguably unnecessary in modern assemblies, the
polishing programs remain available should researchers be interested
in applying additional reads for polishing. 

One approach to aid in the previously mentioned issue of assembly
correctness is to use a combination of long and short reads in what is
known as a hybrid assembly. Combining both highly accurate short reads
with deep coverage along with less accurate but much longer reads can
produce high quality genome assemblies that capture long repetitive
regions. Hybrid assembly approaches have been shown to produce high
quality assemblies in a wide variety of organisms as the combine long
read data with short data to produce assemblies that properly
represent long repetetive regions with additionaly high quality
Illumina sequences for correction. Once assembled, the sequences must
also be evaluated with measures such as N50, L50, coverage, average
contig length and total assembled length to ensure that the genomes
are well assembled, at least based on these
metrics\cite{Nagarajan2013}. Following appropriate assembly protocols
is essential to the further success of a project as downstream
processing such as annotation depends on a high-quality assembly.

\section{Identification of AT-rich Genomic regions}
One important aspect of interest when assembling any form of sequence
is GC content or percent GC of the assembled sequence. Large regions
of anomolous GC content may be of interest to researchers as they may
contain repetitive regions and unique features responsible for traits
specific to the organism in question.

\section{Repeat Identification and Masking}
Repeat identification within assembled genomes is a problem that needs
to be considered during the genome annotation process. Regions with
long repeats can have a significant impact on genome assembly as well
as gene finding due to the limitation of short reads used in some
assemblies\cite{Treangen2011}. Short reads may be unable to bridge or
cover entire repeat regions within a genome, so it is important to
consider the use of long reads from technologies such as Nanopore or
PacBio to provide a complete picture of these regions when pursuing a
new genome assembly project. It is also possible for repetitive
regions to contain genes as well, making for an interesting
investigation in regards to \textit{Trichoderma}, as fungal genomes
have been shown to contain many repeat regions with a high
concentration of A and T
nucleotides\cite{10.1371/journal.pgen.1007467}. Once these repetitive
regions have been identified, the genome could be masked to exlude
these regions in downstream processing if desired, as these regions
may be poorly assembled and may result in found genes that do not
truly exist in those regions. However, this may not be as common
today, as repetetive regions have been shown to contain genes as
well\cite{Slotkin2018}. This may affect the gene finding process
described later and may be an interesting topic to look into
considering the large number of available gene finding programs.

\section{Gene Finding Methods}
Gene finding (or gene annotation) has been a long standing
computational problem in bioinformatics, which concerns itself with
identifying potential genes within assemblies based on patterns or
pre-existing experimental evidence evidence considered by the gene
finding program. This process is critical for unraveling and
understanding the complex processes occurring in all forms of life
with applications in medical science, agriculture, biomanufacturing,
environmental studies and many others. In a general sense, gene
finding programs operate by searching for patters or indicators
showing that a gene of feature may be present. The most basic
indicators being start and stop codons, with introns and exons in
between should the sequence match the applied model. The results
produced by gene finding tools can vary considerably for a number of
reasons, including quality of the assembly, the intrinsic model used
by the gene finder, filtering criteria, and even the nature of the
organism and assembly itself. Given the broad applications, choice of
gene finding tools, and the variability of assemblies being
considered, it is important that we gain a deeper understanding of
these tools prior to putting them to use.

There are two common methods for gene finding, those methods being
\textit{ab initio} methods, where programs search for patterns and
gene structures, and similarity or evidence-based searches, which use
prior information such as RNAseq data, expressed sequence tags and
expressd protein sequences to identify genes within a new
genome\cite{Ejigu2020}. Complicating the process more is the
introduction of introns and alternative splicing in eukaryotes, making
it possible for one gene to have several possible transcripts at the
same locus. An example of an \textit{ab initio} method would be
GeneMark-ES\cite{10.1093/nar/gki937}, while an evidence based tool
would be Braker2\cite{Bruna2021}.
\textit{Ab initio} gene finders typically predict genes using a Hidden
Markov Model (HMM)\cite{Ejigu2020}. These predictions are based on
'signals' or features associated with a gene, such as the usual start,
stop, exon and intron portions of a gene as well as upstream promoter
sequences and more. In this case, these signals would be considered
states in the terminology associated with HMMs. Gene finders wish to
predict these states based on observations, or sequences presented to
the model. HMMs in gene finding tools are trained beforehand and then
applied to a sequence. This means that a gene finding program may not
be trained in the context of any assembly provided to it, and thus may
miss genes that are unique to the assembly in question.
On the other hand, while still relying on HMMs for a 'base' set of
predictions, evidence-based gene finding tools leverage new evidence
that may be outside the scope of the pre-existing
model\cite{Keller2011}.  As an example, an evidence-based model would
be useful in a situation where you are interested in annotating a new
assembly for a non-model organism. The addition of experimental data
provides context specific to your assembly of interest while still
retaining the predictions from existing HMM models.

There are also other aspects of gene finding tools that are important
to consider. These include features such as whether or not the gene
finders find non-coding RNAs, annotation of 5' and 3' UTR regions, and
in the case of ab-initio methods, the assumptions made by the
underlying models used for gene finding. These features and others can
influence a user's decision on which gene finding tool to consider and
will complicate comparative analysis of multiple gene finding
tools. (citation needed somewhere in here)

\section{File Formats}

\subsection{FASTA}
Sequences used in the asembly and gene finding process
