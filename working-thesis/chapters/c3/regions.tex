\section{Region Identification}

Based on the methodology described earlier, we have identified
sections of contigs with one or more overlapping predicted sequences
(features). The result of this processing is a GFF formatted file with
each entry containing information about a region on a specific
contig. Also contained in each entry, is a list of all features
included in that region with start/stop position and strand, the tools
used to predict the features, and a count of how many prediction tools
that were included in said region. Initial analysis of the output is
discussed in this section.

With our results, the first interesting area to look at is simply the
total number of regions identified for any given assembly as well as
number of regions that each tool was a member(weird wording) of. In
figure~\ref{regioncounts}, we see that Braker2 deviates from both the
RefSeq and GeneMark predictions greatly, except in the case of
\textit{T.reesei}. As speculated earlier, this may be due to the
training dataset supplied to Braker2 as it happens to be present in a
similar number of regions when compared to RefSeq and GeneMark in
\textit{T. reesei}. In general, it appears that the RefSeq predictions
are present in more regions than both Braker2 and GeneMark. 

\begin{table}
  \begin{center}
    \begin{tabular}{|c|c|c|c|c|c|c|}
      \hline
      Assembly & Regions (total) & RefSeq & Braker2 & GeneMark \\ \hline
      DC1 & 11268 & NA & 8485 & 11265 \\ \hline
      Tsth20 & 12285 & NA & 8747 & 12279 \\ \hline
      \textit{T. harzianum} & 13377 & 13122 & 8051 & 11529 \\ \hline
      \textit{T. virens} & 12045 & 11864 & 7554 & 11413 \\ \hline
      \textit{T. reesei} & 9818 & 8836 & 9202 & 8897\\ \hline
    \end{tabular}
  \end{center}
  \caption{Counts of regions identified in total and total number of
    regions where a prediction from each individual tool was found.}
  \label{regioncounts}
\end{table}

Figure~\ref{regioncounts} demonstrates that the composition of regions
can differ significantly based on the tools used. To illustrate
agreement of tools in regards to identified regions, Venn diagrams
were generated showing agreement of predictions in a region for each tool.


\section{Genes in Regions of Anomalous GC Content}

Evaluating gene finder performance in regions of anomalous GC content
is one of the key topics of this research. One simple way to evaluate
performance is whether or not gene finding tools predict genes
uniformly throughout a given sequence. Biologically, we know that
regions of anomalous nucleotide composition are less likely to contain
coding sequences than typical genomic regions, leading us to the
problem of first identifying predicted genes in standard and anomalous
regions. After identifying low GC segments within each assembly, we
can include them in the region identification method. From this
result, we classified predicted genes into two classes; genes in
regions with normal GC content, and genes in regions with anomalous
content. In this case, anomalous content is defined as a window of
genomic sequence containing a percent GC composition of 28\% or
lower. This number was chosen based on the plots of GC content
presented in the assembly section of the results. After classifying
predicted genes, two-sided binomial tests were performed with the null
hypothesis being that predicted genes are distributed uniformly
throughout an assembly. Framed differently, we expect the sum of genes
predicted in both regular and irregular regions to be proportional to
the sum of lengths of those regions, respectively. This is not the
case as demonstrated in table~\ref{table:gc-binomial}.

\begin{table}
  \begin{center}
    \begin{tabular}{|c|c|c|c|c|c|}
      \hline
      Tool & DC1 & Tsth20 & \textit{T. reesei} & \textit{T. harzianum} & \textit{T. virens} \\ \hline
      Braker2 & $9.56^-181$ & $1.14^-259$ & $2.68^-96$ & $4.05^-140$ & $1.35^-35$ \\ \hline
      GeneMark & $5.12^-216$ & $0.0$ & $5.66^-49$ & $5.37^-219$ & $5.31^-35$ \\ \hline
      RefSeq & N/A & N/A & $1.29^-49$ & $2.44^-205$ & $7.40^-33$ \\ \hline
    \end{tabular}
  \end{center}
  \caption{\textit{p} values produced from a two-sided binomial test
    for each combination of tool and assembly.}
  \label{table:gc-binomial}
\end{table}
