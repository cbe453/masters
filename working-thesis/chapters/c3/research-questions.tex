\section{Research Questions}

With an ever-increasing number of gene prediction tools available to
users, it is important to assess and understand their behaviour and
performance in the context, particularly in the context of new genome
assemblies of lesser studied organisms, where a reference prediction
set may not be available. To assess behaviour and performance in these
cases, we have defined X problems to profile the selected gene finding
tools. In addition to applying gene finders to existing assemblies of
\textit{Trichoderma}, we will also be considering two newly assembled
\textit{Trichoderma} genomes for which we will be including assembly
metrics in the results as well.

\section{\textit{Trichoderma} Assembly Results}

The main purpose of this research is to evaluate and compare gene
finding tools in the context of \textit{Trichoderma} assemblies where
a gold standard set of gene predictions does not exist. While we do
include genome assemblies from NCBI and their RefSeq annotation, we
will also be applying gene finding tools to two newly assembled
\textit{Trichoderma} assemblies to see if observations from reference
sets extend to the new assemblies. Since gene finding tools operate on
an assembled genomic sequence, it must follow that the results will be
influenced by the supplied assembly. With that said, before applying
gene finders to the new assemblies, we should first investigate the
new assemblies, by generating general assembly metrics for the new
assemblies to contrast and compare with existing assemblies. These
metrics include, total assembly length, number of contigs, N50, L50,
GC content and repeat content. With these assemblies being
\textit{Trichoderma}, we expect assembly metrics to be similar in
nature to existing assemblies from NCBI.

\section{Number of Features Predicted}

One common method for evaluating gene finding tools is by looking at
the number of features predicted by each tool. The term 'feature' here
is somewhat ambiguous, referring to many possible categories of
genomic feature. This makes an obvious point of comparison for
selected gene finding tools. We will identify the number of features
predicted by each gene finding tool and perform tests of statistical
significance to see if different gene finding tools predict different
numbers of features. This workflow will be applied to several
different classes of features. To test the hypothesis that tools
predict the same number of features, a Chi2 goodness of fit test will
be applied to the following categories of feature with the null
hypothesis being that gene finding tools predict the same number of
features.

\subsection{Gene Prediction}

The main feature predicted by gene finding tools is a gene. This class
of feature on it's own marks the presence of a potential gene along
with it's start and stop points. Identifying and comparing the number
of genes predicted by each gene finding tool is a good general metric
for for predicitive performance with general agreement that more is
better in most cases.

\subsection{Transcript Prediction}

While genes mark the general presence of a gene or gene product,
transcripts mark a more detailed structure within a gene. Genes may
contain more than one transcript, so the ability of a gene finding
tool to identify multiple transcripts for potential genes can be
considered a good performance metric to compare between tools.

\subsection{Coding Sequences}

From transcripts, we identify coding sequences. The number of coding
sequences is another important metric that we can compare between gene
finding tools.

\section{Lengths of Predicted Genes}

Genome assemblies can contain a wide range of gene lengths. For some
users, genes of a specific length may be a key point of interest, so
the ability of a gene prediction tool to capture the broad range of
possible gene lengths is another important metric for comparison. The
first statistical tool to be applied is ANOVA (analysis of variance)
to compare the mean lengths genes predicted by each gene finding tool
with the null hypothesis being that the mean of predicted gene lengths
should be the same across all tools considered. In addition to ANOVA,
pairwise comparisons of the distributions using a Kolmogorov–Smirnov
test is appropriate. The null hypothesis in this test would be that
the gene lengths are sample from the same distribution.

\section{Performance in A-typical Genomic Sequence}

One of the inspirations for this research is the unique composition of
genomic sequence in \textit{Trichoderma}. Results from the assembly
process show that GC content in \textit{Trichoderma} strains is
abnormal throughout most assemblies. These regions of assemblies
present an interesting opportunity to assess gene finding performance
in regions of anomalous GC content. To do this, a binomial test will
be used, with the null hypothesis being that the number of genes
predicted in regions of normal and abnormal GC content should be
proportional to the length of normal and abnormal GC content regions
in the assembly. For example, if 30 percent of the genome is comprised
of anomalous GC content, then we would expect 30 percent of predicted
genes to be present in those regions. In addition to anomalous GC
content, this test can be applied to repetitive content in assemblies
as well.

\section{BUSCO Completeness}

As more and more genomes are assembled for new organisms, a tool was
developed to evaulate assemblies and subsequent annotations from the
pserpective of gene orthology. As genomes diverge evolutionarily, it
is expected that some genes will be conserved. The BUSCO (Benchmarking
Universal Single-Copy Orthologs) tool and datasets were developed to
assess completeness of an annotation in comparison to evolutionarily
conserved genes. The BUSCO method was applied using two BUSCO subsets,
one generally applicable for fungi, and another targeting an
evolutionary branch more closely related to \textit{Trichoderma}.


\section{Identifying Regions of Agreement and Disagreement}

With predicted genes from several tools available, the question we
would like to ask is whether or not the gene finders agree with one
another for any given prediction. To answer this question, we will
identify 'regions' of overlapping predictions. A region can be defined
as a start and stop position of a set of individual or overlapping
features from one or more gene finding tools and external
sources. With regions identified, we can determine agreement, or more
importantly, disagreement in predictions between gene finding
tools. From these results, Venn diagrams will be generated with
Jaccard index calculated for each combination of gene finding
tools. The region identification process can also be extended to
include features identified by other tools, such as BLAST hits to
validated gene models from other organisms and small RNAs. Chi2
goodness of fit tests can then be applied to counts of 'validated'
gene predictions or other features.
