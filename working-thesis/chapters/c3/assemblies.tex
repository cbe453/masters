\section{Assemblies of DC1 and Tsth20}

From figure~\ref{assemblies}, we can make several observations, the
first of which is total contig counts for each assembly. For DC1 and
Tsth20, the total contig counts are an order of magnitude smaller when
compared to the other NCBI RefSeq assemblies, inidicating highly
contiguous assemblies from nextDenovo and nextPolish. This is likely
due to the use of long-read sequencing used in the assemblies of DC1
and Tsth20. For all assemblies, the total assembly lengths are
similar, hovering around the 38-42Mb range, except for
\textit{T. reesei}, which is known to have a significantly smaller
genome length (ref) at roughly 33Mb. The largest contig size for each
assembly vary greatly. DC1 and Tsth20 have the largest contigs of all
assemblies being considered, which is again likely due to the
inclusion of long-read sequencing data in the assembly process. The
N50 values for all assemblies are above 1Mb, with DC1 and Tsth20 N50s
being at minimum three times larger than others assemblies.

\begin{figure}
  \centering
  \makebox[0.8\textwidth][c]{
    \begin{tabular}{|c|c|c|c|c|c|c|}
      \hline
      Strain & Total Contigs & Total Length & Largest Contig & GC\% & N50 & L50 \\ \hline
      DC1 & 8 & 38.6 Mb & 11.49 Mb & 47.97 & 5.69 Mb & 3 \\ \hline
      Tsth20 & 7 & 41.58 Mb & 8.02 Mb & 47.33 & 6.52 Mb & 3 \\ \hline
      \textit{T. harzianum} & 532 & 40.98 Mb & 4.08 Mb & 47.61 & 2.41 Mb & 7 \\ \hline
      \textit{T. virens} & 93 & 39.02 Mb & 3.45 Mb & 49.25 & 1.83 Mb & 8 \\ \hline
      \textit{T. reesei} & 77 & 33.39 Mb & 3.75 Mb & 52.82 & 1.21 Mb & 9 \\ \hline
    \end{tabular}
  }
  \caption{General assembly metrics produced by QUAST (a
    genome quality assement tool).}
  \label{assemblies}
\end{figure}
