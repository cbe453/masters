\section{BUSCO Analysis}

The results from BUSCO analysis of the gene sets produced by each
prediction method prove promising. With all gene sets being 99.2\%
complete or higher based on the fungal dataset provided to BUSCO. This
higher number indicates that these gene finding tools capture nearly
completely the set of evolutionarily conserved single-copy orthologs
pre-defined by BUSCO curators. In the case of the fungal dataset,
there are 758 genes considered during analysis.

The most glaring observation from this analysis is the duplication
level found in the gene sets produced by Braker2 in comparison to the
GeneMark and RefSeq datasets. The Braker gene sets show a single-copy
match for roughly 80-85\% of the total gene call set with duplicates
making up the other 15-20\% of the set. The likliest reason for this
difference is the presence of isoforms in gene sets produced by
Braker. As shown in figure \ref{fig:genecounts}, Braker2 does produce
isoforms in its output, which would be identified as duplicates in
the BUSCO process. However, the number of isoforms predicted per gene
does not make up 20\% of the total set of gene calls. It is possible
that the BUSCO dataset contains a large fraction of the Braker2 gene
calls with isoforms, but that is unlikely and warrants further
investigation into the genes matching the BUSCO dataset.  Another
possible explanation is that Braker2 is predicting genes that are
actually isoforms as separate genes. BUSCO also makes note of isoforms
of a gene being the cause of high number of duplicates, and recommends
users remove isoforms prior to running BUSCO, although that step was
not performed for this work.



