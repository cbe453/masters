\section{Assemblies of DC1 and Tsth20}

Overall, the assemblies of DC1 and Tsth20 resulted in what can be
described as sequences suitable for further downstream
processing. Assembled lengths for both samples are similar to closely
related \textit{Trichoderma} accessions, indicating that the
assemblies are of appropriate length, but this could be confirmed with
further wetlab experiments in the future. The assembly of DC1 resulted
in eight contigs, with two sequences, ctg000040 being significantly
shorter at 65Kb than it's counterparts, which deviates from the
expected 7 chromosome-scale contigs expected from a
\textit{Trichoderma} assembly\cite{Kubicek2019}. Why this sequence is
significantly shorter than other contigs is unclear, but could arise
for several reasons. The most obvious reason would be mis-assembly,
which could arise from complications caused by both the anomolous GC
content reginos and highly repetitive regions. Even with the
inclusion of Nanopore sequencing data, it is possible that there was
insufficient support for connection between these contigs and
others. This highlights the complexity of assembling sequences with
anomolous content and supports the idea that multiple rounds and forms
of sequencing data may be required to produce scaffold and reference
level assemblies of non-model organisms. Another possible explanation
for an unaccounted for contig could be the assembly of a mitochondrial
genome, although this appears somewhat unlikely due to the length of
the ctg000040, it is not entirely out of the question. Previous work
focusing on mitochondrial genomes in \textit{Trichoderma} shows the
lengths of mitochondrial assemblies from six \textit{Trichoderma}
samples to be between 27Kb and 42Kb.

The assembly of Tsth20 resulted in seven contigs, matching that of the
expected seven chromosomes detected in most \textit{Trichoderma}
strains. The total assembled length of Tsth20 is again similar but
slightly larger than existing \textit{Trichoderma} assemblies.
