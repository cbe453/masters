\section{Selection of Gene Finding Tool}\label{chapter:conclusion}

With all of these results, it makes sense to explore the question of
which gene finding tool one should choose for optimal gene prediction
performance. Comparisons drawn in this section are made in the context
of \textit{Trichoderma} assemblies and may not extend to other
datasets. Observations from the results portion of this work have been
converted to scores for each gene finder, relative to the other gene finders. Scores range
from zero to three, with values being: 0 \- failing performance, 1 \-
passing performance, 2 \- good performance, and 3 \- excellent
performance. Results from this work are summarized in
Table~\ref{table:final-score}. Ignoring availability and use of the
gene finding tools, it would appear that RefSeq performs the best in
the remaining categories, earning top marks in every category except
in its ability to predict very short genes. Braker2 earns second
place; however, this does not capture Braker2's failure in predicting
accurate numbers of genes in DC1, Tsth20, \textit{T. harzianum} and
\textit{T. virens}. GeneMark comes in last, excelling only in number
of genes predicted and Pfam support for the genes that it predicts. We do note that these criteria are somewhat arbitrary and subjective in some cases, however they do provide a useful framework for comparing the gene finders, and are based on objective results presented in this work. 

Relating these observations to use-case scenarios, in the case a researcher's organism of interest has a RefSeq annotation associated with it,
the RefSeq gene prediction process appears to produce the best set of
predictions. If users also have experimental evidence, such as RNAseq
data under experimental conditions, it may be worthwhile training a
Braker2 model and predicting genes with Braker2 to supplement the
already well performing RefSeq gene predictions. We note that users should take the quality of the RefSeq assembly into account and investigate the input data and assembly protocols to ensure high-quality predictions. In a similar case, if
the organism is of RefSeq status, but the assembly in question is
novel to the work being performed, training and prediction using a
Braker2 model with experimental evidence either from the novel genome
or from the RefSeq accession in addition to use of the RefSeq
annotation would likely be the best option. If the organism of
interest is not a RefSeq individual but training data is available for
that organism, Braker2 is the next best option, although it is
important to note that the application of a trained Braker2 prediction
model to an organism from which the training data did not originate is
not advised based on the results presented in this work (see~\ref{section:gene-finding}). While it is true that the
\textit{T. reesei} genome differs from other \textit{Trichoderma}
genomes, it's status as a representative RefSeq organism makes it
somewhat of a gold standard. In this case, applying a gene model
trained using evidence from the gold standard produces biased numbers
of genes predicted in other \textit{Trichoderma} genomes. While
Braker2 technically scores the second highest, users must be very
careful when selecting training data, and ensure that the training
data either comes from the organism of interest, or comes from a very
closely related organism with a highly similar genome. In the case
that no appropriate training data is available, GeneMark is still an
option, and users can be confident that the tool predicts a reasonably
accurate number of genes with supporting Pfam matches. It is also
important to note that GeneMark does not perform as well in AT-rich
regions as Braker2 and RefSeq, does not predict isoforms, and
systematically fails to predict some BUSCO orthologs.

When availability of a gene finding tool becomes a concern and RefSeq
is not considered, the scores drop significantly for Braker2 and
GeneMark as seen in the final row of Table~\ref{table:final-score}. In
this situation, Braker2 still outperforms GeneMark. If the organism of
interest is not considered a representative RefSeq individual, but
supporting evidence specific to that organism or a very closely
related organism is available, a trained Braker2 prediction model will
perform well. Again, in the case that the organism is not a RefSeq
individual and no appropriate training data is available, GeneMark is
still a reasonable option even with the previously identified caveats.

\begin{table}
  \centering
  \begin{tabular}{|c|c|c|c|}
    \hline
    Category & Braker2 & GeneMark & RefSeq \\ \hline
    Availability & 3 & 3 & 0 \\ \hline
    Ease of install & 1 & 2 & 0 \\ \hline
    Ease of use & 3 & 3 & 0 \\ \hdashline
    \makecell{\# of genes\\predicted} & 0 & 3 & 3 \\ \hline
    \makecell{\# of transcripts\\predicted} & 3 & 0 & 2 \\ \hline
    \makecell{Predicts shortest\\genes} & 2 & 1 & 0 \\ \hline
    \makecell{Predicts more\\shorter genes} & 1 & 0 & 3 \\ \hline
    BUSCO Performance & 2 & 1 & 3 \\ \hline
    \makecell{Performance in\\AT-rich sequence} & 2 & 1 & 3 \\ \hline
    \makecell{Predictions with \\InterProScan support} & 3 & 3 & 3 \\ \hline
    \makecell{Final Score\\(Publicly Available)} & 20 & 17 & N/A \\ \hline
    \makecell{Final Score\\(Ignoring Availability)} & 13 & 9 & 17 \\ \hline
  \end{tabular}
  \caption[Final scoring table]{Table with scores attributed to
    performance of each gene finder in several categories. The score
    definitions for performance are as follows: 0 \- fail, 1 \- pass, 2
    \- good, 3 \- excellent. Since RefSeq is not publicly available, it is
    marked as N/A in the publicly available final scores. The dashed
    line separates categories associated with operation and use
    from categories describing gene prediction performance.}\label{table:final-score}
\end{table}

In summary, these results indicate that if your organism is a RefSeq
organism, use the RefSeq annotation. If no RefSeq predictions are
available, but appropriate training data is, one should use Braker2. If
the training data is of questionable similarity or not available at
all, users can fall back on \textit{ab initio} gene finders such as
GeneMark, which while not ideal, still predict genes with supporting
evidence.

\section{Future Work}

While this work is extensive, there are still many areas that could be
expanded on. The sheer number of gene finding tools available means
that many more could be compared to the ones presented here. In
addition, supplying more training data to Braker2 may improve gene
finding performance, and training on RNAseq from different organisms
could be used to improve the performance of Braker2. With more data in mind, different types of training data such as RNAseq,
expressed sequence tags (ESTs), and protein sequences could also be
used to train models that support external evidence. Another
limitation of this work is that the gene finders were run with default
parameters, and it is possible that tuning the parameters of the gene
finders could improve performance.

Following the gene prediction process, there are a number of different
analyses that could be performed on the predicted genes. For example,
functional annotation of the predicted genes could be performed using
tools such as BLAST2GO to assign Gene Ontology (GO) terms and KEGG
pathways to the predicted genes. This could provide further insight
into the functional roles of the predicted genes and their potential
applications. More importantly in the context of this work, analysis
of predicted genes with a tool like antiSMASH could be used to
identify biosynthetic gene clusters (BGCs) in the predicted genes.

Expanding the number of genomes used in this work would also be
beneficial, as the results presented here are based on a limited
number of genomes. Including more genomes from different species and
strains of \textit{Trichoderma} could provide a more comprehensive
understanding of gene finding performance across the
genus. Additionally, comparing the performance of gene finders on
other fungal genomes could provide insights into whether the findings
are generalizable. Examining the performance of gene finders on
different assemblies of the same genome could also be useful, as
different assemblies may have different levels of completeness and
accuracy. This could help to identify tangible benefits of using one
assembly over another, and could also provide insights into the
limitations of gene finders when applied to different assemblies.

Finally, the results of this work could be used to inform future
research in the field of fungal genomics, particularly in the case of
the DC1 and Tsth20 genome assemblies. The results of these genome
assemblies are high-quality, and contain near-chromosomal scale
sequences. The predicted genes from these genomes are a rich resource
for future research, and could be used to identify novel genes and
gene families in DC1 and Tsth20. Further understanding of the
mechanisms behind salt and drought tolerance as well as degradation of
hydrocarbon in suboptimal environments could be achieved by studying
the predicted genes in these genomes.
