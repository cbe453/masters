\section{Selection of Gene Finding Tool}
\label{chapter:conclusion}

With all of these results, it makes sense to explore the question of
which gene finding tool one should choose for optimal gene prediction
performance. Comparisons drawn in this section are made in the context
of \textit{Trichoderma} assemblies and may not extend to other
datasets. Observations from the results portion of this work have been
converted to scores relative to the other gene finders. Scores range
from zero to three, with values being: 0 - failing performance, 1 -
passing performance, 2 - good performance, and 3 - excellent
performance. Results from this work are summarized in Table
\ref{table:final-score}. Ignoring availability and use of the gene
finding tools, it would appear that RefSeq performs the best in the
remaining categories, earning top marks in every category except in
it's ability to predict very short genes. Braker2 earns second place;
however, this does not capture Braker2's failure in predicting
accurate numbers of genes in DC1, Tsth20, \textit{T. harzianum} and
\textit{T. virens}. GeneMark comes in last, excelling only in number
of genes predicted and Pfam support for the genes that it predicts.

Relating these observations to use-case scenarios, in the case that
your organism of interest has a RefSeq annotation associated with it,
the RefSeq gene prediction process appears to produce the best set of
predictions. If users also have experimental evidence, such as RNAseq
data under experimental conditions, it may be worthwhile training a
Braker2 model and predicting genes with Braker2 to supplement the
already well performing RefSeq gene predictions. In a similar case, if
the organism is of RefSeq status, but the assembly in question is
novel to the work being performed, training and prediction using a
Braker2 model with experimental evidence either from the novel genome
or from the RefSeq accession in addition to use of the RefSeq
annotation would likely be the best option. If the organism of
interest is not a RefSeq individual but training data is available for
that organism, Braker2 is the next best option, although it is
important to note that the application of a trained Braker2 prediction
model to an organism from which the training data did not originate is
not advised based on the results presented in this work (see
\ref{section:gene-finding}). While it is true that the
\textit{T. reesei} genome differs from other \textit{Trichoderma}
genomes, it's status as a representative RefSeq organism makes it
somewhat of a gold standard. In this case, applying a gene model
trained using evidence from the gold standard produces biased numbers
of genes predicted in other \textit{Trichoderma} genomes. While
Braker2 technically scores the second highest, users must be very
careful when selecting training data, and ensure that the training
data either comes from the organism of interest, or comes from a very
closely related organism with a highly similar genome. In the case
that no appropriate training data is available, GeneMark is still an
option, and users can be confident that the tool predicts a reasonably
accurate number of genes with supporting Pfam matches. It is also
important to note that GeneMark does not perform as well in AT-rich
regions as Braker2 and RefSeq, does not predict isoforms, and
systematically fails to predict some BUSCO orthologs.

When availability of a gene finding tool becomes a concern and RefSeq
is not considered, the scores drop significantly for Braker2 and
GeneMark as seen in the final row of Table \ref{table:final-score}. In
this situation, Braker2 still outperforms GeneMark. If the organism of
interest is not considered a representative RefSeq individual, but
supporting evidence specific to that organism or a very closely
related organism is available, a trained Braker2 prediction model will
perform well. Again, in the case that the organism is not a RefSeq
individual and no appropriate training data is available, GeneMark is
still a reasonable option even with the previously identified caveats.

\begin{table}
  \centering
  \begin{tabular}{|c|c|c|c|}
    \hline
    Category & Braker2 & GeneMark & RefSeq \\ \hline
    Availability & 3 & 3 & 0 \\ \hline
    Ease of install & 1 & 2 & 0 \\ \hline
    Ease of use & 3 & 3 & 0 \\ \hdashline
    \makecell{\# of genes\\predicted} & 0 & 3 & 3 \\ \hline
    \makecell{\# of transcripts\\predicted} & 3 & 0 & 2 \\ \hline
    \makecell{Predicts shortest\\genes} & 2 & 1 & 0 \\ \hline
    \makecell{Predicts more\\shorter genes} & 1 & 0 & 3 \\ \hline
    BUSCO Performance & 2 & 1 & 3 \\ \hline
    \makecell{Performance in\\AT-rich sequence} & 2 & 1 & 3 \\ \hline
    \makecell{Predictions with \\InterProScan support} & 3 & 3 & 3 \\ \hline
    \makecell{Final Score\\(Publicly Available)} & 20 & 17 & N/A \\ \hline
    \makecell{Final Score\\(Ignoring Availability)} & 13 & 9 & 17 \\ \hline
  \end{tabular}
  \caption[Final scoring table]{Table with scores attributed to
    performance of each gene finder in several categories. The score
    definitions for performance are as follows: 0 - fail, 1 - pass, 2
    - good, 3 - excellent. Since RefSeq is not publicly available, it is
    marked as N/A in the publicly available final scores. The dashed
    line separates categories associated with availability and use
    from categories describing gene prediction performance.}
  \label{table:final-score}
\end{table}

In summary, these results indicate that if your organism is a RefSeq
organism, use the RefSeq annotation. If no RefSeq predictions are
available but appropriate training data is, one should use Braker2. If
the training data is of questionable similarity or not available at
all, users can fall back on \textit{ab initio} gene finders such as
GeneMark, which while not ideal, still predict genes with supporting
evidence.

\section{Future Work}
