\textit{Trichoderma} species are fungi from the Hypocreaceae family
that are commonly found in soil and aid in the decomposition of plant
matter. They are also known for their ability to produce a variety of
enzymes involved in a number of different roles, including cellulose
degradation, nitrogen fixation, antibody production, and cellular
signalling. Some of these species are also used in agriculture as
biocontrol agents against plant pathogens. Some of the enzymes
responsible for these roles are produced in large quantities, making
them of interest to researchers in both industry and academia. In
particular, the elevated production of secondary metabolites
in \textit{Trichoderma} species has been shown to have a positive
effect on plant growth and health, making them an attractive option
for inoculation of agricultural crops. The genomes
of \textit{Trichoderma} species contain a wealth of information about
the genes and proteins involved in these processes, but they must
first be annotated to identify the genes and their functions, which is
a non-trivial task. In addition, there are a number of different tools
available for genome annotation, each with its own strengths and
weaknesses, making it difficult for a researcher to choose the right
tool for their needs. Gene finding tools are commonly compared to one
another in common model organisms, but there is little work done
comparing these tools in fungi, and even fewer in \textit{Trichoderma}
species. A focus on \textit{Trichoderma} is important in this case, as
the gene finders rely on gene models that may not be representative of
the genes found in these species.  The sequencing of two
novel \textit{Trichoderma} genomes, DC1 and Tsth20, provides an
opportunity to compare the performance of several gene finding tools
in these species. Results from this work may be of interest to
researchers studying crop resistance and plant growth promotion, as
well as those interested in the mechanisms at play behind other
benefits of \textit{Trichoderma} species. In addition, this work will
provide insight into the performance of gene finding tools in fungi
and their applicability to non-model organisms.

The main goal of this work is to compare the performance of several
gene finding tools in the genomes of two novel \textit{Trichoderma}
species, DC1 and Tsth20. In addition to these two genomes, we also use
the RefSeq assemblies and annotations of three
other \textit{Trichoderma} species, \textit{T. reesei, T. harzianum}
and \textit{T. virens}, as an additional basis for comparison, with
the goal of comparing these tools in a generic context, rather than in
the context of a gold standard, as a gold standard is not available
for these species.The exact research questions pursued in this work
are given briefly in Section~\ref{intro-objectives} explained in detail in Chapter~\ref{chap:research-questions}.

The general structure of this thesis is as follows. In
Chapter~\ref{chap:background}, we provide background information
on \textit{Trichoderma} species, genome assembly and annotation, and
gene finding tools. As mentioned before, the research questions are
explored in Chapter~\ref{chap:research-questions}, while in
Chapter~\ref{chap:methods}, we describe the methods used to answer
those questions. In Chapter~\ref{chap:results}, we present the results
of our analyses, including comparisons of the characteristics of the assemblies and
annotations of the DC1 and Tsth20 genomes to those
of \textit{T. reesei, T. harzianum} and \textit{T. virens}, as well as
comparisons of the performance of the gene finding tools. Finally, in
Chapter~\ref{chap:conclusions}, we discuss the implications of our
findings, as well as potential future work in this area. We make note
that this thesis refers to annotations from NCBI's RefSeq
database as `RefSeq', although the annotations may have been generated
using a variety of tools and methods.

\section{Objectives}\label{intro-objectives}

\noindent Objectives of this work include the following:
\begin{itemize}
    \item Assemble and evaluate the genomes of \textit{Trichoderma} species DC1
        and Tsth20.
    \item Profile and compare qualitative features of GeneMark and Braker2 gene
        finding tools.
    \item Rank the performance of GeneMark and Braker2 gene finding tools using several metrics.
    \begin{itemize}
        \item Number of genes and coding sequences predicted.
        \item Distributions of gene lengths.
        \item BUSCO completeness.
        \item InterProScan functional annotations.
        \item Similarity searches with tblastn against related RefSeq
            annotations.
        \item Secondary metabolite cluster predictions with antiSMASH.
        \item Overall agreement between gene predictions.
        \item Performance in AT-rich regions of genomes.
    \end{itemize}
\end{itemize}

