\textit{Trichoderma} species are fungi from the Hypocreaceae family
 that are commonly found in soil and aid in the decomposition of plant matter.
 They are also known for their ability to produce a variety of enzymes invovled in a number of different roles, including cellulose degradation, nitrogen fixation, antibody production, and cellular signalling. Some of these species are also used in agriculture as biocontrol agents against plant pathogens. Some the the enzymes responsible for these roles are produced in large quantities, making them of interest to both industry and academia. In particular, the elevated production of secondary metabolites in \textit{Trichoderma} species has been shown to have a positive effect on plant growth and health, making them an attractive option for inoculation of agricultural crops. The genomes of these species are of interest to researchers in the field of bioinformatics, as they contain a wealth of information about the genes and proteins involved in these processes. These genomes must first be annotated to identify the genes and their functions, which is a non-trivial task. In addition, there are a number of different tools available for genome annotation, each with its own strengths and weaknesses. Gene finding tools are commonly compared to one another in common model organisms, but there is little work done comparing these tools in fungi, and even fewer in \textit{Trichoderma} species. The sequencing of two novel \textit{Trichoderma} genomes, DC1 and Tsth20, provides an opportunity to compare the performance of several gene finding tools in these species. 

 In this work, we explore the following research areas:
 \begin{itemize}
    \item How do the assemblies of the DC1 and Tsth20 genomes compare to other \textit{Trichoderma} species?
    \item Profiling of gene finding tools: how are they implemented, and what features do they predict?
    \item How many genes and coding sequences do gene finders identify in the selected genomes?
    \item Do gene finders agree on the lengths of predicted genes?
    \item Do gene finders agree on their predictions, and if not, to what extent do they disagree?
    \item Do the predicted genes contain functional signatures, as identified by InterProScan?
    \item Do predicted genes show similarity to known proteins in \textit{Trichoderma}?
    \item How do gene finders perform in benchmarking tests, such as BUSCO?
    \item How do gene finders perform in AT-rich genomic sequence?
 \end{itemize}

 We first assembled the DC1 and Tsth20 genomes, and applied two gene finding tools, GeneMark and Braker2, to these genomes. The assemblies of DC1 and Tsth20 appear to be of high quality, with near-chromosomal scale contigs as well as N50 and L50 values greater than the RefSeq assemblies of \textit{T. reesei, T. harzianum} and \textit{T. virens}. We also applied the gene finding tools Braker2 and GeneMark to the DC1 and Tsth20 genomes, and compared the results to the RefSeq annotations of \textit{T. reesei, T. harzianum} and \textit{T. virens}. We found that gene finding tools are rarely in complete agreement with one another, and that the gene models produced by these tools can differ significantly in terms of start and stop positions, number of exons and introns, and presence of alternative splicing. All three sets of gene predictions performed well in BUSCO tests, and the gene finding tools also performed well in InterProScan tests, with all gene finders identifying proteins with Pfam matches. We also compared gene finding performance in AT-rich genomic sequence, and found that gene finding tools did not perform well in these regions, which may provide insight into evolutionary processes in \textit{Trichoderma} species. Finally, we present recommendations for selection of a gene finding tool based on the results of this work. This work has produced a rich dataset of gene predictions and annotations for the DC1 and Tsth20 genomes, which can be used to further investigate the biology of these species and the performance of gene finding tools in fungi.