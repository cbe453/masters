\section{Initial Gene Counts}
Initial gene counts from Braker2 and Genemark show a similar trend in
all assemblies included in this analysis. Braker tends to predict
fewer genes in comparison to GeneMark and RefSeq results except in the
case of \textit{T. reesei}. The cause of this difference is likely
two-fold in nature. Most importantly, Braker2 was trained on an RNAseq
dataset derived from \textit{T. reesei}. This additional information
in the gene finding model is likely why Braker2 finds more genes and
transcripts than GeneMark. The effects of this training set may also
be why Braker2 tends to predict fewer genes and transcripts in other
assemblies when compared to results from GeneMark and RefSeq. The
nature of gene models in \textit{T. reesei} likely differs than those
of other genomes, resulting in 'poorer' relative performance than in
\textit{T. reesei}. This highlights the fact that users should choose
training sets carefully when planning to use a hybrid gene finding
approach. Choosing a dataset that is inappropriate will end with
results that are skewed or biased towards the training set of
interest, although they will likely be of higher confidence. This is
an important trade-off to consider when preparing for a hybrid gene
finding approach.

Another potential explanation of Braker2's low gene count trend could
again be due to the reference training set used in training. However
in this case, not necessarily due to the RNAseq data itself, but the
nature and size of the genome from whih the RNAseq data was gathered
from. The \textit{T. reesei} genome is significantly smaller than
other assemblies considered as shown in figure (blah), resulting in
less physical space for coding sequences to be found. This would
explain why GeneMark predicts a similar number of genes as Braker2 in
\textit{T. reesei} but not in any of the other assemblies. In theory,
it would make sense that after training, Braker2 would predict a
fraction of all possible genes in a larger assembly, since it was
trained with RNAseq data from a genome that is a fraction of the
size. Again, this shows that the choice of training data is paramount
when planning a hybrid assembly approach to gene finding.


