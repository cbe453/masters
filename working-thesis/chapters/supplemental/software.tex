\section{Platform and Software Installation}
(Possibly supporting materials or discussion)

\subsection{Platform}
All analysis was performed on the RSMI server hosted on Copercius at
the University of Saskatchewan. This server is equipped with 64 cores
in addition to 1.5 TB of memory. The server is running RedHat
Enterprise Linux 7 as of writing this thesis. All data is stored
either on datastore, or in the RSMI scratch space.

\subsection{NextDenovo and NextPolish Installation}
Installation of nextDenovo was straightforward. Simply download the
compressed tar file from their website and unpack it. NextDenovo
requires Python versions 2 and 3 along with a package called parallel
to aid in parallel processing of datasets. The parallel package was
installed using pip in the bioinformatics conda environment in the
scratch space of Copernicus. NextPolish was installed in a Python
environment by a member of the research computing team that manages of
our system. Assistance was required for this as the version of RHEL
used by the server introduces glibc version conflicts with Anaconda
when trying to install nextPolish. 

\subsection{RepeatMasker Installation}

The installation procedure was somewhat indepth, requiring
RepeatMasker configuration, which itself requires downloading an
appropriate repeat database (Dfam in this case, included with
RepeatMasker), installation of Tandem Repeat Finder (TRFM) and
installation of a sequence search tool, for which I chose HMMER from
the list of potential tools as we were generally familiar with its
use. The path to the installation of TRFM is required during
configuration along with the search tool of choice, a simple selection
of 4 tools that will have an autocompleted path in this case, since
HMMER is installed via anaconda.

\subsection{GeneMark-ES Installation}
GeneMark-ES was successfully installed by downloading and unpacking
the package from their website along with a key required for use.

\subsection{Braker2 Installation}
Braker2 was also successfully installed by a member of the research
computing team who has set up several modules including an
initialization script to get things up and running as well as create a
reloadable environment for use again in the future. Once the
environment has been loaded, one must load the Hisat2 module from
Compute Canada as well as an htslib module (more detail to come). Once
all modules are loaded, there are a few environemnt variables that
need to be set, those being AUGUSTUS\_CONFIG\_PATH and
TSERBA\_CONFIG(?figure this out). In addition, a software package named
TSEBRA from the same developers as Braker2 must be installed for
consolidating gene calls. The variables can be set within the
braker2.pl command, which have higher priority over environment
variables and probably makes things easier to track.
