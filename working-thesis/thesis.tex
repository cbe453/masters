\documentclass[12pt]{article}
\usepackage[noblocks]{authblk}

\title{Project Proposal: Comparative analysis of Gene Finding tools                                                                         
  when applied to \textit{Trichoderma} genomes} \author{Connor
  Burbridge} \affil{USask NSID: cbe453 \\ USask ID no.\ 11162928
  \\ Supervisors: Dave Schneider \& Tony Kusalik\\}

\begin{document}
\parindent=14pt
\maketitle

\clearpage
\tableofcontents
\clearpage

\section{Pre-work}

\subsection{Existing selection of Genome Assemblies}

Comparison of several different \textit{Trichoderma} genomes is
important in the context of gene finding tools as different gene
finding tools will find different genes and understanding how these
tools behave in the context of different \textit{Trichoderma} genomes
could prove useful for those looking to find genes in similar fungal
genomes in the future. To accent the processing for genomes of
interest, those being DC1 and Tsht20, we should include other
previously assembled \textit{Trichoderma} assemblies. Currently
selected genomes include \textit{Trichoderma reesei},
\textit{Trichoderma harzianum}, and \textit{Trichoderma virens?}, with
\textit{Trichoderma reesei} being the 'reference' in this case, as it
is well studied and there are several patents involving it's use a
organsim for production of compounds such as antibiotics in industrial
applications.

\subsection{Assembly}

The foundation of this project is base ond the sequencing of two novel
\textit{Trichoderma} strains identified in prairie regions of Canada
(Alberta and Saskatchewan). To assemble these genomes, a hybrid
assembly process was used, following default assembly parameters with
MASuRcA, which utilizes the Flye assembler if both Nanopore and
Illumina data are used as inputs, which are the inputs in this
case. The next paragraph describes the process of working with MASuRcA.

MASuRcA 4.0.3 was run using the Compute Canada software stack
available on Copernicus.  Prior to loading the MASuRcA environment,
the GCC/9.0 and StdEnv/2020 modules must be loaded first. This version
of the software is not ideal, but the Anaconda installation of versino
4.0.9 consistently failed, even in a fresh environment.  Building the
software from scratch is a potential option. In addition to this
difficulty, the assemblies were performed in the p2irc\_rsmi scratch
space on Copernicus as I was encountering permissions issues when
trying to run the assembly in the Roots datastore.  I don't know
exactly why (microsoft permissions problems from datastore?, but there
were permission issues associated with scripts being copied to
datastore as part of the assembly process. All assembly materials were
copied back to datastore after assembly.

Initially, a configuration file must be generated to run MASuRcA with
the optimal combination of assembly tools for the data supplied (Flye
+ polishing). Running MASuRcA from the commandline tool utilizes the
CABOG celera-based assembler, which is noted as being slower and
results in an assembly with similar or worse quality than one using
Flye.

To generate the config file, run the following:
masurca -g config.txt

The config file was then altered to provide input file, options and
allowable number of threads for assembly. All other assembly
parameters were left untouched. Insert lengths for the Illumina read
data used the recommended values (stated to work for most Illumina
reads), although these could be modified with input from Brendan. The
config files for both assemblies are available in the assebly
directories.

To generate the assembly.sh script, run:
masurca config.txt

Once the assembly.sh script is generated, execute the assembly using:
./assemble.sh

Final assemblies are placed in directories with the prefix flye.mr.*
Quast analysis of the genomes was also performed, with the output
being placed in directories named quast within the assembly directory.

To run Quast:
quast -o ./ -t 16 assembly-file.fasta

In an attempt to produce higher quality assemblies of DC1 and Tsth20,
It has been suggested that I try a set of tools call NextDenovo and
NextPolish as they have produced excellent assemblies based on
previous experience from supervisors.

Installation of NextDenovo was straightforward. Simply download the
compressed tar file from their website and unpack it. NextDenovo
requires Python versions 2 and 3 along with a package called parallel
to aid in parallel processing of datasets. I installed parallel using
pip in the bioinformatics conda environment in the scracth space of
Copernicus.

Initial attempts to run the example dataset resulted in some
permissions errors, which I have encountered with other tools in the
past. Thank you datastore. To remedy this, I copied the installation
to RSMI's scratch space on Copernicus. Once the approriate permissions
were given to run nextDenovo, I was able to run the example dataset
assembly without issue. Future assemblies of DC1 and Tsth20 will be
performed in this scratch space to avoid permissiions issues and
results will be copied to datastore.

\subsection{Repeat Masking}

In order to evaluate the performance of gene finding tools in
repetitive or low complexity regions in the context of
\textit{Trichoderma} genomes, we must first identify said regions in
the genomes considered. To do this, RepeaatMasker has been selected as
a tool to identify repeat regions based on a fungal subset of the Dfam
database by specifying the fungo species tag to RepeatMasker when
running the program. The program was configured with options to
produce several output formats for each genome considered, which will
allow for more informative downstream analysis of results. All
commands for repeat masking are located withing the processing
directory for each strain/genome.

The Installation procedure was somewhat indepth, requiring
RepeatMasker configuration, which itself requires downloading an
appropriate repeat database (Dfam in this case, included with
RepeatMasker), installation of Tandem Repeat Finder (TRFM) and
installation of a sequence search tool, for which I chose HMMER from
the list of potential tools as I am generally familiar with its use.

General command for running RepeatMasker:
/datastore/Roots/Connor/masters/software/repeatmasker/RepeatMasker/RepeatMasker
-pa 10 -a -small -species fungi -html -gff -dir ./
path-to-genome/genome.fasta


\section{Software Installation}

Currently, GeneMark-ES and Braker2 have been difficult to install and
have not been successfully installed yet.

GenomeThreader installed successfully via Anaconda in the
bioinformatics environments on cnic-gifs-aio-18001 (rsmi01).

\section{Gene Finding}

Now that we have covered information about assembling and installation
of these tools, we can cover the gene finding portion of this work.

To begin, I ran GeneMark-ES as it requires no prior information or
alignments in order to run. In this case GeneMark-ES has an option
specifically for fungal genomes, which I chose to use in this case.


GenomeThreader is currently undergoing a test run with only two cDNA
files for the SRA accession SRR5229930. The command itself was
straightforward to run, although I am waiting for a successful run to
finish to confirm that. The only other thing to mention was that
GenomeThreader seems to only accept FASTA files as input. FastQ files
were not accepted and failed with an illegal character error for the @
headers on the FastQ files. To rectify this, I used seqret to convert
the FastQ files to FastA files.

GenomeThreader update: GenomeTHreader was running for over 1000 hours,
so I eventually ended the task. Unsure on whether ot not this is my
fault or an issue with the program or installation. The program was
using 250+ GB of memory while running so I assume something was
happening, but it could be a memory leak or just a really slow
program. Either way, I don't think it is an appropriate option for
this project.

Seqret basic syntax:
\textbf{seqret -fastq seqfile.fastq -fasta seqfile.fasta}


Braker2 has been successfully run on the RSMI box with the help of
Brook from Research Computing. Issues with Anaconda and glibc
incompatibilities have been frustrating and difficult to deal
with. Brook has set up several modules including an initialization
script to get things up and running AND create a reloadable
environment for reuse. Once the environment has been loaded, one must
load the Hisat2 module from compute canada as well as an htslib module
(more detail to come). Once all modules are loaded, there are a few
environemnt variables that need to be set. Alternatively, these
variables can be set within the braker2.pl command, which have higher
priority over environment variables and probably makes things easier
to track.

The variables that need to be set are AUGUSTUS\_CONFIG\_PATH and
TSEBRA\_PATH. Augustus, by defuault, tries to write species
information to the location where the software is installed. In this
case, we don'thave write permissions to the compute canada software
stack hosted byt Research Computing, so the AUGUSTUS\_CONFIG\_PATH
variable must be set in order to create a writeable directory. As long
as that path has a directory within it called braker, and a species
directory within the braker directory, things should go
smoothly. TSEBRA is a set of scripts also made by the creators of
Braker and is required to merge results from the various gene
prediction tools involved in the Braker2 pipeline. The TSEBRA\_PATH
simply points to the directory where TSEBRA is located Both Braker2
and TSEBRA can be cloned directly from GitHub (links to come)

\section{Identification of Overlapping Features and Regions}

Feature Identification: To first undertand how gene prediction tools
perform in comparison to other gene prediction tools, we must identify
features. This identification of features will help us descirbe the
similarities, and differences between gene finding tools. A feature,
in this context, is any feature stated within a Genomic Feature Format
file (GFF) provided to the program, in which mutliple GFF files can be
provided. The definition of a feature, for this application, is an
object that contains a contig ID, a start position, an end position
and a strand property. In the context of features on different
strands, start and stop positions of features are sorted based on left
and right positions of the feature in respect to the reference
sequence.

Region Identification: In addition to feature creation, we will also
identify regions of overlapping features based on the precitions from
each gene finding tool. These regions will help identify the
agreements, or disagreements, between different gene-finding tools. A
region, in this context, is a set of overlapping features, all of
which overlap at least one other feature in the region. With each
overlap, there will be an overlap type. These types can be defined
based on Allen's Interval Calculus (reference), with the exception of
features that start beyond the end point of the current region.



\end{document}
