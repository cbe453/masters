\documentclass{uofsthesis-cs}
\usepackage[noblocks]{authblk}
\usepackage{graphicx}
\usepackage{caption}
\usepackage{subcaption}
\usepackage{titlesec}
\usepackage{indentfirst}
\usepackage{hyperref}
\usepackage{booktabs}
\usepackage{makecell}
\usepackage{arydshln}
\usepackage{tikz}
\usepackage{algpseudocode}
\usepackage{algorithm}
\usepackage{textcomp}

\def\checkmark{\tikz\fill[scale=0.4](0,.35) -- (.25,0) -- (1,.7) -- (.25,.15) -- cycle;}

\title{Comparative analysis of Gene Finding tools when applied to
  \textit{Trichoderma} genomes} 
  \author{Connor Burbridge}

\parindent=14pt

\abstract{
This thesis presents a comparative analysis of gene prediction tools applied to genomes of the genus \textit{Trichoderma}. The study evaluates the performance of several widely used gene finding algorithms, assessing their accuracy, sensitivity, and specificity using both computational benchmarks and biological validation. Genomic datasets from multiple \textit{Trichoderma} species were analyzed, and the predicted gene models were compared against reference annotations and experimental data. The results highlight the strengths and limitations of each tool, providing recommendations for optimal gene annotation strategies in fungal genomics. Briefly summarizing the results, we found that gene finders generally showed high sensitivity in detecting conserved genes, and the predicted gene models contained functional domains as validated by InterProScan. However, discrepancies were observed among the tools in start and stop positions, exon-intron structures, and alternative splicing events. Notably, gene prediction accuracy declined in AT-rich genomic regions, where few genes were identified. 

When selecting a gene finding tool, users should consider available annotations, genomic context, and training data. Braker2 performed best overall when relevant high-quality training data was available. RefSeq ranked second but performed poorly in BUSCO and InterProScan categories. GeneMark performed worst overall but still provides useful gene predictions when other options are not available. From these results, we make the following recommendations when selecting a gene finding tool for \textit{Trichoderma} genomes: use Braker2 when high-quality training data exists; use RefSeq when training data is unavailable but RefSeq annotations exist; otherwise use GeneMark. Users should critically evaluate RefSeq annotations, as underlying assembly quality may impact predictions.

In addition to the main findings, the genomes of two novel \textit{Trichoderma} species, DC1 and Tsth20, were successfully assembled and annotated, resulting in high-quality genomic resources. The assemblies exhibited near-chromosomal scale contigs and performed well in standard assembly metrics. The gene predictions generated by the evaluated tools produced a rich dataset of annotations for these genomes, which can be leveraged for further biological investigations.

}

\acknowledgements{I would like to acknowledge both
  Dr. Kusalik and Dave Schneider for their excellent support and
  mentorship during both COVID and my MSc. project. I would also like
  to thank Brendan Ashby and Dr. Leon Kochian for providing both data
  and additional financial support. I also thank my family and girlfriend for their unwavering support and encouragement throughout my studies.}

\loa{
\abbrev{A}{Adenine}
\abbrev{BGC}{Biosynthetic gene cluster}
\abbrev{BLAST}{Basic local alignment search tool}
\abbrev{BUSCO}{Benchmarking universal single-copy orthologs}
\abbrev{C}{Cytosine}
\abbrev{CDF}{Cumulative distribution function}
\abbrev{CDS}{Coding sequence}
\abbrev{CPU}{Central processing unit}
\abbrev{DNA}{Deoxyribonucleic acid}
\abbrev{G}{Guanine}
\abbrev{GFF}{General feature format}
\abbrev{GC}{Guanine cytosine}
\abbrev{GMO}{Genetically modified organism}
\abbrev{HGT}{Horizontal gene transfer}
\abbrev{HMM}{Hidden Markov model}
\abbrev{HPC}{High performance computing}
\abbrev{Kb}{Kilobase\(s\)}
\abbrev{Mb}{Megabase\(s\)}
\abbrev{Mb}{Mega-base}
\abbrev{MITE}{Minature inverted-repeat transposable element}
\abbrev{NCBI}{National center for biotechnology information}
\abbrev{NGS}{Next generation sequencing}
\abbrev{PCR}{Polymerase chain reaction}
\abbrev{RIP}{Repeat-induced point (mutation)}
\abbrev{RNA}{Ribonucleic acid}
\abbrev{RSMI}{Root, soil and microbial interactions}
\abbrev{T}{Thymine}
\abbrev{TE}{Transposable element}
\abbrev{TDR}{Terminal direct repeat}
\abbrev{TIR}{Terminal inverted repeat}
\abbrev{UTR}{Untranslated region}
\abbrev{WGS}{Whole genome shotgun (sequencing)}
}


% From Yujie: Assume I am only working on chapter 2 for now, so only include that
% chapter in the main document. This is useful for working on a single
% chapter at a time, and speeds up compilation.
%\includeonly{chapter-2}


\begin{document}
\maketitle
\frontmatter


\include{intro}
\chapter{Background}
\section{Genomics}
Genomics is a wide area of study focusing on the genomes of organisms
from all varieties of life. A genome is a sequence of characters that
contains the fundamental set of 'rules' used to create what we know as
life. One can think of a genome as a set of instructions that our
cells use in order to complete the tasks that make us function. A
genome is comprised of tightly bundled sequences of DNA, which are
stored in the nucleus of cells. These bundles of DNA contain sections
known as genes, which can be thought of as the tools described by the
set of instructions. These tools carry out a vast number of processes
ranging from no known function at all to genes that are key in
protecting against diseases cancer. Genomes can vary widely in size,
ranging from small bacterial genomes of roughly 4 Mb up to
approximately 149000 Mb. Piecing together genomes provides numerous
opportunities to understand other 'omics' within cells, such as
proteomics, metabolomics, transcriptomics and epigenomics.

\subsection{Sequencing}
Sequencing data is a pivotal form of data used in nearly all
applications of Bioinformatics. To understand the processes used by
organisms for day to day survival or in unique circumstances, we must
have an initial set of data points to work with. These sequences,
referred to as reads after sequencing, are the foundation for solving
problems ranging from taxonomical classification to the understanding
or complex biological functions like signaling pathways. Reads may
come in a variety of forms and formats depending on the desired
application.

\section{Whole Genome Shotgun Sequencing}
Whole Genome Shotgun sequencing (WGS), is a method to produce a large
number of genomic sequences from a sample of interest. This is form of
sequencing is quite common as it is has a wide variety of applications
in research~\cite{Adams2008}. WGS involves slicing up genomic DNA into
smaller segments. These small segments are then processed further
resulting in a set of physical molecules that can be supplied to a
compatible sequencing platform of which there are a variety. Modern
sequencing platforms are comprised of next generation sequencing (NGS)
and 3rd generation sequencing approaches.

\section{Next Generation Sequencing - Illumina}
Illumina sequencing is one of the most popular NGS platforms currently
available. Illumina sequencing produces a very large number of high
quality short reads, typically between 75 and 250 base pairs in
length. Sequencing libraries can be prepared to produce reads solely
from one end of a sequence fragment (single-end) or both ends
(paired-end). Advantages of paired end sequences are the additional
context provided by the paired sequence on the opposite end of the
fragment. This context is leveraged by read processing tools to
identify features such as repetitive regions and genomic
rearrangments, which can be significant in downstream
analyses. Illumina sequence librairies are generated by first
fragmenting the DNA samples, amplifying them via PCR, ligating
adapters that allow the sequence to bind to the sequencing plate, and
finally identifying each fragment's sequence of nucleotides using
fluorescently-labeled nucleotides that bind to the
fragments~\cite{Goodwin2016}.

\section{3rd Generation Sequencing - Nanopore}

Nanopore sequencing data is relatively recent approach to sequencing
projects. While Illumina reads are considered to be short, Nanopore
reads are much larger, ranging from 10Kb to 300Kb depending on the
approach used. Long reads are beneficial due to their ability to
bridge the gaps between difficult to assemble regions when performing
sequence assembly. An example of a difficult to assemble region would
be a region with a high repeat content, where a large number of small
repeats may be collapsed during the assembly process, resulting in an
assembly that does not represent the true nature of the sequence being
studied~\cite{Marx2023}. Whlie Nanopore was previously known for
having lower quality base calls when compared to Illumina, that is no
longer the case at this time. Nanopore sequencing works by passing
long seqments of genetic sequence through a membrane bound protein and
measuring changes in electrical current, which is characteristic of
the nucleotide at a given position.

\section{Trichoderma}

Crop resistance to environmental stressors is a necessity for crop
health and overall crop yields. Current popular methods for crop
protection involve the use of pesticides and genetically modified
organisms, which can be expensive and potentially politically dividing
in the case of GMOs\cite{doi:10.1080/10408390600762696}. In addition,
crops suffer when soils are not sufficient for crop growth and
health. Soil insufficiencies can result in drought stress as well as
nutrient stress, leading to poor overall yields.

\textit{Trichoderma} is a fungi that can both communicate with and
colonize the roots of plants in a non-toxic, non-lethal, opportunistic
symbiotic relationship\cite{Woo2023}. Many strains of
\textit{Trichoderma} have been shown to provide resistance to
pathogenic bacteria and other fungi in soils through the use of
polyketides, non-ribosomal peptide synthetases and other antibiotic
products\cite{Woo2023}. Recently, two strains of \textit{Trichoderma}
have been identified in the prairie regions of Alberta and
Saskatchewan. These two strains, named Tsth20 and DC1, have been found
to have beneficial properties when used as an inoculant for plants in
the soils mentioned before. In addition to these beneficial
properties, the two strains mentioned previously provide even further
protection for plants in dry, salty soils and one strain also has
potential for use as a bioremediation tool in soils contaminated with
hydrocarbon content. Bioremediation and resistance to drought
tolerance has also been investigated in other strains of
\textit{Trichoderma} as well\cite{10.3389/fpls.2023.1190304}. However,
little is known about the mechanisms at work in these strains, so DC1
and Tsth20 were sequenced by the Global Institute for Food Security
(no publication yet) in an initial attempt to better understand the
details of these genomes. While this research does not directly
identify genomic elements related to the secretome of these genomes,
it may serve as a foundation for future research of
\textit{Trichoderma}.

\section{Genome Assembly}

Sequence assembly has been a long-standing problem in the field of
bioinformatics\cite{Nagarajan2013}. Determining the correct order and
combination of smaller subsequences into an accurate complete sequence
assembly is computationally difficult in terms of compute resources
such as memory, CPU cycles and storage required for input
sequences\cite{Nagarajan2013}. In addition to these difficulties,
there can be other issues encountered during asssembly due to the
nature of the data or genomes themselves, such as low quality base
calls for long read data, which is not necessarily the case today, or
the inherent content of genomes themselves using repetitive regions as
an example. Insufficient data used in an assembly may result in short,
fragmented assemblies, depending on the size of the genomes, while
sequence data that is not long enough can fail to fully capture
repetitive regions in an assembly. To solve this problem, a wide range
of assembly tools have been developed with their own unique approaches
to the genome assembly problem, so it is important to use an
appropriate assembler for the task at hand, and also important to
evaluate the assembly thoroughly.

Genome assembly tools generally approach the assembly problem using a
graph-based approach. The most common graph-based approach is the de
Bruijn graph assembly~\cite{Compeau2011}. A graph in this context, is
set of nodes (\textit{k}-mers from sequences) connected by edges
(overlaps between \textit{k}-mers). Traversing through this graph
results in longer subsequences that ultimately result in a set of
consensus sequences and final assembly. In the early years of long
read sequence data, sequencing platforms encountered difficulties
producing consistently high scores for base calls when seuqncing. To
combat this, some assembly workflows may also include a polishing or
correction step once the initial assembly is completed in which high
quality short read sequences are supplied as supplemental information
to correct low quality base calls in the assembly. These low quality
base calls are typically not present in modern long read sequencing
approaches as the methodology and quality of calls have improved
drastically. While the polishing step is arguably unnecessary in
modern assemblies, the polishing programs remain available should
researchers be interested in applying additional reads for polishing.

One approach to aid in the previously mentioned issue of assembly
correctness is to use a combination of long and short reads in what is
known as a hybrid assembly. Combining both highly accurate short reads
with deep coverage along with less accurate but much longer reads can
produce high quality genome assemblies that capture long repetitive
regions. Hybrid assembly approaches have been shown to produce high
quality assemblies in a wide variety of organisms as the combine long
read data with short data to produce assemblies that properly
represent long repetetive regions with additionaly high quality
Illumina sequences for correction. Once assembled, the sequences must
also be evaluated with measures such as N50, L50, coverage, average
contig length and total assembled length to ensure that the genomes
are well assembled, at least based on these
metrics\cite{Nagarajan2013}. Following appropriate assembly protocols
is essential to the further success of a project as downstream
processing such as annotation depends on a high-quality assembly.

\section{Identification of AT-rich Genomic regions}
One important aspect of interest when assembling any form of sequence
is GC content or percent GC of the assembled sequence. Large regions
of anomolous GC content may be of interest to researchers as they may
contain repetitive regions and unique features responsible for traits
specific to the organism in question.

\section{Repeat Identification and Masking}
Repeat identification within assembled genomes is a problem that needs
to be considered during the genome annotation process. Regions with
long repeats can have a significant impact on genome assembly as well
as gene finding due to the limitation of short reads used in some
assemblies\cite{Treangen2011}. Short reads may be unable to bridge or
cover entire repeat regions within a genome, so it is important to
consider the use of long reads from technologies such as Nanopore or
PacBio to provide a complete picture of these regions when pursuing a
new genome assembly project. It is also possible for repetitive
regions to contain genes as well, making for an interesting
investigation in regards to \textit{Trichoderma}, as fungal genomes
have been shown to contain many repeat regions with a high
concentration of A and T
nucleotides\cite{10.1371/journal.pgen.1007467}. Once these repetitive
regions have been identified, the genome could be masked to exlude
these regions in downstream processing if desired, as these regions
may be poorly assembled and may result in found genes that do not
truly exist in those regions. However, this may not be as common
today, as repetetive regions have been shown to contain genes as
well\cite{Slotkin2018}. This may affect the gene finding process
described later and may be an interesting topic to look into
considering the large number of available gene finding programs.

\section{Centromere Identification}
A centromere is a region of a chromosome that is crucial for the
proper cell division. These regions are the main anchor for
microtubules, which are a cellular structures used that attach to
centromeres to separate chromosomes during both mitosis and
meiosis. Centromeres are critical to the survival of an organism, with
malfunctions in the process of cell division usually resulting in
potential disease and fatal outcomes\cite{Plohl2014}. Centromeric
sequences can be comprised of several different genetic components,
with repetititve regions being the most prevalent in the forms of
satellite DNA and transposable elements. In addition to centromeric
regions, there are flanking pericentric regions with their own
properties, including potential candidates for small-interfering
RNAs\cite{Plohl2014}. Identification and consideration of centromeric
regions may prove useful when comparing the outputs of gene finding
tools, as the underlying properties and structure of the genetic
sequence differ in comparison to typical coding regions of DNA.

\section{Gene Finding Methods}
Gene finding (or gene annotation) has been a long standing
computational problem in bioinformatics, which concerns itself with
identifying potential genes within assemblies based on patterns or
pre-existing experimental evidence evidence considered by the gene
finding program. This process is critical for unraveling and
understanding the complex processes occurring in all forms of life
with applications in medical science, agriculture, biomanufacturing,
environmental studies and many others. In a general sense, gene
finding programs operate by searching for patters or indicators
showing that a gene of feature may be present. The most basic
indicators being start and stop codons, with introns and exons in
between should the sequence match the applied model. The results
produced by gene finding tools can vary considerably for a number of
reasons, including quality of the assembly, the intrinsic model used
by the gene finder, filtering criteria, and even the nature of the
organism and assembly itself. Given the broad applications, choice of
gene finding tools, and the variability of assemblies being
considered, it is important that we gain a deeper understanding of
these tools prior to putting them to use.

There are two common methods for gene finding, those methods being
\textit{ab initio} methods, where programs search for patterns and
gene structures, and similarity or evidence-based searches, which use
prior information such as RNAseq data, expressed sequence tags and
expressd protein sequences to identify genes within a new
genome\cite{Ejigu2020}. Complicating the process more is the
introduction of introns and alternative splicing in eukaryotes, making
it possible for one gene to have several possible transcripts at the
same locus. An example of an \textit{ab initio} method would be
GeneMark-ES\cite{10.1093/nar/gki937}, while an evidence based tool
would be Braker2\cite{Bruna2021}.
\textit{Ab initio} gene finders typically predict genes using a Hidden
Markov Model (HMM)\cite{Ejigu2020}. These predictions are based on
'signals' or features associated with a gene, such as the usual start,
stop, exon and intron portions of a gene as well as upstream promoter
sequences and more. In this case, these signals would be considered
states in the terminology associated with HMMs. Gene finders wish to
predict these states based on observations, or sequences presented to
the model. HMMs in gene finding tools are trained beforehand and then
applied to a sequence. This means that a gene finding program may not
be trained in the context of any assembly provided to it, and thus may
miss genes that are unique to the assembly in question.
On the other hand, while still relying on HMMs for a 'base' set of
predictions, evidence-based gene finding tools leverage new evidence
that may be outside the scope of the pre-existing
model\cite{Keller2011}.  As an example, an evidence-based model would
be useful in a situation where you are interested in annotating a new
assembly for a non-model organism. The addition of experimental data
provides context specific to your assembly of interest while still
retaining the predictions from existing HMM models.

There are also other aspects of gene finding tools that are important
to consider. These include features such as whether or not the gene
finders find non-coding RNAs, annotation of 5' and 3' UTR regions, and
in the case of ab-initio methods, the assumptions made by the
underlying models used for gene finding. These features and others can
influence a user's decision on which gene finding tool to consider and
will complicate comparative analysis of multiple gene finding
tools. (citation needed somewhere in here)

\section{InterProScan}
The outputs from gene finding tools are a set of potential genes that
fit the model used by each tool. While they are considered genes, the
use of the word gene is used in a very loose sense, in that these
genes may or may not be functional or match any existing gene
sequences from previous research. Typically, to confirm the
'correctness' of predicted genes, the outputs from a given tool are
used in a sequence similarity search against a reference set of genes
or a large datasbase comprised of multiple organisms. This approach is
straightforward, but can introduce bias from database choice and also
allows for vague or loose matches, depending on the parameters used
and the interpretation of the results. Another approach is to use
InterProScan, which is a tool used for functional annotation of
proteins using evidence from a variety of databases
\cite{10.1093/nar/gkac993}. The presence of some form of functional
domain or annotated structure in a predicted gene sequence is
reasonable evidence for the existence of a predicted gene. In
addition. This approach also avoids the problems associated with
similarity-based approaches.


\section{File Formats}

\subsection{FASTA}
One of the most popular formats for sequences of DNA, RNA and amino
acids is the FASTA format. The FASTA format consists of one or more
entries containing two or more lines. The first line of an entry is
the ID line, which must begin with a greater-than ('>') character,
followed by an ID and any other pertinent inormation for the following
sequence. The greater-than character is the indicator that a new
sequence has begun. The following line(s) contain the actual sequenced
nucleotides or amino acids, which can be contained on one line or
split across many lines. An example of multiple FASTA entries are
shown in figure~\ref{fig:fasta-example}.

\begin{figure}
  \centering
  \includegraphics[width=0.8\textwidth]{figures/fasta-example.png}
  \caption{Example of two FASTA sequence entries. One example with
    sequence characters split across multiple lines, and one showing
    all sequence characters on the same line.}
  \label{fig:fasta-example}
\end{figure}

\subsection{FASTQ}
Another popular sequencing format is the FASTQ format. This format is
very similar to the FASTA format but with the addition of two more
lines per sequence entry and a change to the character indicating the
beginning of a new sequence entry. An example of a FASTQ entry is
shown in figure~\ref{fig:fastq-example}. In FASTQ formatted entries,
the greater-than ('>') character is swapped with the at ('@')
character. The IDs for the sequence also follow a specific format,
which provide information about the sequencing run and flowcell that
the read was sequenced on. This information can then be traced back to
the sequencing experiment in the case that there were errors or
anomalies in the output from the experiment. Following the ID is the
string of base calls. The third line in a FASTQ entry is a plus ('+')
character, which indicates that the sequences character line has
finished. Following the plus ('+') character is another sequence of
characters, this time indicatintg the quality of basecall for the
corresponding nucleotide base calls in the second line. The quality
information included in FASTQ files are used to assess the quality of
a sequencing run and extensively used in downstream processing steps,
most notably in alignments.

\begin{figure}
  \centering
  \includegraphics[width=0.8\textwidth]{figures/fastq-example.png}
  \caption{Example of the four lines in a FASTQ entry.}
  \label{fig:fastq-example}
\end{figure}

\subsection{General Feature Format - GFF}
General feature format (GFF) is a popular format for storing
information about features relative to a position on an genetic
sequence, and comprises a large portion of annotation results from
this work. These features can be whatever the user desires, as long as
the feature entry follows the required GFF guidelines. Relative to a
reference sequence, each GFF entry contains the following
tab-delimited columns: sequence ID, source, feature type, start
position, end position, score, strand, phase, and a semi-colon
delimited list of attributes. GFF files are widely supported accorss
bioinformatics tools, making them highly versatile while also
remaining relatively simple in nature but also allowing for storage of
more complicated items via the attributes column. One significatn
useage of GFF files is in visualization of features against the
reference sequence from which they were derived. Most genome viewers
(or browsers) support GFF files as input, allowing intuitive
visualization of many features when overlayed on a reference
sequence. An example of a GFF entry can be seen in
figure~\ref{fig:gff-example}.

\begin{figure}
  \centering
  \includegraphics[width=0.8\textwidth]{figures/gff-example.png}
  \caption{An example of GFF entries for a sinlge gene.}
  \label{fig:gff-example}
\end{figure}

Background for BUSCO (from research questions)
As more and more genomes are assembled for new organisms, a tool was
developed to evaulate assemblies and subsequent annotations from the
pserpective of gene orthology. As genomes diverge evolutionarily, it
is expected that some genes will be conserved. The BUSCO (Benchmarking
Universal Single-Copy Orthologs) tool and datasets were developed to
assess completeness of an annotation in comparison to evolutionarily
conserved genes.


\chapter{Results}
\section{Assemblies of DC1 and Tsth20}

For general assembly metrics of DC1 and Tsth20, the QUAST tool was
used. Results from QUAST are shown in figure~\ref{table:assemblies},
from which we can make several observations. In DC1 and Tsth20, the
total contig counts are an order of magnitude smaller when compared to
the other NCBI RefSeq assemblies, inidicating highly contiguous
assemblies from nextDenovo and nextPolish. This is likely due to the
use of long-read sequencing used in the assemblies of DC1 and
Tsth20. The total assembly lengths are similar, ranging from 38Mb to
42Mb, except in the case of \textit{T. reesei}, which is known to have
a significantly smaller genome length \cite{Kubicek2019} at roughly
33Mb. The largest contig size for each assembly varies greatly. DC1
and Tsth20 have the largest contigs of all assemblies being
considered, which is again likely due to the inclusion of long-read
sequencing data in the assembly process. The N50 values for all
assemblies are above 1Mb, with DC1 and Tsth20 N50s being at minimum
three times larger than others assemblies. Results from this table
indicate that the assemblies of DC1 and Tsth20 are more contiguous
than the assemblies of \textit{Trichoderma reesei, harzianum and
  virens} also considered in this analysis. While contiguity is not
the sole indicator of genome quality, it does provide confidence in
the quality of the input data and resulting assemblies.

\begin{table}
  \begin{center}
    \begin{tabular}{|c|c|c|c|c|c|c|}
      \hline
      Strain & Total Contigs & Total Length & Largest Contig & GC\% & N50 & L50 \\ \hline
      DC1 & 8 & 38.6 Mb & 11.49 Mb & 47.97 & 5.69 Mb & 3 \\ \hline
      Tsth20 & 7 & 41.58 Mb & 8.02 Mb & 47.33 & 6.52 Mb & 3 \\ \hline
      \textit{T. harzianum} & 532 & 40.98 Mb & 4.08 Mb & 47.61 & 2.41 Mb & 7 \\ \hline
      \textit{T. virens} & 93 & 39.02 Mb & 3.45 Mb & 49.25 & 1.83 Mb & 8 \\ \hline
      \textit{T. reesei} & 77 & 33.39 Mb & 3.75 Mb & 52.82 & 1.21 Mb & 9 \\ \hline
    \end{tabular}
  \end{center}
  \caption{General assembly metrics produced by QUAST (a
    genome quality assement tool).}
  \label{table:assemblies}
\end{table}

During initial investigation of the sequences used as input to the
assembly process, we observed that the reads contained abnormal ratios
of GC content. To see if this observation extended to the assemblies
as well, sliding windows of GC content were calculated for all
assemblies included in this analysis. The results of this analysis are
shown in figure~\ref{fig:assembly-gc}. Of the included assemblies,
anomalous GC content in the form of AT-rich sequences were identified
in DC1, Tsth20, \textit{T. reesei} and \textit{T. harzianum}, with
\textit{T. virens} showing no anomalous GC content. Anomolous GC
content is visualized on the left tails of the distributions with a
local peak around sequences containing 10 percent GC content. In
addition to the confirmation of anomalous GC content, it appears that
the distribution of GC content in \textit{T. reesei} differs from the
other assemblies. The curve of GC content for \textit{T. reesei},
visualized in green in \ref{fig:assembly-gc}, lies farther to the
right, indicating higher GC content in its assembly. While the left
tail of the curve also shows an increase in AT rich sequence
composition, it is shifted farther right than other
\textit{Trichoderma} assemblies. Investigation of these anomalous
regions is continued in section ...

\begin{figure}
  \begin{center}
    \includegraphics[width=0.8\textwidth]{figures/gc-plot.pdf}
  \end{center}
  \caption{Plots showing the frequency of GC values calculated from
    sliding windows for each assembly.}
  \label{fig:assembly-gc}
\end{figure}



\section{Initial Gene Finding Results} 

Counts of genes predicted by Braker2, GeneMark and RefSeq are shown in
table \ref{table:gene-counts}. Immediately we see that Braker2
predicts far fewer genes in all assemblies, except in the case of
\textit{Trichoderma reesei.} This is possibly due to the effects of
training the Braker2 gene model using data from \textit{Trichoderma
  reesei}, which has a significantly smaller genome in comparison to
other \textit{Trichoderma} assemblies, although genome size is not
always indicative of gene content. Regardless, there is a significant
difference in the number of genes predicted by Braker2 in comparison
to GeneMark and RefSeq. The number of genes predicted by GeneMark and
RefSeq are similar, except in the case of \textit{T. harzianum}, in
which RefSeq predicts roughly 17\% more genes than GeneMark. Braker2
consistently predicts more transcripts than GeneMark and
RefSeq. RefSeq also appears to predict multiple transcripts for each
gene but in fewer numbers than Braker2. Transcript prediction counts
in \textit{T. harzianum} from RefSeq are also interesting, with RefSeq
predicting fewer transcripts than genes. Why this is occurs is unknown
but may warrant further investigation.

\begin{table}
  \centering
  \begin{tabular}{|c|c|c|c|c|c|c|}
    \hline
    Assembly & Braker2 & & GeneMark & & RefSeq & \\ \hline
     & Genes & Transcripts & Genes & Transcripts & Genes & Transcripts \\ \hline
    DC1 & 8546 & 8637 & 11353 & 11353 & N/A & N/A \\ \hline
    Tsth20 & 8784 & 8858 & 12362 & 12362 & N/A & N/A \\ \hline
    \textit{T. reesei} & 9659 & 10175 & 9196 & 9196 & 9109 & 9118 \\ \hline
    \textit{T. harzianum} & 8314 & 8385 & 12164 & 12164 & 14269 & 14090 \\ \hline
    \textit{T. virens} & 7801 & 7863 & 11866 & 11866 & 12405 & 12406 \\ \hline
  \end{tabular}
  \caption[Gene prediction counts]{Number of genes predicted by each
    gene finder for each \textit{Trichoderma} genome.}
  \label{table:gene-counts}
\end{table}



\section{BLAST Results}
Results from the T-BLAST-N runs are presented in table
\ref{table:tblastn}. Initial BLAST results appear promising for both
the \textit{T. atroviride} and \textit{Fusarium} datasets. All
assemblies considered contain at minimum 89\% of the reference protein
sequences in the case of \textit{T. atroviride} and a minimum of 75\%
in the case of \textit{Fusarium}. Following the trend of gene calls,
\textit{T. virens} returns the fewest hits of the selected assemblies
in all cases while the other assemblies report a similar number of
hits as each other. In the case of \textit{S. cerevisiae}, a minimum
of 57\% of reference proteins matched. These results provide rough
validation that the assemblies contain potential for protein coding
sequences. Successful hits from this process will be retained and used
in the region identification process as validation for gene calls from
selected tools.


\begin{table}
  \centering
  \begin{tabular}{|c|c|c|c|c|c|c|}
    \hline
    Reference & Ref. Proteins & DC1 & Tsth20 & \textit{T. reesei} & \textit{T. harzianum} & \textit{T. virens}  \\ \hline
    \textit{T. atroviride} & 11807 & 11552 & 11080 & 10601 & 11081 & 11078 \\ \hline 
    \textit{Fusarium} & 13312 & 10327 & 10429 & 10064 & 10434 & 10490 \\ \hline
    \textit{S. cerevisiae} & 6014 & 3537 & 3517 & 3445 & 3509 & 3500 \\ \hline
  \end{tabular}
  \caption{tBLASTn hits from reference protein sequences to selected
    assemblies of intereset. Hits are reported if the alignment length
    is greater than 30\% of the reference protein length and if 30\%
    of the aligned length have identical matches.}
  \label{table:tblastn}
\end{table}

\section{BUSCO Results}\label{section:busco}

Results of BUSCO analysis using the sordariomycetes\_Odb12 dataset
provided by BUSCO are presented in Figure~\ref{fig:busco-counts} and
Table~\ref{table:busco}. The results indicate that all gene sets
considered in this analysis have a BUSCO completeness of 94.1\% or
higher, with a maximum completeness of 98.4\% in the case of Braker2
and DC1. In general, Braker2 and GeneMark have the most BUSCO complete
sets of gene predictions of the three tools considered. Interestingly,
Braker2 produces far more duplicated BUSCO matches than both GeneMark
and RefSeq. Examining the BUSCO output logs, this appears to be due to
Braker2 predicting more than one coding sequence for some genes
predictions, resulting in multiple similar proteins. Interestingly,
the coding sequences in the RefSeq annotations seem to miss more genes
than the other two gene finders while also having a higher number of
fragmented BUSCO genes.\cbstart~This may be due to human curation of the
RefSeq datasets, or the proprietary Gnomon~\cite{zotero-item-392} gene prediction pipeline used by NCBI
to produce these annotations\cbend. Further investigation is required to
determine the exact cause of this discrepancy. Finally, it appears
that \textit{T. reesei} tends to have slightly lower BUSCO
completeness than the other \textit{Trichoderma} species considered in
this analysis, regardless of gene finder used. Why this is the case is
unknown, but may be due to the Gnomon annotation process used, the fragmented nature of the \textit{T. reesei} assembly used in this
analysis, or that \textit{T. reesei} may not possess some of the missed orthologs. While these results do not capture the entire set of genes
possibly present in these \textit{Trichoderma} assemblies, they do
confirm that the gene finders are at minimum predicting many
evolutionarily conserved fungal genes.

\begin{figure}
  \centering
  \begin{subfigure}{0.9\textwidth}
    \centering
    \includegraphics[width=\textwidth]{figures/busco-complete-counts.pdf}
  \end{subfigure}
  \hfill
  \begin{subfigure}{0.9\textwidth}
    \centering
    \includegraphics[width=\textwidth]{figures/busco-missing-counts.pdf}
  \end{subfigure}
  \caption[BUSCO counts]{BUSCO complete and missing gene counts for each gene finder across all \textit{Trichoderma} genome assemblies. There were a total of 4492 markers in the sordariomycetes\_Odb12 BUSCO dataset. For DC1 and Tsth20, RefSeq annotations are not available, so their values are set to 0.}\label{fig:busco-counts}
\end{figure}

\begin{table}
  \begin{center}
    \begin{subtable}{\textwidth}
      \centering
      \begin{tabular}{|c|c|c|c|c|c|c|}
        \hline
        Strain & Complete & Single & Duplicated & Fragmented & Missing \\ \hline
        DC1 & 4419 & 3547 & 872 & 40 & 33 \\ \hline
        Tsth20 & 4416 & 3585 & 831 & 42 & 34 \\ \hline
        \textit{T. reesei} & 4321 & 3587 & 734 & 80 & 91 \\ \hline
        \textit{T. harzianum} & 4408 & 3572 & 836 & 49 & 35 \\ \hline
        \textit{T. virens} & 4409 & 3530 & 879 & 53 & 30 \\ \hline
      \end{tabular}
      \caption{Braker2}
      \vspace{0.5cm}
    \end{subtable}
    \begin{subtable}{\textwidth}
      \centering
      \begin{tabular}{|c|c|c|c|c|c|c|}
        \hline
        Strain & Complete & Single & Duplicated & Fragmented & Missing \\ \hline
        DC1 & 4415 & 4401 & 14 & 43 & 34 \\ \hline
        Tsth20 & 4409 & 4392 & 17 & 47 & 36 \\ \hline
        \textit{T. reesei} & 4351 & 4345 & 6 & 69 & 72 \\ \hline
        \textit{T. harzianum} & 4399 & 4382 & 17 & 53 & 40 \\ \hline
        \textit{T. virens} & 4391 & 4369 & 22 & 61 & 40 \\ \hline
      \end{tabular}
      \caption{GeneMark}
      \vspace{0.5cm}
    \end{subtable}
    \begin{subtable}{\textwidth}
      \centering
      \begin{tabular}{|c|c|c|c|c|c|c|}
        \hline
        Strain & Complete & Single & Duplicated & Fragmented & Missing \\ \hline
        \textit{T. reesei} & 4274 & 4267 & 7 & 112 & 106 \\ \hline
        \textit{T. harzianum} & 4383 & 4366 & 17 & 60 & 49 \\ \hline
        \textit{T. virens} & 4345 & 4321 & 24 & 92 & 55 \\ \hline  
      \end{tabular}
      \caption{RefSeq}
    \end{subtable}
  \end{center}
  \caption[BUSCO results]{Results from BUSCO using the fungal analysis option
    organized by gene finding tool. The sordariomycetes\_Odb12 dataset
    contains 4492 markers. For more information on the categories
    assigned by BUSCO, please refer to the \href{https://busco.ezlab.org/busco\_userguide.html\#interpreting-the-results}{BUSCO user guide}.}
    \label{table:busco}
\end{table}

While BUSCO matches are a good metric for general performance of gene
finders, it is also important to investigate BUSCO proteins that were missing from the gene predictions. Of interest are BUSCO proteins that are systematically missed by a gene finder in multiple or all assemblies, and whether these BUSCO proteins are also missed by other gene finders. Table~\ref{table:missed-busco-all} lists 17 BUSCO proteins and their corresponding annotations that were not found in any set of predictions from Braker2, GeneMark or RefSeq. The annotations of these BUSCO proteins do not indicate any obvious reason why they would be systematically missed by all three gene finders, and further investigation is required to determine why these genes are not being predicted. It is possible that these genes are simply not present in these \textit{Trichoderma} species, but this may be unlikely given the evolutionary conservation of BUSCO proteins. Each gene finder also has one or more BUSCO proteins that are missed in all assemblies but not by one or both of the other gene finders. These genes are presented in Table~\ref{table:missed-all-gf}. Again, the annotations of these BUSCO proteins do not indicate any obvious reason why they would be systematically missed by a particular gene finder, and further investigation is required to determine why these genes are not being predicted. While not presented here, we also identified a number of other genes that were missed by the gene finders which may provide insight into their performance and the evolution of \textit{Trichoderma}. Overall, it appears that there are very few BUSCO proteins that are systematically missed by all gene finders, indicating that the gene finders are generally capable of predicting the majority of conserved fungal genes.


\begin{table}[h]
  \centering
  \begin{tabular}{|c|c|}
    \hline
    BUSCO ID & Annotation \\ \hline
    191658at147550 & Rossmann-fold NAD (+)-binding protein \\ \hline
    250641at147550 & Aspartic-type endopeptidase \\ \hline
    279527at147550 & Phosphatidic acid-preferring phospholipase A1, contains DDHD domain \\ \hline
    578862at147550 & Transcription factor \\ \hline
    579967at147550 & Alpha-L-rhamnosidase C \\ \hline
    627082at147550 & Ferric reductase \\ \hline
    628206at147550 & Zinc finger domain-containing protein \\ \hline
    636196at147550 & ATP-dependent RNA helicase \\ \hline
    646880at147550 & Lipase class 3 \\ \hline
    647592at147550 & Conserved hypothetical protein \\ \hline
    652375at147550 & Conserved hypothetical protein \\ \hline
    657983at147550 & Conserved hypothetical protein \\ \hline
    658912at147550 & Aromatic amino acid aminotransferase \\ \hline
    659938at147550 & HAUS augmin-like complex subunit 1 \\ \hline
    672116at147550 & Nitrogen regulatory protein AreA, GATA-like domain \\ \hline
    677025at147550 & Mating-type switching protein Swi10 \\ \hline
    689133at147550 & Myosin heavy chain \\ \hline
  \end{tabular}
  \caption[Missing BUSCO IDs]{BUSCO IDs and their corresponding annotations that were not found in any set of predictions from Braker2, GeneMark or RefSeq.}\label{table:missed-busco-all}
\end{table}

\begin{table}[h]
  \centering
  \begin{tabular}{|c|c|c|}
    \hline
    BUSCO ID & Tool & Annotation \\ \hline
    282444at147550 & RefSeq & Conserved hypothetical protein \\ \hline
    291262at147550 & RefSeq & HAUS augmin-like complex subunit 6, N-terminal \\ \hline
    632369at147550 & RefSeq & Peptidyl-prolyl cis-trans isomerase, FKBP-type \\ \hline
    632579at147550 & RefSeq & Cyanate hydratase \\ \hline
    672871at147550 & RefSeq & MAU2 chromatid cohesion factor \\ \hline
    677279at147550 & RefSeq & Pentatricopeptide repeat domain-containing \\ \hline
    688724at147550 & GeneMark \&Braker2 & Putative protein of unknown function \\ \hline
  \end{tabular}
  \caption[Additional missing BUSCO IDs]{Additional BUSCO IDs along with their corresponding annotations not found in any set of predictions by one or more tools but not all. The tools that missed each BUSCO ID are also listed.}\label{table:missed-all-gf}
\end{table}

%\begin{center}
% \begin{table}
% \makebox[\textwidth]{
% \begin{tabular}{|c|c|c|c|c|c|c|c|}
%   \hline
%   Tool & BUSCO ID & Annotation & DC1 & Tsth20 & \textit{T. reesei} & \textit{T. harzianum} & \textit{T. virens} \\ \hline
%   Braker2 & 195619at4751 & \makecell{Pyridoxal phosphate-dependent \\ transferase} & \  & \checkmark & \checkmark & \checkmark & \checkmark \\ \hline
%   Braker2 & 285254at4751 & Aminoacyl-tRNA synthetase & \checkmark & \checkmark & \checkmark &  & \checkmark \\ \hline
%   Braker2 & 348020at4751 & Formyl transferase &  &  & \checkmark &  & \checkmark \\ \hline
%   Braker2 & 497024at4751 & Zinc finger C2H2-type &  & \checkmark & \checkmark & \checkmark & \checkmark \\ \hline
%   GeneMark & 195619at4751 & \makecell{Pyridoxal phosphate-dependent \\ transferase} &  & \checkmark & \checkmark & \checkmark & \checkmark \\ \hline
%   GeneMark & 285254at4751 & Aminoacyl-tRNA synthetase & \checkmark & \checkmark & \checkmark &  & \checkmark \\ \hline
%   GeneMark & 348020at4751 & Formyl transferase &  &  &  &  & \checkmark \\ \hline 
%   GeneMark & 438731at4751 & LSM domain & \checkmark & \checkmark &  & \checkmark & \checkmark  \\ \hline
%   GeneMark & 470813at4751 & Ubiquitin-conjugating enzyme &  &  &  &  &  \\ \hline
%   GeneMark & 497024at4751 & Zinc finger C2H2-type &  & \checkmark & \checkmark & \checkmark & \checkmark \\ \hline
%   RefSeq & 494at4751 & Midasin & N/A & N/A &  &  & \checkmark\\ \hline
%   RefSeq & 315802at4751 & tRNA dimethylallyltransferase & N/A & N/A & \checkmark & \checkmark &  \\ \hline
%   RefSeq & 352224at4751 & YEATS & N/A & N/A &  & \checkmark &  \\ \hline
% \end{tabular}
% }
% \caption[GeneMark missed BUSCO proteins]{The presence (\checkmark)
%   or absence of all BUSCO IDs missed by Braker2, GeneMark and RefSeq
%   in each \textit{Trichoderma} assembly.}
% \label{table:genemark-busco}
%\end{table}
%\end{center}

%\begin{table}
%  \centering
%  \begin{tabular}{|c|c|c|c|c|c|c|}
%    \hline
%    BUSCO ID & Annotation & DC1 & Tsth20 & \textit{T. reesei} & \textit{T. harzianum} & \textit{T. reesei} \\ \hline
%    494at4751 & Midasin & N/A & N/A & X & X & \checkmark\\ \hline
%    315802at4751 & tRNA dimethylallyltransferase & N/A & N/A & \checkmark & \checkmark & X \\ \hline
%    352224at4751 & YEATS & N/A & N/A & X & \checkmark & X \\ \hline
%  \end{tabular}
%  \caption[RefSeq missed BUSCO proteins]{The presence (\checkmark) or
%    absence (X) of all BUSCO IDs missed by RefSeq in each
%    \textit{Trichoderma} assembly.}
%  \label{table:refseq-busco}
%\end{table}

Braker2, GeneMark and RefSeq all demonstrate excellent coverage of the
BUSCO fungal protein set, indicating that these gene finders are
capable of predicting genes that are expected to be present in these
assemblies. From this we can say that the foundations of the
underlying gene models used by each gene finder are solid. Braker2
produces more duplicate matches than GeneMark and RefSeq, but this is
likely due to multiple isoforms of possible genes being present in the
input data. Despite excellent coverage of the BUSCO fungal proteins,
all three gene finders miss some BUSCO proteins in their
predictions. 
Finally, we reiterate the possibility that human curation of RefSeq datasets is responsible for these differences, but this requires further investigation.

\section{Region Identification}

The region identification process begins with identification of
regions sharing gene calls from each tool. For this project, we have
classified regions as complete, partial or singleton. A complete
region is a set of overlaps which contains a feature from each tool
considered in the region finding process. A partial region is a set of
overlaps which includes more than one but not all tools considered. A
singleton is a region in which a feature from only one tool is
present. Table \ref{table:regioncounts} displays the results from the
region finding process when applied to only the features of type
'gene' predicted by each tool.

\begin{table}
  \begin{center}
    \begin{tabular}{|c|c|c|c|c|c|c|}
      \hline
      Assembly & Regions (total) & Complete Agreement & Partial Agreement & Singletons\\ \hline
      DC1 & 11269 & 8483 & N/A & 2786  \\ \hline
      Tsth20 & 12272 & 8737 & N/A & 3535  \\ \hline
      \textit{T. reesei} & 9823 & 8282 & 557 & 984  \\ \hline
      \textit{T. harzianum} & 13388 & 8009 & 3314 & 2065  \\ \hline
      \textit{T. virens} & 12045 & 7537 & 3715 & 793  \\ \hline
    \end{tabular}
  \end{center}
  \caption{Counts of regions identified in total and total number of
    regions where a prediction from each individual tool was
    found. Partial agreement values for DC1 and Tsth20 are set as N/A
    as there were only two tools in consideration.}
  \label{table:regioncounts}
\end{table}

The results of the region finding process when applied to gene calls
show a mix of agreement and disagreement between the tools considered
here. While regions of complete agreement make up the majority of
regions in all assemblies, there are more partial agreements and
singletons than one would expect under the assumption that gene
finding tools are equal. Both DC1 and Tsth20 have a large number of
singleton regions present in comparison to the RefSeq datasets.


\section{Genes in Regions of Anomalous GC Content}

Evaluating gene finder performance in regions of anomalous GC content
is one of the key topics of this research. One simple way to evaluate
performance is whether or not gene finding tools predict genes
uniformly throughout a given sequence. Biologically, we know that
regions of anomalous nucleotide composition are less likely to contain
coding sequences than typical genomic regions, leading us to the
problem of first identifying predicted genes in standard and anomalous
regions. After identifying low GC segments within each assembly, we
can include them in the region identification method. From this
result, we classified predicted genes into two classes; genes in
regions with normal GC content, and genes in regions with anomalous
content. In this case, anomalous content is defined as a window of
genomic sequence containing a percent GC composition of 28\% or
lower. This number was chosen based on the plots of GC content
presented in the assembly section of the results. After classifying
predicted genes, two-sided binomial tests were performed with the null
hypothesis being that predicted genes are distributed uniformly
throughout an assembly. Framed differently, we expect the sum of genes
predicted in both regular and irregular regions to be proportional to
the sum of lengths of those regions, respectively. This is not the
case as demonstrated in table~\ref{table:gc-binomial}.

\begin{table}
  \begin{center}
    \begin{tabular}{|c|c|c|c|c|c|}
      \hline
      Tool & DC1 & Tsth20 & \textit{T. reesei} & \textit{T. harzianum} & \textit{T. virens} \\ \hline
      Braker2 & $9.56^-181$ & $1.14^-259$ & $2.68^-96$ & $4.05^-140$ & $1.35^-35$ \\ \hline
      GeneMark & $5.12^-216$ & $0.0$ & $5.66^-49$ & $5.37^-219$ & $5.31^-35$ \\ \hline
      RefSeq & N/A & N/A & $1.29^-49$ & $2.44^-205$ & $7.40^-33$ \\ \hline
    \end{tabular}
  \end{center}
  \caption{\textit{p} values produced from a two-sided binomial test
    for each combination of tool and assembly.}
  \label{table:gc-binomial}
\end{table}


\chapter{Data and Methodology}
\label{chap:methods}
\section{Methodology}

\section{Workflow Overview}

The general methodology for this work is described in figure
\ref{fig:workflow}. Each portion of this figure is discussed in detail
in this section.

\begin{center}
  \begin{figure}
    \includegraphics[width=1.15\textwidth]{./figures/data-flowchart.pdf}
    \caption{A flowchart of the methodology followed for this
      research. The workflow is broken up into sections based on the
      stage of the pipeline. Oval-shaped nodes represent steps
      involved in the processing of data, while rectangular nodes
      represent intermediate datasets and results.}
    \label{fig:workflow}
  \end{figure}
\end{center}
      
\section{Assembly and Annotation}

\subsection{Assembly}

Sequencing for the assembly process came in the form of Nanopore and
Illumina sequence data. Nanopore data was not processed prior to
assembly. Illumina sequencing data was filtered using Trimmomatic with
filtering criteria as follows: SLIDINGWINDOW:4:28, LEADING:28,
TRAILING:28, MINLEN:75. Only surviving paired-end reads were used for
further analysis.

Genomic Nanopore sequences from DC1 and Tsth20 were assembled in to
contigs using NextDenovo\ref{Hu02024} v2.5.0 and then polished with
Illumina sequences using NextPolish\ref{Hu2020} v1.4.1. Assembly and
polishing were both performed using default parameters.

%In an attempt to produce high quality assemblies of DC1 and Tsth20, We
%decided on a set of tools named NextDenovo and NextPolish as they have
%produced excellent assemblies based on previous experience. (should
%find a citation to confirm this)

%(Might be better for discussion or omitted since it is specific to our
%setup) Initial attempts to run the example dataset resulted in
%permissions errors due to the management of the storage system being
%used, which were encountered with other tools in the past. To remedy
%this, the software installation was copied to RSMI's scratch space on
%Copernicus. Once the approriate permissions were given to run
%nextDenovo, the example dataset was run without issue.

%Following assembly using nextDenovo, Illumina sequence data from DC1
%and Tsth20 was used to polish each respective genome using
%nextPolish. Default parameters were used from assembly except for
%modification of the parallel option to reduce processing times.

%\subsection{Repeat Masking}

%In order to evaluate the performance of gene finding tools in
%repetitive or low complexity regions in the context of
%\textit{Trichoderma} genomes, we must first identify said regions in
%the genomes considered. To do this, the GenericRepeatFinder tool was
%used, which is a \textit{de novo} repeat detection tool
%\cite{10.1104/pp.19.00386}. GenerifRepeatFinder detects three
%different types of repeats, those being MITEs, TDRs and TIRs. Commands
%used for this program follow the example commands provided on the
%GitHub page for the GenericRepeatFinder project.

\subsection{Gene prediction with GeneMark-ES}

\textit{Ab initio} gene finding was performed using GeneMark-ES
v4.71. Default parameters were used in all cases except for the fungal
option, which was set to yes in order to use a GeneMark model specific
to fungal genomes. Convenience options were also used to produce GFF3
output annotation files and an increased number of threads to reduce
processing time.

%General command structure for GeneMark-ES:
%
%gmes\_petap.pl --ES --fungus
%--format gff3 --cores 48 --sequence /path/to/sequence

\subsection{Braker2}

For evidence-based gene finding, Braker2 v3.0.2 was used with
\textit{Trichoderma reesei} selected as the reference or gold-standard
genome to use as evidence. RNAseq datasets from \textit{T. reesei}
were downloaded from the NCBI short-read archive using the sra-toolkit
and were not trimmed or filtered prior to their use with
Braker2. Default parameters were used to train a Braker2 model on
\textit{Trichoderma reesei} data except in the case of the fungal
option, which was used for this analysis for improved gene-finding
performance in fungi. The Braker2 \textit{T. reesei} model was then
applied to all assemblies.

%The variables that need to be set are AUGUSTUS\_CONFIG\_PATH and
%TSEBRA\_PATH. Augustus, by defuault, tries to write species
%information to the location where the software is installed. In this
%case, we don'thave write permissions to the compute canada software
%stack hosted byt Research Computing, so the AUGUSTUS\_CONFIG\_PATH
%variable must be set in order to create a writeable directory. As long
%as that path has a directory within it called braker, and a species
%directory within the braker directory, things should go
%smoothly. TSEBRA is a set of scripts also made by the creators of
%Braker and is required to merge results from the various gene
%prediction tools involved in the Braker2 pipeline. The TSEBRA\_PATH
%simply points to the directory where TSEBRA is located Both Braker2
%and TSEBRA can be cloned directly from GitHub (links to come)

\section{BUSCO Performance}

To assess the general performance of gene-finders, BUSCO v5.7.1 was
selected as a tool to determine a gene-finder's ability to predict a
set of conserved ortholagous genes. The BUSCO v5.7.1 Metaeuk pipeline
was run using the Odb10 fungal lineage dataset to capture highly
conserved genes in fungal genomes. This pipeline was applied to all
genome assemblies used in this work. Used transcriptome mode... This needs to be discussed.

\section{Region Identification}

Feature Identification: To first undertand how gene prediction tools
perform in comparison to other gene prediction tools, we must identify
features. This identification of features will help us descirbe the
similarities, and differences between gene finding tools. A feature,
in this context, is any feature stated within a Genomic Feature Format
file (GFF) provided to the program, in which mutliple GFF files can be
provided. The definition of a feature, for this application, is an
object that contains a contig ID, a start position, an end position
and a strand property. In the context of features on different
strands, start and stop positions of features are sorted based on left
and right positions of the feature in respect to the reference
sequence.

Region Identification: In addition to feature creation, we will also
identify regions of overlapping features based on the precitions from
each gene finding tool. These regions will help identify the
agreements, or disagreements, between different gene-finding tools. A
region, in this context, is a set of overlapping features, all of
which overlap at least one other feature in the region. With each
overlap, there will be an overlap type. These types can be defined
based on Allen's Interval Calculus (reference), with the exception of
features that start beyond the end point of the current region.

=======
Example command for braker2:

/scratch/p2irc/p2irc\_rsmi/cbe453/masters/software/braker2/BRAKER/scripts/braker.pl
--gff3 --threads 60
--TSEBRA\_PATH=/scratch/p2irc/p2irc\_rsmi/cbe453/masters/software/braker2/tsebra/TSEBRA/bin/
--genome /path/to/sequence --species=TreeseiFungal --fungus
--useexisting

BUSCO methodology (from research questions)
The BUSCO method was applied using two BUSCO subsets,
one generally applicable for fungi, and another targeting an
evolutionary branch more closely related to \textit{Trichoderma}.

Stats for length analysis (from research questions)
The first statistical tool to be applied is ANOVA (analysis of
variance) to compare the mean lengths genes predicted by each gene
finding tool with the null hypothesis being that the mean of predicted
gene lengths should be the same across all tools considered. In
addition to ANOVA, pairwise comparisons of the distributions using a
Kolmogorov–Smirnov test is appropriate. The null hypothesis in this
test would be that the gene lengths are sample from the same
distribution.

Stats for binomial tests (from research questions)
To do this, a binomial test will be used, with the null hypothesis
being that the number of genes predicted in regions of normal and
abnormal GC content should be proportional to the length of normal and
abnormal GC content regions in the assembly. For example, if 30
percent of the genome is comprised of anomalous GC content, then we
would expect 30 percent of predicted genes to be present in those
regions. In addition to anomalous GC content, this test can be applied
to repetitive content in assemblies as well.

Stats for regions (from research questions)
From these results, Venn diagrams will be generated with
Jaccard index calculated for each combination of gene finding
tools. The region identification process can also be extended to
include features identified by other tools, such as BLAST hits to
validated gene models from other organisms and small RNAs. Chi2
goodness of fit tests can then be applied to counts of 'validated'
gene predictions or other features with the same null hypothesis that
gene finders should predict the same number of features.

%\section{Analysis of Results}

After completion of the processing portion of this work, the results
must be processed in a useful way, which includes both the biological
implications of the gene calls as well as the computational, or gene
finding features, of the selected programs. To better understand
how gene finders perform in these two classes, we must define an
appropriate plan for analysis of the results produced so
far. Currently, downstream analysis plan has been broken down into
several sections.

\subsection{Basic Analysis}

Basic analysis of gene finding results is an important part of this
research. Total gene, transcript and protein counts will be identified
for each genome and gene finding tool combination. Comparing the
general outputs of these programs will provide an idea of their
performance in different \textit{Trichoderma} genomes In addition to
these basic outputs, analysis will also be performed for the
following: distribution of gene lengths, intersection of gene calls,
smallRNAs and repetetive regions, shared gene content with a close
fungal relative. Analysis for these results can be performed through
simple shell scripting with grep and other unix tools, although
processing through Python might provide results that are easier to
reproduce with proper programming. Having one script with several
modules that can be rerun at will would be easier to handle than
multiple shell scripts. This thinking for processing will be applied
to subsequent sections of this as well.

\subsection{Distribution of Gene Lengths}

One important aspect of gene finding tools to consider is the
distribution of gene lengths predicted by each individual
tool. Certain tools, such as GeneMark are based on pre-defined models,
which may limit the length of predicted genes, while tools such as
Braker2, which incorporate RNAseq data, may predict a wider
distribution of gene lengths depending on the input dataset
used. Regardless, the ability of a gene finding tool to predict a
wider range of gene lengths can be usefull if users are looking for
short or larger genes. To help determine whether these tools
find shorter genes, or small RNAs, the genomes of interest have been
annotated using Infernal along with the Rfam database to identify
small RNAs as a ground truth. These annotation results will also be
included with results from other annotation processes further down the
line. Again, these results can be produced with a Python script. The
resulting data could then be used as input to violin plots for each
genome and set of tools considered in this analysis process. Violin
plots should provide a good visualization of gene lengths as well as
the number of genes found with specific lengths. Means could also be
compared staatistically for genomes and the mutliple tools considered
as well.

Analysis of gene lengths was performed using a Python script. Combined
predicted CDS sequences for each predicted gene were used as input for
the total gene length. CDS sequences predicted by Braker2, were
directly available in the output directories when the program was
run. CDS sequences from GeneMark required extraction of the CDS
sequences from the genome FASTA files. This process was performed
using the gffread tool from the Cufflinks package. Predicted CDS
sequences were loaded into Python using Biopython's SeqIO
package. Sequence lengths were then placed in a list and analysed
using a combination of pandas and numpy. A log10 transformation was
applied to the sequence lengths as the original distribution was
heavily skewed due to long outlier CDS sequences. After
transformation, the CDS length distribution appears as a normal
distribution, although there are interesting troughs that occur in
several of the peaks for several of the genomes considered. These
troughs did not appear in a comparable Yeast reference dataset,
although the log transformed data still appears to be a normal
distribution.

\subsection{Intersection of Gene Calls, smallRNAs and Repetitive Regions}

Annotation of all three features in the title are important in
assessing the ability of gene finding tools. Even more important, is
the potential for overlap between gene calls and small RNAS as
well as repetitive regions the genomes. As discussed in the previous
subsection, the distribution of gene lengths predicted by a gene
finding tool can be an important metric for users. Overlapping
predicted genes from tools alongside the output from Infernal and the
Rfam database may provide insight into whether these gene
finding tools are able to predict RNAs of very short length. In
addition to small RNAs, repetitive regions in \textit{Trichoderma}
genomes hold potential for recombination and gene content, although
the inherent nature of these repetitive regions (low nucleotide
diveristy) suggests that gene content should be low, based on the
nucleotides required for start and stop codons. Analysis of these
intersections can be perfomed via bedtools or through biopython (I
believe). Again, having all processing steps included in one script as
separate functions that can be called at whim will make further
processing easier if changes need to be made.

\subsection{Methodology for Indetifying Overlapping Features}

To analyse the results from multiple gene-finding tools, we must first
define two conceptual topics.

Definition of a Feature: First, a feature, in the context of this
research, is any item contained within a Genomic Feature Formated file
(GFF). Each feature contains a contig ID, a start position, end
position, and a feature ID based on the information from the GFF
file. These features will be used in the process of identifying
regions, or overlapping features from prediction tools.

Definition of a Region: Secondly, a region is defined as any overlap
between features relative to the reference sequence being
considered. To identify regions, every feature from each GFF file
being considered, is sorted by start position. For clarification, the
start position is based on the left most position in the GFF file,
regardless of the strand that the feature is predicted on. Once the
features have been sorted by left position, the sorted features are
iterated over to identify regions, as long as the left position of the
next feature is consistent with the left position, within the left
and right position, or equal to the right position of the current
region. One drawback to this approach is that ideally, identification
of overlap types based on Allen's interval algebra would be performed
at this point. However, the methodology of initially sorting features
based on left position somewhat prevents this process from
happening. This implementation requires further processing of
identified regions late on the process, simplifying the all to all
comparison that would occur if all features were considered at once.

Identification of overlapping features results in different 'classes'
of regions. A few basic cases are outlined in figure
~\ref{fig:igv-cases}. The simplest of case is a region where the
gene finders agree unanimously on the gene (agreement currently means
same start and end point). 


\begin{figure}[b]
  \centering
  \begin{subfigure}{0.9\textwidth}
    \includegraphics[width=\textwidth]{figures/igv/igv-agreement-thin.png}
    \label{fig:igv-agree}
    \caption{Complete Agreement}
  \end{subfigure}
  \begin{subfigure}{0.9\textwidth}
    \includegraphics[width=\textwidth]{figures/igv/igv-start-stop-thin.png}
    \label{fig:igv-start-stop}
    \caption{Start/Stop Disagreement}
  \end{subfigure}
  \phantomcaption
\end{figure}
\begin{figure}[t]
  \ContinuedFloat
  \centering
  \begin{subfigure}{0.9\textwidth}
    \includegraphics[width=\textwidth]{figures/igv/igv-missing-thin.png}
    \label{fig:igv-missing}
    \caption{Missing Prediction}
  \end{subfigure}
  \begin{subfigure}{0.9\textwidth}
    \includegraphics[width=\textwidth]{figures/igv/igv-complicated-thin.png}
    \label{fig:igv-complicated}
    \caption{Complete Disagreement}
  \end{subfigure}
  \caption[Examples of Potential Regions]{Several visual examples of
    regions using IGV. \textbf{a)} Region where all prediction tools
    are in agreement. \textbf{b)} Prediction tools agree that gene is
    present, but not on the exact start and/or stop
    positions. \textbf{c)} A region where one tool does not predict a
    gene while the others do. \textbf{d)} A region with a combination
    of disagreeing predictions.}
  \label{fig:igv-cases}
\end{figure}

\subsection{Shared Gene Content with Closely Related Organisms}

While considering novel gene calls can be useful, comparing those
calls to a well-studied close relative can provide a rudimentary
validation of the calls as a ground truth. This process will confirm
that at least most of a closely related fungal genome's coding
sequences are predicted and shared by the gene calls for
\textit{Trichoderma}. Results for this processing can be produced with
a simple BLAST search and apropriate cutoff values (i.e. query
coverage, percent identity, E-score, etc.). While running BLAST is a
simple process, the selection of a closely related organism is more
difficult. One initial choice would be to work with
\textit{Saccharomyces cerevisiae}, or bakers yeast, as it is extremely
well studied and would be considered a model organism, similar to
\textit{Arabidopsis} and \textit{Mus musculus}. However,
\textit{Saccharomyces cerevisiae} diverged evolutionarily millions of
years ago, which may make it a poor candidate for a comparative
analysis. The second candidate considered for comparison is
\textit{Fusarium avenaceum}, as it is also well studied and more
closely related to \textit{Trichoderma} than yeast, and has a genome
similar in size to that of \textit{Trichoderma} species, at roughly
40Mb. Finally, comparison of assemblies and predicted genes to another
\textit{Trichoderma} strain is a reasonable approach. In this case,
\textit{Trichoerma atroviride} was selected as it is not included in
the species used in the gene prediction portion of this analysis. From
these assemblies, the RefSeq proteins (queries) from NCBI will be used
in a tblastn search against each genome sequence from DC1, Tsth20,
\textit{T.reesei}, \textit{T. harzianum}, and \textit{T. virens}
(subjects). Resulting BLAST hits will then be filtered based on
suitable alignment coverage and identity, which will be determined by
the reference sequence being considered. Total number of BLAST hits
reported for each organism will provde information about completeness
of assemblies and overall coverage of the gene/coding sequence space
in the query sequences (need to be clear with language used for blast
subjects and query). In addition, BLAST hits will be analysed with the
region identification approach to identify coverage of protein and coding
sequences in relation to gene predictions generated previously.

\subsection{BUSCO Analysis}
Another method for assessing the completeness of a set of predicted
genes is Benchmarking Universal Single-Copy Orthologue (BUSCO)
analysis\cite{10.1093/molbev/msab199}. BUSCO analysis is similar to
analysis of overlapping or shared gene content with a close relative
in that we are comparing the predicted gene sets to an existing
standard or reference. With BUSCO analysis, the reference set has a
far more strict definition. The datasets used for BUSCO analysis are
single-copy orthologs generally found in a genome of
interest. What this means is that BUSCO searches for single-copy genes
that should be present in an organism based on the database selected
for analysis. As an example in fungi, if one were interested in
assessing the completeness of their annotated genes in a similarly
related fungi, there should be a set, or subset, of single-copy genes
present in the new annotation that are expected to be present after
evolutionary divergence(...). This can be thought of as similar to a
'core' gene set. Results from BUSCO analysis are typically reported in
a percentage of the gene set included in the BUSCO
dataset. Percentages of single and duplicated hits are reported as
well. While a high reported coverage of the BUSCO dataset is
considered good, it is not fully indicative of excellent gene finding
performance

\subsection{Comparative Genomics}

With the data produced by this research, it is possible to perform som
commparative genomics (time permitted), mostly related to the
assemblies generated during this work along with the RefSeq genomes
included from NCBI. Mummer is a potential tool to use for all to all
genome alignments, although there may be difficulty in the ordering of
contigs/scaffolds/chromosomes when performing thes alignments. This
work is not necessarily required but would be interesting from a
biological perspective to identify rearrangements, inversions and
such.


\chapter{Results and Discussion}
\label{chap:results}
\section{Overview}

Results from this work are ordered in a particular manner relative to
the order in which data was processed. Since there are novel
\textit{Trichoderma} assemblies included in this work, the results
begin with an overview of the assemblies of DC1 and Tsth20 in Section
\ref{section:assemblies}. Following this, is Section
\ref{section:profiling}, which discusses the installation, application
and other features associated with the gene finders selected for this
work. Results of the initial gene finding process are then presented
in Section \ref{section:gene-finding} followed by more detailed
analysis of predicted CDS lengths in Section
\ref{section:lengths}. Gene predictions from each gene finder and
assembly combination are supplied to the BUSCO metric process in
Section \ref{section:busco}. Results from the initial spatial
comparison of gene predictions using the region identification process
are then presented in Section \ref{section:regions}. The region
identification process is then applied with additional information
from tblastn alignments, InterProScan annotation, and GC sequence
composition in Sections \ref{section:interproscan},
\ref{section:blast} and \ref{section:gc-regions}, respectively. A
final comparison of all results and selection of an appropriate gene
finder are then discussed in the Conclusion \ref{chapter:conclusion}.

RefSeq assemblies in these results are referred to by their scientific
name for context and not by their associated NCBI accession
IDs. Instead, NCBI accessions for the RefSeq assemblies and their
associated annotations are listed here: \textit{T. reesei} -
GCF\_000167675.1\_v2.0, \textit{T. harzianum} -
GCF\_003025095.1\_Triha\_v1.0, GCF\_000170995.1\_TRIVI\_v2.0.

\section{Assemblies of DC1 and Tsth20}

For general assembly metrics of DC1 and Tsth20, the QUAST tool was
used. Results from QUAST are shown in figure~\ref{table:assemblies},
from which we can make several observations. In DC1 and Tsth20, the
total contig counts are an order of magnitude smaller when compared to
the other NCBI RefSeq assemblies, inidicating highly contiguous
assemblies from nextDenovo and nextPolish. This is likely due to the
use of long-read sequencing used in the assemblies of DC1 and
Tsth20. The total assembly lengths are similar, ranging from 38Mb to
42Mb, except in the case of \textit{T. reesei}, which is known to have
a significantly smaller genome length \cite{Kubicek2019} at roughly
33Mb. The largest contig size for each assembly varies greatly. DC1
and Tsth20 have the largest contigs of all assemblies being
considered, which is again likely due to the inclusion of long-read
sequencing data in the assembly process. The N50 values for all
assemblies are above 1Mb, with DC1 and Tsth20 N50s being at minimum
three times larger than others assemblies. Results from this table
indicate that the assemblies of DC1 and Tsth20 are more contiguous
than the assemblies of \textit{Trichoderma reesei, harzianum and
  virens} also considered in this analysis. While contiguity is not
the sole indicator of genome quality, it does provide confidence in
the quality of the input data and resulting assemblies.

\begin{table}
  \begin{center}
    \begin{tabular}{|c|c|c|c|c|c|c|}
      \hline
      Strain & Total Contigs & Total Length & Largest Contig & GC\% & N50 & L50 \\ \hline
      DC1 & 8 & 38.6 Mb & 11.49 Mb & 47.97 & 5.69 Mb & 3 \\ \hline
      Tsth20 & 7 & 41.58 Mb & 8.02 Mb & 47.33 & 6.52 Mb & 3 \\ \hline
      \textit{T. harzianum} & 532 & 40.98 Mb & 4.08 Mb & 47.61 & 2.41 Mb & 7 \\ \hline
      \textit{T. virens} & 93 & 39.02 Mb & 3.45 Mb & 49.25 & 1.83 Mb & 8 \\ \hline
      \textit{T. reesei} & 77 & 33.39 Mb & 3.75 Mb & 52.82 & 1.21 Mb & 9 \\ \hline
    \end{tabular}
  \end{center}
  \caption{General assembly metrics produced by QUAST (a
    genome quality assement tool).}
  \label{table:assemblies}
\end{table}

During initial investigation of the sequences used as input to the
assembly process, we observed that the reads contained abnormal ratios
of GC content. To see if this observation extended to the assemblies
as well, sliding windows of GC content were calculated for all
assemblies included in this analysis. The results of this analysis are
shown in figure~\ref{fig:assembly-gc}. Of the included assemblies,
anomalous GC content in the form of AT-rich sequences were identified
in DC1, Tsth20, \textit{T. reesei} and \textit{T. harzianum}, with
\textit{T. virens} showing no anomalous GC content. Anomolous GC
content is visualized on the left tails of the distributions with a
local peak around sequences containing 10 percent GC content. In
addition to the confirmation of anomalous GC content, it appears that
the distribution of GC content in \textit{T. reesei} differs from the
other assemblies. The curve of GC content for \textit{T. reesei},
visualized in green in \ref{fig:assembly-gc}, lies farther to the
right, indicating higher GC content in its assembly. While the left
tail of the curve also shows an increase in AT rich sequence
composition, it is shifted farther right than other
\textit{Trichoderma} assemblies. Investigation of these anomalous
regions is continued in section ...

\begin{figure}
  \begin{center}
    \includegraphics[width=0.8\textwidth]{figures/gc-plot.pdf}
  \end{center}
  \caption{Plots showing the frequency of GC values calculated from
    sliding windows for each assembly.}
  \label{fig:assembly-gc}
\end{figure}



\section{Profiling Gene Finding Tools}

While Braker2, GeneMark and RefSeq all provide lists of possible genes
for a provided reference, the implementation of each tool is
different, requiring more or less effort to install and run depending
on the tool used. The computing platform used in this research is
hosted and managed by University of Saskatchewan services, which is
modelled around the HPC platform used by the Digital Research Alliance
of Canada and software is managed similarily. 

First, we will briefly discuss the RefSeq annotation process. RefSeq
annotation is only applied to data that is submitted to NCBI. The
RefSeq Eukaryotic Genome Annotation Pipeline\cite{NCBI2024} is a
genome annotation process developed and maintained by NCBI. The
pipeline is not directly publicly available to public users, and
requires submission of data to NCBI. Once data is submitted to NCBI,
the RefSeq annotation pipeline may be applied upon request only if the
genome is the highest quality assembly for the species in question or
if the genome is of significant interest to the scientific community,
limiting reach of the annotation process to many users. The pipeline
supplies existing RNAseq, CDS and protein sequences to NCBI's in-house
gene prediction tool Gnomon, which produces trained models for gene
prediction. While the tools used for alignment and processing of
supporting sequence information are listed, the inner workings of
Gnomon are not well documented, at least from the public perspective,
and I was unable to find Gnomon in any compilable or executable form
during my search. Recreation of the RefSeq pipeline would prove
extremely challenging if not impossible without supporting
information. Run times for this pipeline are difficult to determine
due to the hidden nature of the pipeline, unknown compute resources
and varying quantities of data used. The RefSeq annotation process
produces comprehensive outputs, including CDS sequences, translated
CDS, RNA from genomic sequences, proteins, feature counts and tables,
and finally GFF and GTF formatted annotation files for these features.

Next we will discuss the handling, installation, and running of both
Braker2 and GeneMark packages. Several points will be discussed for
each tool, with those points being sourcing and downloading,
components and prerequisites, installation, execution, and
output. GeneMark\cite{10.1093/nar/gki937} is a gene finding tool
developed by the Georgia Institute of Technology with packages
prepared for Linux and MacOS. It is provided as licensed product in
the form of a package which can be downloaded from their website after
submitting a form. Once the necessary information is submitted, the
user is provided with a key that must be placed in the appropriate
location once the software is downloaded and unpacked. The core
controlling methods of GeneMark are written in Perl, accompanied by
several Python scripts and compiled exectuables. GeneMark was tested
by the developers with Perl version 5.10, and Python 3.3. A number of
Perl dependencies are also required, which can be installed via
YAML. The user will have to know which implementation of GeneMark they
are wanting to use for their application, as GeneMark has several
variations it can run depending on the desired application. In this
work, the GeneMark-ES variant of GeneMark was executed, as it is the
self-training \textit{ab initio} GeneMark method for eukaryotic
organisms .Options required by GeneMark at runtime are documented in
the help message, and simple enough that any user with familiarity of
bioinformatics tools should be able to run GeneMark, although
documentation for use is only provided by the help message when
running the program and not online. In regards to run-time, running
the GeneMark-ES pipeline on DC1 with 56 threads finished in 16
minutes. Upon completion, GeneMark produces a GTF or GFF file of
predicted genes as well as a number of other outputs related to the
run.

Braker2\cite{Bruna2021} is hosted on GitHub as a repository that
receives relatively frequent updates with Braker3 being released while
working on this thesis. Braker is maintained by Katharina Hoff from
the University of Greifswald and is available under the Open Source
Artistic License. Installation of the repository is a straightforward
pull from GitHub. As with GeneMark, Braker2 uses a combination of
Perl, Python and other exectuables in its regular use. Downloading the
repository itself is not enough for execution, as Braker2 relies on a
number of dependencies and bioinformatics tools including Perl and
(Perl dependencies), Augustus, BamTools, BedTools, GeneMark,
StringTie, GFFRead and a few others. Manual installation of these
dependencies would be difficult, time consuming and in general advised
against. In this case, many of Braker2's requirements are satisfied by
modules already included in the environment, making installation
relatively simple if you know the ins and outs of the Digital Research
Alliance of Canada's software stack. This case still required
installation of some Perl modules in addition to loading necessary
modules. Alternatively, one could use a package manager like Anaconda
or Minoconda to handle installation of packages. This is perfectly
reasonable, but also requires knowledge specific to Anaconda, which
can be complicated and frustrating for users with little software
management experience. The newly released Braker3, also includes a
containerized version of the software, allowing users to build and
execute Braker3 with ease in its own environment. Once installation is
finished, the Barker2 pipeline is relatively straightforwad to run as
well, with excellent documentation included both online and through
the built in help message. In this research, the Braker2 pipeline was
run in two modes. The first mode is a training mode, where sequence
files are supplied to Braker2 to train a gene-calling model. The
training and gene finding steps are run as part of the same command,
and a model built using the training data is saved for bookkeeping and
future use. The training was performed using roughly 145 million
Illumina paired end RNAseq reads on the \textit{Trichoderma reesei}
genome. The training, including RNAseq alignments, and gene finding
pipeline in this case took 1 hour and 17 minutes using 60
threads. Applying the Braker2 trained gene finding model to DC1 with
60 threads took 21 minutes to complete. Run times will of course vary
depending on processing power available to the end user, but in the
case of \textit{Trichoderma} genomes, users can expect quick results
with relatively little computing power. Once annotation is complete,
Braker2 produces a GFF file containing predicted genes along with CDS
sequences and amino acid sequences their protein products. Braker2,
when trained with RNAseq data, includes an option to predict UTR
regions, but it is experimental and is not performed by default and is
not included in these runs.

In summation, Braker2 and GeneMark are not direct plug-and-play
software packages. Users should expect to encounter issues when
getting these programs running in addition to normal downloading and
unpacking of software packages so some expertise is
recommended. Neither Braker2 nor GeneMark require users to compile
software, however Braker2's dependencies may require additional
compilation and attention. Once installed, both tools are relatively
simple to use with documention available for both on the commandline
and excellent documentation available for Braker2 on their GitHub
page. Outputs from both tools are similar although Braker2 has the
ability to output coding and amino acid sequences for downstream
processing. Both tools run in reasonable amounts of time, where in the
case of smaller genomes such as \textit{Trichoderma}, users can expect
results within a few hours to a day depending on number of computing
cycles available to them.

\section{Initial Gene Finding Results} 

Counts of genes predicted by Braker2, GeneMark and RefSeq are shown in
table \ref{table:gene-counts}. Immediately we see that Braker2
predicts far fewer genes in all assemblies, except in the case of
\textit{Trichoderma reesei.} This is possibly due to the effects of
training the Braker2 gene model using data from \textit{Trichoderma
  reesei}, which has a significantly smaller genome in comparison to
other \textit{Trichoderma} assemblies, although genome size is not
always indicative of gene content. Regardless, there is a significant
difference in the number of genes predicted by Braker2 in comparison
to GeneMark and RefSeq. The number of genes predicted by GeneMark and
RefSeq are similar, except in the case of \textit{T. harzianum}, in
which RefSeq predicts roughly 17\% more genes than GeneMark. Braker2
consistently predicts more transcripts than GeneMark and
RefSeq. RefSeq also appears to predict multiple transcripts for each
gene but in fewer numbers than Braker2. Transcript prediction counts
in \textit{T. harzianum} from RefSeq are also interesting, with RefSeq
predicting fewer transcripts than genes. Why this is occurs is unknown
but may warrant further investigation.

\begin{table}
  \centering
  \begin{tabular}{|c|c|c|c|c|c|c|}
    \hline
    Assembly & Braker2 & & GeneMark & & RefSeq & \\ \hline
     & Genes & Transcripts & Genes & Transcripts & Genes & Transcripts \\ \hline
    DC1 & 8546 & 8637 & 11353 & 11353 & N/A & N/A \\ \hline
    Tsth20 & 8784 & 8858 & 12362 & 12362 & N/A & N/A \\ \hline
    \textit{T. reesei} & 9659 & 10175 & 9196 & 9196 & 9109 & 9118 \\ \hline
    \textit{T. harzianum} & 8314 & 8385 & 12164 & 12164 & 14269 & 14090 \\ \hline
    \textit{T. virens} & 7801 & 7863 & 11866 & 11866 & 12405 & 12406 \\ \hline
  \end{tabular}
  \caption[Gene prediction counts]{Number of genes predicted by each
    gene finder for each \textit{Trichoderma} genome.}
  \label{table:gene-counts}
\end{table}



\section{Distribution of Predicted Coding Sequence Lengths}\label{section:lengths}

To better understand and compare the distributions of CDS sequences
predicted by Braker2, GeneMark, and RefSeq, the cumulative
distribution function for the lengths of CDS sequences for each gene
finding tool are shown in Figures~\ref{fig:cdf-lengths-1},
~\ref{fig:cdf-lengths-2} and~\ref{fig:cdf-lengths-3}. The $\log_{10}$
values of gene lengths were used as the abscissa for a better
visualization of the distributions. In DC1, the curves from Braker2 and
GeneMark follow each other closely, with the only variation being
genes of short length, where Braker2's curve extends beyond that of
GeneMark, indicating that Braker2 predicts the shortest genes in all
assemblies except \textit{T. virens}. In the case of Tsth20, the
curves are nearly identical. In \textit{T. reesei}, we see
disagreement in the curves for shorter genes, with Braker2 appearing
to predict a larger fraction of shorter genes than GeneMark and
RefSeq. The right sides of the curves trend toward similar predicted
gene lengths. In \textit{T. harzianum}, RefSeq deviates from GeneMark
and Braker2, predicting more genes of short length, while Braker2 and
GeneMark appear to be in near-complete agreement except in the case of
very short genes. Finally, in \textit{T. virens}, we see that RefSeq predicts a larger fraction of shorter genes once again,
although the deviation is not as drastic as in
\textit{T. harzianum}. From these plots, we can say that visually, gene
finding tools appear to predict different lengths of genes. We also
observe that Braker2 typically predicts the shortest genes of all
three gene finding methods.

\begin{figure}
  \centering
  \begin{subfigure}{\textwidth}
    \makebox[\textwidth]{\includegraphics[width=1.1\textwidth]{figures/dc1-cdf-lengths-log.pdf}}\label{fig:dc1-lengths}
    \caption{DC1}
  \end{subfigure}
  \begin{subfigure}{\textwidth}
    \makebox[\textwidth]{\includegraphics[width=1.1\textwidth]{figures/tsth20-cdf-lengths-log.pdf}}\label{fig:tsth20-lengths}
    \caption{Tsth20}
  \end{subfigure}
  \caption[CDF plots for DC1 and Tsth20]{Plots of the cumulative distribution
    function for CDS lengths produced be each gene finding tool
    applied to DC1 (a) and Tsth20 (b)}\label{fig:cdf-lengths-1}
\end{figure}

\begin{figure}
  \centering
  \begin{subfigure}{\textwidth}
    \makebox[\textwidth]{\includegraphics[width=1.1\textwidth]{figures/t-reesei-cdf-lengths-log.pdf}}\label{fig:treesei-lengths}
    \caption{\textit{T. reesei}}
  \end{subfigure}
  \begin{subfigure}{\textwidth}
    \makebox[\textwidth]{\includegraphics[width=1.1\textwidth]{figures/t-harzianum-cdf-lengths-log.pdf}}\label{fig:tharzianum-lengths}
    \caption{\textit{T. harzianum}}
  \end{subfigure}
  \caption[CDF plots for \textit{T. harzianum} and \textit{T. reesei}]{Plots of the cumulative distribution
    function for CDS lengths produced be each gene finding tool
    applied to \textit{T. reesei} (GCF\_000167675.1\_v2.0)(a) and
    \textit{T. harzianum.} (GCF\_003025095.1\_Triha\_v1.0)(b)}\label{fig:cdf-lengths-2}
\end{figure}

\begin{figure}
  \centering
  \begin{subfigure}{\textwidth}
    \makebox[\textwidth]{\includegraphics[width=1.1\textwidth]{figures/t-virens-cdf-lengths-log.pdf}}\label{fig:tvirens-lengths}
    \caption{\textit{T. virens}}
  \end{subfigure}
  \caption[CDF plots for \textit{T. virens}]{Plots of the cumulative distribution
    function for CDS lengths predicted by each gene finding tool when
    applied to \textit{T. virens} (GCF\_000170995.1\_TRIVI\_v2.0).}\label{fig:cdf-lengths-3}
\end{figure}

To confirm that CDF curves differ, two-sided two-sample
Kolmogorov-Smirnov\cite{2008} tests were performed using the
$\log_{10}$ transformed gene lengths, with the null hypothesis being
that the difference between the distributions of frequencies of gene lengths predicted by
 any two tools can be explained by random chance. Results are presented in Table~\ref{table:ks-2s}. In the cases of DC1 and Tsth20, we see that in
agreement with Figure~\ref{fig:cdf-lengths-1}, Braker2 and GeneMark do
not produce statistically different lengths of genes, which is
interesting considering that the Braker2 includes experimental
evidence for another assembly while GeneMark does not. In
\textit{T. reesei}, Braker2's predicted gene lengths are significantly
different from both RefSeq and GeneMark, which is also evident in the
CDF plots. The same cannot be said for \textit{T. harzianum} and
\textit{T. virens}, where RefSeq is significantly different from both
GeneMark and Braker2, which are not significantly different from each
other. It is also notable that RefSeq and GeneMark predict similar
gene lengths in \textit{T. reesei}, but not in \textit{T. harzianum}
and \textit{T. virens.}

\begin{table}
  \begin{center}
    \begin{tabular}{|c|c|c|c|c|c|c|}
      \hline
      Genome & Tool \#1 & Tool \#2 & \textit{P}-value  \\ \hline
      DC1 & Braker2 & GeneMark & $0.999$ \\ \hline
      Tsth20 & Braker2 & GeneMark & $0.965$ \\ \hline
      \textit{T. reesei} & Braker2 & GeneMark & $9.481*10^{-07}$ \\ \hline
      \textit{T. reesei} & GeneMark & RefSeq & $0.002$ \\ \hline
      \textit{T. reesei} & Braker2 & RefSeq & $1.340*10^{-07}$ \\ \hline
      \textit{T. harzianum} & Braker2 & GeneMark & $0.863$ \\ \hline
      \textit{T. harzianum} & GeneMark & RefSeq & $4.313*10^{-52}$ \\ \hline
      \textit{T. harzianum} & Braker2 & RefSeq & $4.674*10^{-55}$ \\ \hline
      \textit{T. virens} & Braker2 & GeneMark & $0.635$ \\ \hline
      \textit{T. virens} & GeneMark & RefSeq & $7.352*10^{-12}$ \\ \hline
      \textit{T. virens} & Braker2 & RefSeq & $1.794*10^{-09}$ \\ \hline
    \end{tabular}
  \end{center}
  \caption[Results of Kolmogorov-Smirnov tests]{Table of \textit{P}-values from two-sided two-sample
    Kolmogorov-Smirnov tests between gene finding tools.}\label{table:ks-2s}
\end{table}

It can clearly be stated that these gene finding tools predict
different distributions of gene lengths, particularly in
\textit{T. reesei, T. harzianum} and \textit{T. virens}. Why that may
be the case is difficult to answer, but warrants further investigation. There
is clearly an underlying difference between RefSeq and the other gene
finding tools. Braker2 and GeneMark tend to be in agreement, except in
the case of \textit{T. reesei}, for which Braker2 was specifically
trained. We observe that when Braker2 is applied to an assembly from
which the training data originated, Braker2 predicts a larger fraction
of shorter genes. Conversely, we observe that when Braker2 is trained
with experimental evidence and applied to a \textit{Trichoderma}
assembly from which the training data did not originate, the
distributions of predicted genes lengths do not significantly differ
from the \textit{ab initio} gene finder GeneMark.

\section{BUSCO Results}\label{section:busco}

Results of BUSCO analysis using the sordariomycetes\_Odb12 dataset
provided by BUSCO are presented in Figure~\ref{fig:busco-counts} and
Table~\ref{table:busco}. The results indicate that all gene sets
considered in this analysis have a BUSCO completeness of 94.1\% or
higher, with a maximum completeness of 98.4\% in the case of Braker2
and DC1. In general, Braker2 and GeneMark have the most BUSCO complete
sets of gene predictions of the three tools considered. Interestingly,
Braker2 produces far more duplicated BUSCO matches than both GeneMark
and RefSeq. Examining the BUSCO output logs, this appears to be due to
Braker2 predicting more than one coding sequence for some genes
predictions, resulting in multiple similar proteins. Interestingly,
the coding sequences in the RefSeq annotations seem to miss more genes
than the other two gene finders while also having a higher number of
fragmented BUSCO genes.\cbstart~This may be due to human curation of the
RefSeq datasets, or the proprietary Gnomon~\cite{zotero-item-392} gene prediction pipeline used by NCBI
to produce these annotations\cbend. Further investigation is required to
determine the exact cause of this discrepancy. Finally, it appears
that \textit{T. reesei} tends to have slightly lower BUSCO
completeness than the other \textit{Trichoderma} species considered in
this analysis, regardless of gene finder used. Why this is the case is
unknown, but may be due to the Gnomon annotation process used, the fragmented nature of the \textit{T. reesei} assembly used in this
analysis, or that \textit{T. reesei} may not possess some of the missed orthologs. While these results do not capture the entire set of genes
possibly present in these \textit{Trichoderma} assemblies, they do
confirm that the gene finders are at minimum predicting many
evolutionarily conserved fungal genes.

\begin{figure}
  \centering
  \begin{subfigure}{0.9\textwidth}
    \centering
    \includegraphics[width=\textwidth]{figures/busco-complete-counts.pdf}
  \end{subfigure}
  \hfill
  \begin{subfigure}{0.9\textwidth}
    \centering
    \includegraphics[width=\textwidth]{figures/busco-missing-counts.pdf}
  \end{subfigure}
  \caption[BUSCO counts]{BUSCO complete and missing gene counts for each gene finder across all \textit{Trichoderma} genome assemblies. There were a total of 4492 markers in the sordariomycetes\_Odb12 BUSCO dataset. For DC1 and Tsth20, RefSeq annotations are not available, so their values are set to 0.}\label{fig:busco-counts}
\end{figure}

\begin{table}
  \begin{center}
    \begin{subtable}{\textwidth}
      \centering
      \begin{tabular}{|c|c|c|c|c|c|c|}
        \hline
        Strain & Complete & Single & Duplicated & Fragmented & Missing \\ \hline
        DC1 & 4419 & 3547 & 872 & 40 & 33 \\ \hline
        Tsth20 & 4416 & 3585 & 831 & 42 & 34 \\ \hline
        \textit{T. reesei} & 4321 & 3587 & 734 & 80 & 91 \\ \hline
        \textit{T. harzianum} & 4408 & 3572 & 836 & 49 & 35 \\ \hline
        \textit{T. virens} & 4409 & 3530 & 879 & 53 & 30 \\ \hline
      \end{tabular}
      \caption{Braker2}
      \vspace{0.5cm}
    \end{subtable}
    \begin{subtable}{\textwidth}
      \centering
      \begin{tabular}{|c|c|c|c|c|c|c|}
        \hline
        Strain & Complete & Single & Duplicated & Fragmented & Missing \\ \hline
        DC1 & 4415 & 4401 & 14 & 43 & 34 \\ \hline
        Tsth20 & 4409 & 4392 & 17 & 47 & 36 \\ \hline
        \textit{T. reesei} & 4351 & 4345 & 6 & 69 & 72 \\ \hline
        \textit{T. harzianum} & 4399 & 4382 & 17 & 53 & 40 \\ \hline
        \textit{T. virens} & 4391 & 4369 & 22 & 61 & 40 \\ \hline
      \end{tabular}
      \caption{GeneMark}
      \vspace{0.5cm}
    \end{subtable}
    \begin{subtable}{\textwidth}
      \centering
      \begin{tabular}{|c|c|c|c|c|c|c|}
        \hline
        Strain & Complete & Single & Duplicated & Fragmented & Missing \\ \hline
        \textit{T. reesei} & 4274 & 4267 & 7 & 112 & 106 \\ \hline
        \textit{T. harzianum} & 4383 & 4366 & 17 & 60 & 49 \\ \hline
        \textit{T. virens} & 4345 & 4321 & 24 & 92 & 55 \\ \hline  
      \end{tabular}
      \caption{RefSeq}
    \end{subtable}
  \end{center}
  \caption[BUSCO results]{Results from BUSCO using the fungal analysis option
    organized by gene finding tool. The sordariomycetes\_Odb12 dataset
    contains 4492 markers. For more information on the categories
    assigned by BUSCO, please refer to the \href{https://busco.ezlab.org/busco\_userguide.html\#interpreting-the-results}{BUSCO user guide}.}
    \label{table:busco}
\end{table}

While BUSCO matches are a good metric for general performance of gene
finders, it is also important to investigate BUSCO proteins that were missing from the gene predictions. Of interest are BUSCO proteins that are systematically missed by a gene finder in multiple or all assemblies, and whether these BUSCO proteins are also missed by other gene finders. Table~\ref{table:missed-busco-all} lists 17 BUSCO proteins and their corresponding annotations that were not found in any set of predictions from Braker2, GeneMark or RefSeq. The annotations of these BUSCO proteins do not indicate any obvious reason why they would be systematically missed by all three gene finders, and further investigation is required to determine why these genes are not being predicted. It is possible that these genes are simply not present in these \textit{Trichoderma} species, but this may be unlikely given the evolutionary conservation of BUSCO proteins. Each gene finder also has one or more BUSCO proteins that are missed in all assemblies but not by one or both of the other gene finders. These genes are presented in Table~\ref{table:missed-all-gf}. Again, the annotations of these BUSCO proteins do not indicate any obvious reason why they would be systematically missed by a particular gene finder, and further investigation is required to determine why these genes are not being predicted. While not presented here, we also identified a number of other genes that were missed by the gene finders which may provide insight into their performance and the evolution of \textit{Trichoderma}. Overall, it appears that there are very few BUSCO proteins that are systematically missed by all gene finders, indicating that the gene finders are generally capable of predicting the majority of conserved fungal genes.


\begin{table}[h]
  \centering
  \begin{tabular}{|c|c|}
    \hline
    BUSCO ID & Annotation \\ \hline
    191658at147550 & Rossmann-fold NAD (+)-binding protein \\ \hline
    250641at147550 & Aspartic-type endopeptidase \\ \hline
    279527at147550 & Phosphatidic acid-preferring phospholipase A1, contains DDHD domain \\ \hline
    578862at147550 & Transcription factor \\ \hline
    579967at147550 & Alpha-L-rhamnosidase C \\ \hline
    627082at147550 & Ferric reductase \\ \hline
    628206at147550 & Zinc finger domain-containing protein \\ \hline
    636196at147550 & ATP-dependent RNA helicase \\ \hline
    646880at147550 & Lipase class 3 \\ \hline
    647592at147550 & Conserved hypothetical protein \\ \hline
    652375at147550 & Conserved hypothetical protein \\ \hline
    657983at147550 & Conserved hypothetical protein \\ \hline
    658912at147550 & Aromatic amino acid aminotransferase \\ \hline
    659938at147550 & HAUS augmin-like complex subunit 1 \\ \hline
    672116at147550 & Nitrogen regulatory protein AreA, GATA-like domain \\ \hline
    677025at147550 & Mating-type switching protein Swi10 \\ \hline
    689133at147550 & Myosin heavy chain \\ \hline
  \end{tabular}
  \caption[Missing BUSCO IDs]{BUSCO IDs and their corresponding annotations that were not found in any set of predictions from Braker2, GeneMark or RefSeq.}\label{table:missed-busco-all}
\end{table}

\begin{table}[h]
  \centering
  \begin{tabular}{|c|c|c|}
    \hline
    BUSCO ID & Tool & Annotation \\ \hline
    282444at147550 & RefSeq & Conserved hypothetical protein \\ \hline
    291262at147550 & RefSeq & HAUS augmin-like complex subunit 6, N-terminal \\ \hline
    632369at147550 & RefSeq & Peptidyl-prolyl cis-trans isomerase, FKBP-type \\ \hline
    632579at147550 & RefSeq & Cyanate hydratase \\ \hline
    672871at147550 & RefSeq & MAU2 chromatid cohesion factor \\ \hline
    677279at147550 & RefSeq & Pentatricopeptide repeat domain-containing \\ \hline
    688724at147550 & GeneMark \&Braker2 & Putative protein of unknown function \\ \hline
  \end{tabular}
  \caption[Additional missing BUSCO IDs]{Additional BUSCO IDs along with their corresponding annotations not found in any set of predictions by one or more tools but not all. The tools that missed each BUSCO ID are also listed.}\label{table:missed-all-gf}
\end{table}

%\begin{center}
% \begin{table}
% \makebox[\textwidth]{
% \begin{tabular}{|c|c|c|c|c|c|c|c|}
%   \hline
%   Tool & BUSCO ID & Annotation & DC1 & Tsth20 & \textit{T. reesei} & \textit{T. harzianum} & \textit{T. virens} \\ \hline
%   Braker2 & 195619at4751 & \makecell{Pyridoxal phosphate-dependent \\ transferase} & \  & \checkmark & \checkmark & \checkmark & \checkmark \\ \hline
%   Braker2 & 285254at4751 & Aminoacyl-tRNA synthetase & \checkmark & \checkmark & \checkmark &  & \checkmark \\ \hline
%   Braker2 & 348020at4751 & Formyl transferase &  &  & \checkmark &  & \checkmark \\ \hline
%   Braker2 & 497024at4751 & Zinc finger C2H2-type &  & \checkmark & \checkmark & \checkmark & \checkmark \\ \hline
%   GeneMark & 195619at4751 & \makecell{Pyridoxal phosphate-dependent \\ transferase} &  & \checkmark & \checkmark & \checkmark & \checkmark \\ \hline
%   GeneMark & 285254at4751 & Aminoacyl-tRNA synthetase & \checkmark & \checkmark & \checkmark &  & \checkmark \\ \hline
%   GeneMark & 348020at4751 & Formyl transferase &  &  &  &  & \checkmark \\ \hline 
%   GeneMark & 438731at4751 & LSM domain & \checkmark & \checkmark &  & \checkmark & \checkmark  \\ \hline
%   GeneMark & 470813at4751 & Ubiquitin-conjugating enzyme &  &  &  &  &  \\ \hline
%   GeneMark & 497024at4751 & Zinc finger C2H2-type &  & \checkmark & \checkmark & \checkmark & \checkmark \\ \hline
%   RefSeq & 494at4751 & Midasin & N/A & N/A &  &  & \checkmark\\ \hline
%   RefSeq & 315802at4751 & tRNA dimethylallyltransferase & N/A & N/A & \checkmark & \checkmark &  \\ \hline
%   RefSeq & 352224at4751 & YEATS & N/A & N/A &  & \checkmark &  \\ \hline
% \end{tabular}
% }
% \caption[GeneMark missed BUSCO proteins]{The presence (\checkmark)
%   or absence of all BUSCO IDs missed by Braker2, GeneMark and RefSeq
%   in each \textit{Trichoderma} assembly.}
% \label{table:genemark-busco}
%\end{table}
%\end{center}

%\begin{table}
%  \centering
%  \begin{tabular}{|c|c|c|c|c|c|c|}
%    \hline
%    BUSCO ID & Annotation & DC1 & Tsth20 & \textit{T. reesei} & \textit{T. harzianum} & \textit{T. reesei} \\ \hline
%    494at4751 & Midasin & N/A & N/A & X & X & \checkmark\\ \hline
%    315802at4751 & tRNA dimethylallyltransferase & N/A & N/A & \checkmark & \checkmark & X \\ \hline
%    352224at4751 & YEATS & N/A & N/A & X & \checkmark & X \\ \hline
%  \end{tabular}
%  \caption[RefSeq missed BUSCO proteins]{The presence (\checkmark) or
%    absence (X) of all BUSCO IDs missed by RefSeq in each
%    \textit{Trichoderma} assembly.}
%  \label{table:refseq-busco}
%\end{table}

Braker2, GeneMark and RefSeq all demonstrate excellent coverage of the
BUSCO fungal protein set, indicating that these gene finders are
capable of predicting genes that are expected to be present in these
assemblies. From this we can say that the foundations of the
underlying gene models used by each gene finder are solid. Braker2
produces more duplicate matches than GeneMark and RefSeq, but this is
likely due to multiple isoforms of possible genes being present in the
input data. Despite excellent coverage of the BUSCO fungal proteins,
all three gene finders miss some BUSCO proteins in their
predictions. 
Finally, we reiterate the possibility that human curation of RefSeq datasets is responsible for these differences, but this requires further investigation.

\section{Region Identification}

The region identification process begins with identification of
regions sharing gene calls from each tool. For this project, we have
classified regions as complete, partial or singleton. A complete
region is a set of overlaps which contains a feature from each tool
considered in the region finding process. A partial region is a set of
overlaps which includes more than one but not all tools considered. A
singleton is a region in which a feature from only one tool is
present. Table \ref{table:regioncounts} displays the results from the
region finding process when applied to only the features of type
'gene' predicted by each tool.

\begin{table}
  \begin{center}
    \begin{tabular}{|c|c|c|c|c|c|c|}
      \hline
      Assembly & Regions (total) & Complete Agreement & Partial Agreement & Singletons\\ \hline
      DC1 & 11269 & 8483 & N/A & 2786  \\ \hline
      Tsth20 & 12272 & 8737 & N/A & 3535  \\ \hline
      \textit{T. reesei} & 9823 & 8282 & 557 & 984  \\ \hline
      \textit{T. harzianum} & 13388 & 8009 & 3314 & 2065  \\ \hline
      \textit{T. virens} & 12045 & 7537 & 3715 & 793  \\ \hline
    \end{tabular}
  \end{center}
  \caption{Counts of regions identified in total and total number of
    regions where a prediction from each individual tool was
    found. Partial agreement values for DC1 and Tsth20 are set as N/A
    as there were only two tools in consideration.}
  \label{table:regioncounts}
\end{table}

The results of the region finding process when applied to gene calls
show a mix of agreement and disagreement between the tools considered
here. While regions of complete agreement make up the majority of
regions in all assemblies, there are more partial agreements and
singletons than one would expect under the assumption that gene
finding tools are equal. Both DC1 and Tsth20 have a large number of
singleton regions present in comparison to the RefSeq datasets.


\section{Genes in Regions of Anomalous GC Content}

Evaluating gene finder performance in regions of anomalous GC content
is one of the key topics of this research. One simple way to evaluate
performance is whether or not gene finding tools predict genes
uniformly throughout a given sequence. Biologically, we know that
regions of anomalous nucleotide composition are less likely to contain
coding sequences than typical genomic regions, leading us to the
problem of first identifying predicted genes in standard and anomalous
regions. After identifying low GC segments within each assembly, we
can include them in the region identification method. From this
result, we classified predicted genes into two classes; genes in
regions with normal GC content, and genes in regions with anomalous
content. In this case, anomalous content is defined as a window of
genomic sequence containing a percent GC composition of 28\% or
lower. This number was chosen based on the plots of GC content
presented in the assembly section of the results. After classifying
predicted genes, two-sided binomial tests were performed with the null
hypothesis being that predicted genes are distributed uniformly
throughout an assembly. Framed differently, we expect the sum of genes
predicted in both regular and irregular regions to be proportional to
the sum of lengths of those regions, respectively. This is not the
case as demonstrated in table~\ref{table:gc-binomial}.

\begin{table}
  \begin{center}
    \begin{tabular}{|c|c|c|c|c|c|}
      \hline
      Tool & DC1 & Tsth20 & \textit{T. reesei} & \textit{T. harzianum} & \textit{T. virens} \\ \hline
      Braker2 & $9.56^-181$ & $1.14^-259$ & $2.68^-96$ & $4.05^-140$ & $1.35^-35$ \\ \hline
      GeneMark & $5.12^-216$ & $0.0$ & $5.66^-49$ & $5.37^-219$ & $5.31^-35$ \\ \hline
      RefSeq & N/A & N/A & $1.29^-49$ & $2.44^-205$ & $7.40^-33$ \\ \hline
    \end{tabular}
  \end{center}
  \caption{\textit{p} values produced from a two-sided binomial test
    for each combination of tool and assembly.}
  \label{table:gc-binomial}
\end{table}

\section{InterProScan as Supporting Evidence for Predicted Genes}\label{section:interproscan}

Pfam hits from InterProScan analysis are presented in
Figure~\ref{fig:ips-counts} and Table~\ref{table:ips-pfam}, from
which we can identify one major trend. In general, roughly 73\-76\% of
proteins predicted by Braker2, GeneMark and RefSeq contain a match to
a Pfam entry, except in the case of \textit{T. harzianum}, which
reports a considerably lower proportion of genes with Pfam matches in
the RefSeq dataset. Why the proportion of RefSeq proteins with Pfam
matches in \textit{T. harzianum} is so low is unknown. This is
promising performance for the gene finders as the RefSeq annotations
demonstrate similar proportions of Pfam hits to predicted
proteins. The total counts may be deceiving however, as predictions
from Braker2 may result in more than one protein product per gene,
whereas in the case of GeneMark, only one protein is produced per gene
model. From visual inspection of Pfam hits mapped back to the
references it appears that, in general, when gene finders agree that a
gene is present, InterProScan reports the same Pfam match in all three
predictions. An example of agreement between Braker2, GeneMark, RefSeq
and InterProScan is shown in Figure ~\ref{fig:basic-agree}. It is
important to note that the position of the Pfam match in IGV does not
indicate the true position of the Pfam match in the gene, but provides
an indication of the presence of Pfam matches. Pfam matches are offset
from the start of the gene based on the start and end position of the
Pfam match in the protein sequence. For example, if a Pfam match has a
start position 10 amino acids into the protein sequence, the
corresponding start position in the resulting GFF is 10bp downstream
from the start of the gene.

\begin{figure}
  \centering
  \includegraphics[width=0.90\textwidth]{figures/interproscan-barplot.pdf}
  \caption[Percentage of proteins with Pfam matches]{Bar plot showing the percentage of predicted proteins with Pfam matches from InterProScan for each gene finder across all \textit{Trichoderma} genome assemblies. For DC1 and Tsth20, RefSeq annotations are not available, so their values are set to 0.}\label{fig:ips-counts}
\end{figure}

\begin{table}[h!]
  \centering
  \begin{tabular}{|c|c|c|c|}
    \hline
    Assembly & Braker2 (\%) & GeneMark (\%) & RefSeq (\%) \\ \hline
    DC1 & 73.73 & 74.12 & N/A \\ \hline
    Tsth20 & 73.26 & 74.10 & N/A \\ \hline
    \textit{T. reesei} & 72.38 & 76.01 & 76.44 \\ \hline
    \textit{T. harzianum} & 73.79 & 74.49 & 66.07 \\ \hline
    \textit{T. virens} & 74.68 & 74.76 & 73.18 \\ \hline
  \end{tabular}
  \caption[InterProScan Pfam Evidence]{Table showing the fractions of predicted
    genes with Pfam annotations from InterProScan as a percent}\label{table:ips-pfam}
\end{table}

% DC1: 73.73\%, 74.12\%, N/A
% Tsth20: 73.26\%, 74.10\%, N/A
% \textit{T. reesei}: 72.38\%, 76.01\%, 76.44\%
% \textit{T. harzianum}: 73.79\%, 74.49\%, 66.07\%
% \textit{T. virens}: 74.68\%, 74.76\%, 73.18\%

%    DC1 & $100\times(\frac{10676}{14479})=73.73\%$ & $100\times(\frac{8416}{11354})=74.12\%$ & N/A \\ \hline
%    Tsth20 & $100\times(\frac{11389}{15546})=73.26\%$ & $100\times(\frac{9168}{12373})=74.10\%$ & N/A \\ \hline
%    \textit{T. reesei} & $100\times(\frac{8471}{11704})=72.38\%$ & $100\times(\frac{6990}{9196})=76.01\%$ & $100\times(\frac{6964}{9111})=76.44\%$ \\ \hline
%    \textit{T. harzianum} & $100\times(\frac{11370}{15408})=73.79\%$ & $100\times(\frac{9061}{12164})=74.49\%$ & $100\times(\frac{9293}{14065})=66.07\%$ \\ \hline
%    \textit{T. virens} & $100\times(\frac{11249}{15062})=74.68\%$ & $100\times(\frac{8871}{11866})=74.76\%$ & $100\times(\frac{9062}{12383})=73.18\%$ \\ \hline

\begin{figure}[h!]
  \centering
  \includegraphics[width=\textwidth]{figures/igv/ips-basic-agree.png}
  \caption[Agreeing Pfam matches]{An IGV capture showing complete
    agreement between gene finders for both gene model and protein
    Pfam hits. The longer segments are predicted genes and the smaller
    segments are annotated Pfam matches, all with agreeing start and
    stop positions. The description in each sub-window indicates the
    agreeing Pfam annotations.}\label{fig:basic-agree}
\end{figure}

While cases of complete agreement are abundant, cases of disagreement
also exist and in strange forms. In many cases, while the gene finders
agree on the presence of a gene, only the RefSeq protein product
contains a match to the Pfam database. Why this may be the case is
unclear, and may warrant further investigation. There are also many
cases in which InterProScan reports the same Pfam hits for individual
proteins, but the gene models in the region do not agree. There are
even cases such as the region shown in Figure
~\ref{fig:agree-bizarre2}, where Braker2 and GeneMark agree that two
genes and their associated proteins and Pfam matches are separate, but
RefSeq only reports one gene with multiple Pfam hits. There are also
several cases where two tools are in agreement with proteins
containing Pfam hits while another is not. Even more interesting are
cases such as the one shown in Figure~\ref{fig:ips-no-refseq}, in
which Braker2 and GeneMark predictions contain Pfam matches while
RefSeq does not report a gene at all. This may be the due to
experimental data used in the RefSeq training process or a result of
curation. Regardless of the tool, these cases demonstrate well that
gene finders are not always in agreement even on well known proteins,
to the point that predictions are not present even though protein
products from other gene finders contain known Pfam matches.

\begin{figure}[h!]
  \centering
  \includegraphics[width=\textwidth]{figures/igv/ips-model-disagree2.png}
  \caption[Split Pfam matches]{An IGV capture showing Braker2 and
    GeneMark reporting two genes and their resulting proteins and
    Pfam hits as separate, while RefSeq reports one gene, one protein
    and three Pfam matches.}\label{fig:agree-bizarre2}
\end{figure}

\begin{figure}
  \centering
  \includegraphics[width=\textwidth]{figures/igv/ips-braker-genemark-norefseq.png}
  \caption[RefSeq absence with IPS evidence]{An IGV capture showing a
    scenario where GeneMark and Braker2 agree on a gene model with
    supporting Pfam evidence and RefSeq does not report any gene.}\label{fig:ips-no-refseq}
\end{figure}

In summary, the protein products from Braker2 and GeneMark predictions
do contain matches to the Pfam database. The proportions of matches to
total proteins are similar to that of the RefSeq annotation, sitting
between 65 and 75 percent. While Pfam hits to Braker2, GeneMark and
RefSeq proteins generally agree, we observe regions in which there
is disagreement in several forms.

\section{BLAST Results}
Results from the T-BLAST-N runs are presented in table
\ref{table:tblastn}. Initial BLAST results appear promising for both
the \textit{T. atroviride} and \textit{Fusarium} datasets. All
assemblies considered contain at minimum 89\% of the reference protein
sequences in the case of \textit{T. atroviride} and a minimum of 75\%
in the case of \textit{Fusarium}. Following the trend of gene calls,
\textit{T. virens} returns the fewest hits of the selected assemblies
in all cases while the other assemblies report a similar number of
hits as each other. In the case of \textit{S. cerevisiae}, a minimum
of 57\% of reference proteins matched. These results provide rough
validation that the assemblies contain potential for protein coding
sequences. Successful hits from this process will be retained and used
in the region identification process as validation for gene calls from
selected tools.


\begin{table}
  \centering
  \begin{tabular}{|c|c|c|c|c|c|c|}
    \hline
    Reference & Ref. Proteins & DC1 & Tsth20 & \textit{T. reesei} & \textit{T. harzianum} & \textit{T. virens}  \\ \hline
    \textit{T. atroviride} & 11807 & 11552 & 11080 & 10601 & 11081 & 11078 \\ \hline 
    \textit{Fusarium} & 13312 & 10327 & 10429 & 10064 & 10434 & 10490 \\ \hline
    \textit{S. cerevisiae} & 6014 & 3537 & 3517 & 3445 & 3509 & 3500 \\ \hline
  \end{tabular}
  \caption{tBLASTn hits from reference protein sequences to selected
    assemblies of intereset. Hits are reported if the alignment length
    is greater than 30\% of the reference protein length and if 30\%
    of the aligned length have identical matches.}
  \label{table:tblastn}
\end{table}

\section{Genes in Regions of Anomalous GC Content}\label{section:gc-regions}

To better understand the characteristics of AT-rich sequences in these assemblies, the same sliding window approach described in Section~\ref{section:assemblies} was used to identify segments of genomic sequence with greater than 72\% AT nucleotide content (less than 28\% GC content). Table~\ref{table:gc-content} shows the total length of AT-rich sequence identified in each assembly along with the number of segments identified and the proportion of the assembly comprised of AT-rich sequence. From these results, we see that DC1 and \textit{T. harzianum} have the largest proportion of AT-rich sequence, while Tsth20 and \textit{T. reesei} have moderate amounts of AT-rich sequence. \textit{T. virens} has very little AT-rich sequence compared to the other assemblies.

\begin{table}
  \begin{center}
    \begin{tabular}{|c|c|c|c|c|}
      \hline
      Genome & Total Length (bp) & Low GC Segments & Low GC Length (bp) & Proportion (\%) \\ \hline
      DC1 & 38,616,239 & 610 & 2,064,202 & 5.35 \\ \hline
      Tsth20 & 41,588,851 & 245 & 891,216 & 2.14 \\ \hline
      \textit{T. reesei} & 33,395,713 & 689 & 1,269,347 & 3.80 \\ \hline
      \textit{T. harzianum} & 40,980,648 & 1,311 & 2,527,773 & 6.17 \\ \hline
      \textit{T. virens} & 39,022,666 & 96 & 162,356 & 0.42 \\ \hline
    \end{tabular}
  \end{center}
  \caption[AT-rich sequence content in Trichoderma assemblies]{AT-rich sequence content (windows with $>$ 72\% AT nucleotides) in each \textit{Trichoderma} genome assembly.}\label{table:gc-content}
\end{table}

Figure~\ref{fig:gc-regions} and Table~\ref{table:gc-regions} shows the
results of applying the same region finding process as described in Section~\ref{section:region-met} to
segments of the genome identified as AT-rich. We see that there
are few regions with gene predictions in AT-rich genomic
segments. In DC1 and Tsth20, there are very few regions with full support
from both Braker2 and GeneMark, but more singletons. Again, as in
Section~\ref{section:regions}, there are no regions with partial
support as only two gene finding tools were applied to those
assemblies. \textit{T. reesei} and \textit{T. harzianum} report more
regions in AT-rich genomic sequence than the other assemblies, but
with the majority of regions belonging to the singleton
category. \textit{T. virens} is an interesting case, reporting a
similar number of regions and genes in AT-rich genomic sequence as DC1
and Tsth20. The likely reason for this can be seen in Table~\ref{table:gc-content}, where \textit{T. virens} is shown to have very little AT-rich genomic sequence compared to the other assemblies, which may lead to fewer opportunities for singleton predictions. \textit{T. virens} is also the only assembly to report zero singleton gene predictions.

\begin{figure}
  \centering
  \includegraphics[width=0.90\textwidth]{figures/atrich-regions-barplot.pdf}
  \caption[Regions of agreement in AT-rich genomic sequence]{Regions of full, partial, and no agreement in AT-rich genomic sequence. It is important to note that in the cases of DC1 and Tsth20, full support indicates supporting gene predictions from both GeneMark and Braker2 and as such, there are no regions with partial support.}\label{fig:gc-regions}
\end{figure}

\begin{table}
  \begin{center}
    \begin{tabular}{|c|c|c|c|c|}
      \hline
      Assembly & Full Support & Partial Support & Singletons & Total Genes \\ \hline
      DC1 & 11 & N/A & 20 & 42  \\ \hline
      Tsth20 & 2 & N/A & 9 & 13  \\ \hline
      \textit{T. reesei} & 25 & 18 & 54 & 194  \\ \hline
      \textit{T. harzianum} & 26 & 43 & 68 & 265  \\ \hline
      \textit{T. virens} & 8 & 11 & 0 & 49  \\ \hline
    \end{tabular}
  \end{center}
  \caption[Agreement of gene predictions in AT-rich regions]{Regions of full and partial agreement as well as singleton regions in AT-rich genomic sequence. It is important to note that in the cases of DC1 and Tsth20, full support indicates supporting gene predictions from both GeneMark and Braker2. Column five shows the total number of genes from all gene finders in AT-rich regions for each assembly.}\label{table:gc-regions}
\end{table}

In addition to a breakdown of regions in AT-rich genomic sequence,
understanding which gene finders predict more or fewer genes in these
regions may be of interest. It is possible that an HMM, trained on
genomic sequence with varying nucleotide content, may predict genes
differently to an HMM trained only on sequences with uniformly
distributed nucleotide composition as the variation in nucleotide
composition may affect the various states and relationships between
them in an HMM. Figure~\ref{fig:gc-gene-counts} and Table~\ref{table:gc-gene-counts} shows the number of
genes predicted by each gene finding tool in regions of AT-rich
genomic sequence. GeneMark appears to predict the fewest genes in
AT-rich regions, while RefSeq appears to predict the most. Braker2
lies somewhere in the middle. Again, \textit{T. virens} appears as an
odd case, with very few predictions from all gene finders, which is likely a result of the very small amount of AT-rich genomic sequence present in that assembly.

\begin{figure}
  \centering
  \includegraphics[width=0.90\textwidth]{figures/atrich-genes-barplot.pdf}
  \caption[Gene counts in AT-rich regions]{Number of genes predicted by each gene finder in AT-rich genomic sequence.}\label{fig:gc-gene-counts}
\end{figure}

\begin{table}
  \begin{center}
    \begin{tabular}{|c|c|c|c|}
      \hline
      Assembly & Braker2 & GeneMark & RefSeq \\ \hline
      DC1 & 31 & 11 & N/A \\ \hline
      Tsth20 & 11 & 2 & N/A \\ \hline
      \textit{T. reesei} & 39 & 48 & 107 \\ \hline
      \textit{T. harzianum} & 81 & 30 & 154 \\ \hline
      \textit{T.virens} & 21 & 8 & 20 \\ \hline
    \end{tabular}
  \end{center}
  \caption[Number of genes predicted in AT-rich regions]{Number of genes predicted by Braker2, GeneMark and RefSeq
    in AT-rich genomic sequence from each assembly.}\label{table:gc-gene-counts}
\end{table}

Finally, to test the probability of any given gene prediction falling
in an AT-rich genomic sequence, a two-sided binomial test was
performed to determine if the number of genes predicted in AT-rich
sequences is proportional to the fraction of genomic sequence they
comprise. The null hypothesis in this case is that the gene finding tools predict the same proportion of genes in AT-rich genomic sequences as they do in typical genomic sequence. The results of the test are shown in Table~\ref{table:gc-binomial}. In all cases, it appears that the null hypothesis is rejected, meaning that the gene finders do not predict genes in AT-rich genomic sequence at the same rate as they do in typical genomic sequence. 

\begin{table}
  \begin{center}
    \begin{tabular}{|c|c|c|c|c|c|}
      \hline
      Tool & DC1 & Tsth20 & \textit{T. reesei} & \textit{T. harzianum} & \textit{T. virens} \\ \hline
      Braker2 & $9.56*10^{-181}$ & $1.14*10^{-259}$ & $2.68*10^{-96}$ & $4.05*10^{-140}$ & $1.35*10^{-35}$ \\ \hline
      GeneMark & $5.12*10^{-216}$ & $0.0$ & $5.66*10^{-49}$ & $5.37*10^{-219}$ & $5.31*10^{-35}$ \\ \hline
      RefSeq & N/A & N/A & $1.29*10^{-49}$ & $2.44*10^{-205}$ & $7.40*10^{-33}$ \\ \hline
    \end{tabular}
  \end{center}
  \caption[Binomial test results]{\textit{p}-values produced from a two-sided binomial test
    for each combination of tool and assembly.}\label{table:gc-binomial}
\end{table}

In summary, very few regions with gene predictions are present in
AT-rich genomic sequence. Additionally, genes predicted in these
regions tend to be isolated and not supported by other gene finders,
although some agreement is observed. In terms of number of genes
predicted, RefSeq tends to predict the most genes in these AT-rich
regions while GeneMark predicts the fewest. It was also observed that the proportion of genes predicted in AT-rich genomic sequence is not proportional to the amount of AT-rich genomic sequence present in the assemblies. In addition, the number of genes predicted in AT-rich genomic sequence varies among the different assemblies, even when the proportions of AT-rich sequence content are similar, as in the case of DC1 and \textit{T. harzianum}, although only two gene finding tools were applied to DC1. These observations suggest that gene prediction in AT-rich genomic sequence is a challenging task, and further research is required to better understand the factors influencing gene prediction in these regions.

%\section{Selection of Gene Finding Tool}

With all of these results, it makes sense to explore the question of
which gene finding tool one should choose for optimal gene prediction
performance. Comparisons drawn in this section are made in the context
of \textit{Trichoderma} assemblies and may not extend to other
datasets. Results from this work are summarized in table
\ref{table:final-score}. Ignoring availability and use of the gene
finding tools, it would appear that RefSeq performs the best in the
remaining categories, earning top marks in every category except in
it's ability to predict very short genes. Braker2 earns second place;
however, this does not capture Braker2's failure in predicting
accurate numbers of genes in DC1, Tsth20, \textit{T. harzianum} and
\textit{T. virens}. GeneMark comes in last, excelling only in number
of genes predicted and Pfam support for the genes that it predicts.

Relating these observations to use-case scenarios, in the case that
your organism of interest has a RefSeq annotation associated with it,
the RefSeq gene prediction process appears to produce the best set of
predictions. If users also have experimental evidence, such as RNAseq
data under experimental conditions, it may be worthwhile training a
Braker2 model and predicting genes with Braker2 to supplement the
already well performing RefSeq gene predictions. If the organism of
interest is not a RefSeq individual but training data is available for
that organism, Braker2 is the next best option, although it is
important to note that the application of a trained Braker2 prediction
model to an organism from which the training data did not originate is
not advised based on the results presented in this work (see
\ref{section:gene-finding}). While it is true that the
\textit{T. reesei} genome differs from other \textit{Trichoderma}
genomes, it's status as a representative RefSeq organism makes it
somewhat of a gold standard. In this case, applying a gene model
trained using evidence from the gold standard produces biased numbers
of genes predicted in other \textit{Trichoderma} genomes. While
Braker2 technically scores the second highest, users must be very
careful when selecting training data, and ensure that the training
data either comes from the organism of interest, or comes from a very
closely related organism with a highly similar genome. In the case
that no appropriate training data is available, GeneMark is still an
option, and users can be confident that the tool predicts a reasonably
accurate number of genes with supporting Pfam matches. It is also
important to note that GeneMark does not perform as well in AT-rich
regions as Braker2 and RefSeq, does not predict isoforms, and
systematically fails to predict some BUSCO orthologs.

When availability of a gene finding tool becomes a concern and RefSeq
is not considered, the scores drop significantly for Braker2 and
GeneMark as seen in the final row of table \ref{table:final-score}. In
this situation, Braker2 still outperforms GeneMark. If the organism of
interest is not considered a representative RefSeq individual, but
supporting evidence specific to that organism or a very closely
related organism is available, a trained Braker2 prediction model will
perform well. Again, in the case that the organism is not a RefSeq
individual and no appropriate training data is available, GeneMark is
still a reasonable option even with the previously identified caveats.

\begin{table}
  \centering
  \begin{tabular}{|c|c|c|c|}
    \hline
    Category & Braker2 & GeneMark & RefSeq \\ \hline
    Availability & 3 & 3 & 0 \\ \hline
    Ease of install & 1 & 2 & 0 \\ \hline
    Ease of use & 3 & 3 & 0 \\ \hdashline
    \makecell{\# of genes\\predicted} & 0 & 3 & 3 \\ \hline
    \makecell{\# of transcripts\\predicted} & 3 & 0 & 2 \\ \hline
    \makecell{Predicts shortest\\genes} & 2 & 1 & 0 \\ \hline
    \makecell{Predicts more\\shorter genes} & 1 & 0 & 3 \\ \hline
    BUSCO Performance & 2 & 1 & 3 \\ \hline
    \makecell{Performance in\\AT-rich sequence} & 2 & 1 & 3 \\ \hline
    \makecell{Predictions with \\InterProScan support} & 3 & 3 & 3 \\ \hline
    \makecell{Final Score\\(Publicly Available)} & 20 & 17 & N/A \\ \hline
    \makecell{Final Score\\(Ignoring Availability)} & 13 & 9 & 17 \\ \hline
  \end{tabular}
  \caption[Final scoring table]{Table with scores attirbuted to
    performance of each gene finder in several categories. The score
    definitions for performance are as follows: 0 - fail, 1 - pass, 2
    - good, 3-excellent. Since RefSeq is not publicly available, it is
    marked as N/A in the publicly available final scores. The dashed
    line separates categories associated with availability and use
    from categories describing gene prediction performance.}
  \label{table:final-score}
\end{table}

In summary, these results indicate that if your organism is a RefSeq
organism, use the RefSeq annotation. If no RefSeq predictions are
available but appropriate training data is, one should use Braker2. If
the training data is of questionable similarity or not available at
all, users can fall back on \textit{ab initio} gene finders such as
GeneMark, which while not ideal, still predict genes with supporting
evidence.


\chapter{Conclusions}

\section{Selection of Gene Finding Tool}\label{chapter:conclusion}

With all of these results, it makes sense to explore the question of
which gene finding tool one should choose for optimal gene prediction
performance in a given application. Comparisons drawn in this section are made in the context
of \textit{Trichoderma} assemblies and may not extend to other
datasets. Observations from the results portion of this work have been
converted to a ranking for each gene finder, relative to the other gene finders. Values for each category are ranked from 0 to 3, with 0 indicating that a feature was unavailable or not applicable, and ranks 1 through 3 when there are three gene finders being considered, but 1 or 2 whether there are two gene finders being considered, indicating performance relative to the other gene finder(s). Ties are allowed if the performance of two gene finders are similar. Categories considered include availability of the gene finding tool, ease of installation, ease of use, number of genes predicted, number of isoforms predicted, BUSCO performance, performance in AT-rich sequence, and predictions with InterProScan support.
Results for these categories are summarized in
Table~\ref{table:final-score}. We note that not all results are included in this table, most notably the region analysis in Section~\ref{section:regions}, as more exploration of those results is needed. The assembly results are also not included, as they are not reflective of the characteristics of the gene finders themselves.
In both the publicly available and non-publicly available scenarios, Braker2 comes out on top, with RefSeq coming in second when availability is not considered. While Braker2 does perform the best, it is important to note that its performance is highly dependent on the quality and relevance of the training data provided. RefSeq performs well across most categories, particularly when considering number of genes predicted, isoform prediction, and predictions in AT-rich sequence, but if the user's organism is not closely represented in RefSeq, it is not an option.
GeneMark comes in last, performing well in terms 
of genes predicted, BUSCO coverage, and Pfam support for the genes that it predicts. We do note that these criteria are not fully objective; however they do provide a useful framework for comparing the gene finders, and are based on observations presented in this work. 

Relating these observations to use-case scenarios requires consideration of the data available to the user. In the case that a user has access to RefSeq annotation data for their organism, the gene predictions will likely be of good quality, although if new training data is available for the organism, gene predictions from a trained Braker2 model will supplement the RefSeq predictions. We note that users should always investigate the quality of the RefSeq assemblies, as some assemblies may be of low quality despite their status as a RefSeq genome assembly, which could affect gene finding performance. In the case that the organism of interest is not a RefSeq organism and no RefSeq annotation is available, Braker2 is the best option, provided that relevant training data can be found. If no training data is available, GeneMark is a reasonable choice, and although its performance is not as good as the other gene finders, it still predicts a useful set of genes. 

\begin{table}
  \centering
  \begin{tabular}{|c|c|c|c|}
    \hline
    Category & Braker2 & GeneMark & RefSeq \\ \hline
    Availability & 2 & 2 & 0 \\ \hline
    Ease of install & 1 & 2 & 0 \\ \hline
    Ease of use & 2 & 2 & 0 \\ \hdashline
    Number of genes predicted & 1 & 3 & 3 \\ \hline
    Number of isoforms predicted & 3 & 0 & 2 \\ \hline
    BUSCO Performance & 3 & 2 & 1 \\ \hline
    Performance in AT-rich sequence & 2 & 1 & 3 \\ \hline
    Predictions with InterProScan support & 2 & 3 & 1 \\ \hline
    Cumulative Rank (Considering Availability) & 16 & 15 & N/A \\ \hline
    Cumulative Rank (Ignoring Availability) & 11 & 9 & 10 \\ \hline
  \end{tabular}
  \caption[Final scoring table]{Table with ranks attributed to
    performance of each gene finder in several categories. The ranks
    for performance are 0 being not applicable, and 1 through 3 being increasing levels of performance. Since RefSeq is not publicly available, it is marked as N/A in the cumulative rank when considering availability. The dashed line separates categories associated with operation and use
    from categories describing gene prediction performance.}\label{table:final-score}
\end{table}


\section{Future Work}

While this work is extensive, there are still many areas that could be
expanded on. The sheer number of gene finding tools available means
that many more could be compared to the ones presented here. In
addition, supplying more training data to Braker2 may improve gene
finding performance, and training on RNAseq from different organisms
could be used to improve the performance of Braker2. With more data in mind, different types of training data --- such as RNAseq,
expressed sequence tags (ESTs), and protein sequences --- could also be
used as supporting evidence for training Braker2. Another
limitation of this work is that the gene finders were run with default
parameters, and it is possible that tuning the parameters of the gene
finders could improve performance. Exploration of newer gene finding models, such as those based on deep learning techniques, could also be
performed to see how they compare to the more established gene finding
tools.

Following the gene prediction process, there are a number of different
analyses that could be performed on the predicted genes. For example,
functional annotation of the predicted genes could be performed using
tools such as BLAST2GO to assign Gene Ontology (GO) terms and KEGG
pathways to the predicted genes. This could provide further insight
into the functional roles of the predicted genes and their potential
applications. More importantly in the context of this work, further analysis
of predicted genes with the antiSMASH tool could be used to
identify biosynthetic gene clusters (BGCs) from the predicted genes.

Expanding the number of genomes used in this work would also be
beneficial, as the results presented here are based on a limited
number of genomes. Including more genomes from different species and
strains of \textit{Trichoderma} could provide a more comprehensive
understanding of gene finding performance across the
genus. Additionally, comparing the performance of gene finders on
other fungal genomes could provide insights into whether the findings
are generalizable. Examining the performance of gene finders on
different assemblies of the same genome could also be useful, as
different assemblies may have different levels of completeness and
accuracy. This could help to identify tangible benefits of using one
assembly over another, and could also provide insights into the
limitations of gene finders when applied to different assemblies.

We also encourage the developers of gene finding tools to consider expanding the selection of genomes used in their testing and training datasets to include a more diverse set of fungal genomes. This could help to improve the performance of gene finders on fungal genomes, and may improve performance of gene finder in AT-rich regions, which are common in fungal genomes. In addition, we encourage developers to continue to improve the documentation and the usability of their tools, as ease of installation and use are important factors for many users. 

Finally, the results of this work could be used to inform future
research in the field of fungal genomics, particularly in the case of
the DC1 and Tsth20 genome assemblies. The results of these genome
assemblies are high-quality, and contain near-chromosomal scale
sequences. The predicted genes from these genomes are a rich resource
for future research, and could be used to identify novel genes and
gene families in DC1 and Tsth20. Further understanding of the
mechanisms behind salt and drought tolerance as well as degradation of
hydrocarbon in suboptimal environments could be achieved by studying
the predicted genes in these genomes. 


%\chapter{Discussion}
%\input{chapters/c6/assembly-discussion}
%\section{Initial Gene Counts}
Initial gene counts from Braker2 and Genemark show a similar trend in
all assemblies included in this analysis. Braker tends to predict
fewer genes in comparison to GeneMark and RefSeq results except in the
case of \textit{T. reesei}. The cause of this difference is likely
two-fold in nature. Most importantly, Braker2 was trained on an RNAseq
dataset derived from \textit{T. reesei}. This additional information
in the gene finding model is likely why Braker2 finds more genes and
transcripts than GeneMark. The effects of this training set may also
be why Braker2 tends to predict fewer genes and transcripts in other
assemblies when compared to results from GeneMark and RefSeq. The
nature of gene models in \textit{T. reesei} likely differs than those
of other genomes, resulting in 'poorer' relative performance than in
\textit{T. reesei}. This highlights the fact that users should choose
training sets carefully when planning to use a hybrid gene finding
approach. Choosing a dataset that is inappropriate will end with
results that are skewed or biased towards the training set of
interest, although they will likely be of higher confidence. This is
an important trade-off to consider when preparing for a hybrid gene
finding approach.

Another potential explanation of Braker2's low gene count trend could
again be due to the reference training set used in training. However
in this case, not necessarily due to the RNAseq data itself, but the
nature and size of the genome from whih the RNAseq data was gathered
from. The \textit{T. reesei} genome is significantly smaller than
other assemblies considered as shown in figure (blah), resulting in
less physical space for coding sequences to be found. This would
explain why GeneMark predicts a similar number of genes as Braker2 in
\textit{T. reesei} but not in any of the other assemblies. In theory,
it would make sense that after training, Braker2 would predict a
fraction of all possible genes in a larger assembly, since it was
trained with RNAseq data from a genome that is a fraction of the
size. Again, this shows that the choice of training data is paramount
when planning a hybrid assembly approach to gene finding.



%\section{BUSCO Analysis}

The results from BUSCO analysis of the gene sets produced by each
prediction method prove promising. With all gene sets being 99.2\%
complete or higher based on the fungal dataset provided to BUSCO. This
higher number indicates that these gene finding tools capture nearly
completely the set of evolutionarily conserved single-copy orthologs
pre-defined by BUSCO curators. In the case of the fungal dataset,
there are 758 genes considered during analysis.

The most glaring observation from this analysis is the duplication
level found in the gene sets produced by Braker2 in comparison to the
GeneMark and RefSeq datasets. The Braker gene sets show a single-copy
match for roughly 80-85\% of the total gene call set with duplicates
making up the other 15-20\% of the set. The likliest reason for this
difference is the presence of isoforms in gene sets produced by
Braker. As shown in figure \ref{fig:genecounts}, Braker2 does produce
isoforms in its output, which would be identified as duplicates in
the BUSCO process. However, the number of isoforms predicted per gene
does not make up 20\% of the total set of gene calls. It is possible
that the BUSCO dataset contains a large fraction of the Braker2 gene
calls with isoforms, but that is unlikely and warrants further
investigation into the genes matching the BUSCO dataset.  Another
possible explanation is that Braker2 is predicting genes that are
actually isoforms as separate genes. BUSCO also makes note of isoforms
of a gene being the cause of high number of duplicates, and recommends
users remove isoforms prior to running BUSCO, although that step was
not performed for this work.






\uofsbibliography[unsrt]{./refs/full-bib}

%\chapter{Supplemental Information}
\section{Platform and Software Installation}
(Possibly supporting materials or discussion)

\subsection{Platform}
All analysis was performed on the RSMI server hosted on Copercius at
the University of Saskatchewan. This server is equipped with 64 cores
in addition to 1.5 TB of memory. The server is running RedHat
Enterprise Linux 7 as of writing this thesis. All data is stored
either on datastore, or in the RSMI scratch space.

\subsection{NextDenovo and NextPolish Installation}
Installation of nextDenovo was straightforward. Simply download the
compressed tar file from their website and unpack it. NextDenovo
requires Python versions 2 and 3 along with a package called parallel
to aid in parallel processing of datasets. The parallel package was
installed using pip in the bioinformatics conda environment in the
scratch space of Copernicus. NextPolish was installed in a Python
environment by a member of the research computing team that manages of
our system. Assistance was required for this as the version of RHEL
used by the server introduces glibc version conflicts with Anaconda
when trying to install nextPolish. 

\subsection{RepeatMasker Installation}

The installation procedure was somewhat indepth, requiring
RepeatMasker configuration, which itself requires downloading an
appropriate repeat database (Dfam in this case, included with
RepeatMasker), installation of Tandem Repeat Finder (TRFM) and
installation of a sequence search tool, for which I chose HMMER from
the list of potential tools as we were generally familiar with its
use. The path to the installation of TRFM is required during
configuration along with the search tool of choice, a simple selection
of 4 tools that will have an autocompleted path in this case, since
HMMER is installed via anaconda.

\subsection{GeneMark-ES Installation}
GeneMark-ES was successfully installed by downloading and unpacking
the package from their website along with a key required for use.

\subsection{Braker2 Installation}
Braker2 was also successfully installed by a member of the research
computing team who has set up several modules including an
initialization script to get things up and running as well as create a
reloadable environment for use again in the future. Once the
environment has been loaded, one must load the Hisat2 module from
Compute Canada as well as a htslib module (more detail to come). Once
all modules are loaded, there are a few environemnt variables that
need to be set, those being AUGUSTUS\_CONFIG\_PATH and
TSERBA\_CONFIG(?figure this out). In addition, a software package named
TSEBRA from the same developers as Braker2 must be installed for
consolidating gene calls. The variables can be set within the
braker2.pl command, which have higher priority over environment
variables and probably makes things easier to track.



\end{document}
