\documentclass[12pt]{article}
\usepackage[noblocks]{authblk}

\title{Project Proposal: Comparative analysis of Gene Finding tools                                                                         
  when applied to \textit{Trichoderma} genomes} \author{Connor
  Burbridge} \affil{USask NSID: cbe453 \\ USask ID no.\ 11162928
  \\ Supervisors: Dave Schneider \& Tony Kusalik\\}

\begin{document}
\parindent=14pt
\maketitle

\clearpage
\tableofcontents
\clearpage

\section{Pre-work}

\subsection{Assembly}

The foundation of this project is base ond the sequencing of two novel
\textit{Trichoderma} strains identified in prairie regions of Canada
(Alberta and Saskatchewan). To assemble these genomes, a hybrid
assembly process was used, following default assembly parameters with
MASuRcA, which utilizes the Flye assembler if both Nanopore and
Illumina data are used as inputs, which are the inputs in this
case. The next paragraph describes the process of working with MASuRcA.

MASuRcA 4.0.3 was run using the Compute Canada software stack
available on Copernicus.  Prior to loading the MASuRcA environment,
the GCC/9.0 and StdEnv/2020 modules must be loaded first. This version
of the software is not ideal, but the Anaconda installation of versino
4.0.9 consistently failed, even in a fresh environment.  Building the
software from scratch is a potential option. In addition to this
difficulty, the assemblies were performed in the p2irc_rsmi scratch
space on Copernicus as I was encountering permissions issues when
trying to run the assembly in the Roots datastore.  I don't know
exactly why (microsoft permissions problems from datastore?, but there
were permission issues associated with scripts being copied to
datastore as part of the assembly process. All assembly materials were
copied back to datastore after assembly.

Initially, a configuration file must be generated to run MASuRcA with
the optimal combination of assembly tools for the data supplied (Flye
+ polishing). Running MASuRcA from the commandline tool utilizes the
CABOG celera-based assembler, which is noted as being slower and
results in an assembly with similar or worse quality than one using
Flye.

To generate the config file, run the following:
masurca -g config.txt

The config file was then altered to provide input file, options and
allowable number of threads for assembly. All other assembly
parameters were left untouched. Insert lengths for the Illumina read
data used the recommended values (stated to work for most Illumina
reads), although these could be modified with input from Brendan. The
config files for both assemblies are available in the assebly
directories.

To generate the assembly.sh script, run:
masurca config.txt

Once the assembly.sh script is generated, execute the assembly using:
./assemble.sh

Final assemblies are placed in directories with the prefix flye.mr.*
Quast analysis of the genomes was also performed, with the output
being placed in directories named quast within the assembly directory.

To run Quast:
quast -o ./ -t 16 assembly-file.fasta

\subsection{Repeat Masking}

In order to evaluate the performance of gene finding tools in
repetitive or low complexity regions in the context of
\textit{Trichoderma} genomes, we must first identify said regions in
the genomes considered. To do this, RepeaatMasker has been selected as
a tool to identify repeat regions based on a fungal subset of the Dfam
database by specifying the fungo species tag to RepeatMasker when
running the program. The program was configured with options to
produce several output formats for each genome considered, which will
allow for more informative downstream analysis of results. All
commands for repeat masking are located withing the processing
directory for each strain/genome.

The Installation procedure was somewhat indepth, requiring
RepeatMasker configuration, which itself requires downloading an
appropriate repeat database (Dfam in this case, included with
RepeatMasker), installation of Tandem Repeat Finder (TRFM) and
installation of a sequence search tool, for which I chose HMMER from
the list of potential tools as I am generally familiar with its use.

General command for running RepeatMasker:
/datastore/Roots/Connor/masters/software/repeatmasker/RepeatMasker/RepeatMasker
-pa 10 -a -small -species fungi -html -gff -dir ./
path-to-genome/genome.fasta


\end{document}
