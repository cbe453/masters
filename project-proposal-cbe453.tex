\documentclass[12pt]{article}
\usepackage[noblocks]{authblk}

\title{Project Proposal: Comparative analysis of Gene Finding tools
  when applied to \textit{Trichoderma} genomes} \author{Connor
  Burbridge} \affil{USask NSID: cbe453 \\ USask ID no.\ 11162928
  \\ Supervisors: Dave Schneider \& Tony Kusalik\\}

\begin{document}
\parindent=14pt
\maketitle

\clearpage
\tableofcontents
\clearpage

\section{Background}

\subsection{Trichoderma}

Crop resistance to environmental stressors is a necessity for crop
health and overall crop yields. Current popular methods for crop
protection involve the use of pesticides and genetically modified
organisms, which can be expensive and potentially politically dividing
in the case of GMOs\cite{GMO}. In addition, crops suffer when soils
are not sufficient for crop growth and health. Soil insufficiencies
can result in drought stress as well as nutrient stress, leading to
poor overall yields.

\textit{Trichoderma} is a type of fungi that can colonize the roots of
plants in a non-toxic, non-lethal, opportunistic symbiotic
relationship\cite{Trichoderma}. Many strains of \textit{Trichoderma}
have been shown to provide resistance to bacteria and other fungi in
soils through the use of polyketides, non-ribosomal peptide
synthetases and other antibiotic
products\cite{Trichoderma}\cite{Secretome}. Recently, two strains of
\textit{Trichoderma} have been identified in the prairie regions of
Alberta and Saskatchewan. These two strains, named Tsth20 and DC1,
have been found to have beneficial properties when used as an
inoculant for plants in the soils mentioned before. In addition to
these beneficial properties, the two strains mentioned previously
provide even further protection for plants in dry, salty soils and one
strain also has potential for use as a bioremediation tool in soils
contaminated with hydrocarbon content. Bioremediation and resistance
to drought tolerance has also been investigated in other strains of
\textit{Trichoderma} as well\cite{Drought}\cite{Kaminskyj}. However,
little is known about the mechanisms at work in these strains, so DC1
and Tsth20 were sequenced by the Global Institute for Food Security
(no publication yet) in an initial attempt to better understand the
details of these genomes and secretomes.

\subsection{Genome Assembly}

Sequence assembly has been a long-standing application problem in the
field of bioinformatics\cite{assembly}. Determining the correct order
and combination of smaller subsequences into an accurate sequence
assembly is also computationally difficult in terms of compute
resources such as memory, CPU cycles and storage required for input
sequences\cite{assembly}. In addition to these difficulties, there can
be other issues encountered during asssembly due to the nature of the
data or genomes themselves, such as low quality base calls for long
read data or the inherent content of genomes themselves with
repetitive regions. Insufficient data used in an assembly may result
in short, fragmented assemblies, depending on the size of the genomes,
while sequence data that is not long enough can fail to fully capture
repetitive regions in an assembly. To solve this problem, a wide range
of assembly tools have been developed with their own unique approaches
to genome assembly problems, so it is important to use an appropriate
assembler for the task at hand, and important to evaluate the assembly
thoroughly. One approach to aid in the previously mentioned issue of
assembly correctness is to use a combination of long and short reads
in what is known as a hybrid assembly. Combining both highly accurate
short reads with deep coverage along with less accurate but much
longer reads can produce high quality genome assemblies that capture
long repetitive regions. A hybrid assembly approach will likely be
used for the assembly of theses \textit{Trichoderma} genomes. However,
this is subject to the performance of available genome assembly
tools. Genome assembly tools to be considered will include
SPAdes\cite{spades}, MaSuRCA\cite{masurca}, and other potential
assembly tools as well.  Assemblies must also be evaluated with
measures such as N50, L50, coverage, average contig length and total
assembled length to ensure that the genomes are well assembled, at
least based on these metrics\cite{assembly}. Following appropriate
assembly protocols is essential to the further success of a project as
downstream processing such as annotation depends on a high-quality
assembly.

\subsection{Repeat Identification/Masking and Identification of AT-rich Genomic regions}
Repeat identification within assembled genomes is a problem that needs
to be considered during the genome annotation process. Regions with
long repeats can have a significant impact on genome assembly as well
as gene finding due to the limitation of short reads used in some
assemblies\cite{Repeats}. Short reads may be unable to bridge or cover
entire repeat regions within a genome, so it is important to consider
the use of long reads from technologies such as Nanopore or PacBio to
provide a complete picture of these regions when pursuing a new genome
assembly project. It is also possible for repetitive regions to
contain genes as well, making for an interesting investigation in
regards to \textit{Trichoderma}, as fungal genomes have been shown to
contain many repeat regions with a high concentration of A and T
nucleotides\cite{fungalrepeats}. Once these repetitive regions have
been identified, the genome could be masked to exlude these regions in
downstream processing if desired, as these regions may be poorly
assembled and may result in found genes that do not truly exist in
those regions. However, this may not be as common today, as repetetive
regions have been shown to contain genes as well\cite{dontMask}. This
may affect the gene finding process described later and may be an
interesting topic to look into considering the large number of
available gene finding programs.

\subsection{Gene Finding}
Gene finding is another long standing computational problem in
bioinformatics, which concerns itself with identifying potential genes
within genomes based on patterns or evidence considered by the gene
finding program. There are two common methods for gene finding, those
methods being \textit{ab initio} methods, where programs search for
patterns and gene structures, and similarity or evidence-based
searches, which use prior information such as RNAseq data, expressed
sequence tags and protein sequences to identify genes within a new
genome\cite{GeneFinding}. Complicating the process more is the
introduction of introns and alternative splicing in eukaryotes, making
it possible for one gene to have several possible transcripts at the
same locus. Examples of \textit{ab initio} methods include tools such
as GeneMark-ES\cite{GeneMarkES} and GlimmerHMM\cite{Glimmer}, while
evidence based methods include tools such as \\ Braker2\cite{Braker2},
which can incorporate existing data such as RNAseq, protein sequences,
etc. in gene prediction models.

As mentioned in the repeat finding section of this proposal, long
repeats and transposable elements can cause issues for gene finding
programs, making this an interesting area of research, at least for
fungal genomes such as \textit{Trichoderma}. Fungal genomes are also
interesting in the topic of gene finding as there are few programs
targeted directly at gene finding in fungal genomes while organisms
such as human, mouse, and \textit{Arabidopsis} benefit from having
many tools tested on them as they are model organisms in the field, at
least according to table 1 of Wang, Z \textit{et
  al}\cite{GeneFinding}. How these different methods and tools perform
when applied to fungal genomes is an important consideration as fungi
have features that can benefit plant growth as mentioned earlier.

There are also other aspects of gene finding tools that are important
to consider. These include features such as whether or not the gene
finders find non-coding RNAs, annotation of 5' and 3' UTR regions, and
in the case of ab-initio methods, the assumptions made by the
underlying models used for gene finding. These features and others can
influence a user's decision on which gene finding tool to consider and
will complicate comparative analysis of multiple gene finding tools.


\section{Research problem}
Genome annotation, and more specifically gene finding, has been a
popular computational problem for decades. The identification of
possible genes provides other researchers with a valuable resource for
future research avenues. Due to the importance and popularity of this
topic, there have been a variety of gene finding tools developed, each
with its own target applications or use cases. The abundance of gene
finding tools raises questions such as which tool to use? Do different
tools provide drastically different results? Are there differences in
the number and/or of accuracy predicted genes in repetitive genomic
regions between \textit{ab initio} methods and similarity-based
methods?  Comparative studies of gene finding tools have been
performed before, however most of these studies are in reference to
popular eukaryotic subjects of study such as humans, mice and
\textit{Arabidopsis}. Few if any comparative studies have been
performed in fungal genomes, and even fewer in the case of
\textit{Trichoderma}.

This lack of comparative analysis in fungal genomes provides a
research avenue to compare results from the application of different
gene finding tools to new and existing \textit{Trichoderma}
assemblies. Results from using different annotation tools will allow
us to identify a core genome along with genes that are not predicted
by all tools in all genomes, or outliers. This analysis will also
allow us to compare results from both evidence-based gene finding
methods as well as \textit{ab initio} gene finding methods. One
interesting area to investigate between these methods would be in
regions of low nuclotide diversity, or the AT-rich regions common to
\textit{Trichoderma} and other fungi. It is possible that gene finding
programs may have difficulty with these regions of genomes as they may
contain repeat regions and, as mentioned before, have low nucleotide
diversity, making it unlikely but not impossible for genes to be found
in these regions.

As mentioned previously, it is important to understand the features of
these gene finders prior to developing an approach for comparing their
outputs. With this in mind, the tools selected for this research
should cover both \textit{ab initio} and evidence-based methods. The
selected tools should also cover the identification of other features
such as small RNAs and untranslated regions as it is possible that
some gene finders may identify features such as small RNAs without
explicitly stating that they do. Said features may also be useful for
GIFS researchers working with this data in the future. To cover
several of these areas of interest, the following tools are proposed
for this project. Note that other tools are also available for
consideration and this set does not have to be final.

\begin{itemize}

\item RepeatMasker\cite{RepeatMasker}: a tool used to identify
  repetititve regions within a genomic sequence.
\item Infernal/Rfam\cite{Infernal}: a tool used in combination with a database to
  analyze genomic sequences for non-coding RNAs. Useful to compare
  against gene calls from the tools proposed below.
\item GeneMark-ES\cite{GeneMarkES}: an \textit{ab initio} HMM-based
  gene finding tool with a specific option for fungal genomes.
\item GenomeThreader\cite{GenomeThreader}: an evidence-based gene finding
  tool that uses dynamic programming to efficiently identify splice
  candidates for genes using splice-aware RNASeq alignments as input.
\item Braker2: a gene finding tool that uses a hybrid approach
  combining information from \textit{ab initio} and evidence-based
  methods. In addition, Braker2 also has the ability to identify UTR
  regions as well.

\end{itemize}

A comparative approach or pipeline will then be developed to compare
outputs and performance of these gene finding tools. Points of
comparison will include gene finding features (proficiencies and
deficiencies), efficiency of selected tools (runtimes, memory
requirements), requirements of select tools (operating system,
additional databases, required librairies etc.), ease of installation,
called genes as compared to aligned RNA sequence data, comparison of
called genes to previously validated subsets of
genes\cite{Validation}, identification of genes in repetitive regions,
identification of small RNAs, distributions of gene lengths along with
maximum/minimum gene lengths and performance of gene finders in
AT-rich regions. Results from this work will provide insight into the
use of different gene finding tools in the context of
\textit{Trichoderma}. This work will also provide a large set of
called genes spanning at minimum two newly assembled
\textit{Trichoderma} genomes, those being DC1 and Tsth20, as well as
other previously assembled \textit{Trichoderma} genomes from NCBI or
other sources.

\section{Deliverables}
\begin{itemize}
\item High-quality hybrid assemblies of both Tsht20 and DC1.
\item Lists of genes for each \textit{Trichoderma} assembly and gene
  finding tool combination considered
\item A consenus or core gene set shared by all genomes considered
  based on different gene finding tools
\item Repeat regions identified in assembled \textit{Trichoderma}
  genomes.
\item A potential list of true positives based on genes which have
  supporting RNAseq eveidence
\item Final comparative tables including the criteria mentioned above
  along with performance for each gene finder relative to the genome
  considered.
\end{itemize}

\section{Timeline}
\begin{itemize}
\item Initial genome assemblies: Collection of existing assemblies (1 week), Assembly of DC1/Tsth20 (2-4 weeks)
\item Gene finding and annotation results: Identification of gene finding tools (1-2 weeks), Gene finding analysis on selected genomes (1-2 months)
\item Downstream analysis of gene finding results (1-2 months)
\end{itemize}

\begin{thebibliography}{99}
\bibitem{GMO} Zhang, C., Wohlhueter, R., Zhang, H. (2016). Genetically
  modified foods: A critical review of their promise and
  problems. \textit{Food Science and Human Wellness}, \textit{5}(3),
  116-123.

\bibitem{Trichoderma} Hermosa, R., Viterbo, A., Chet, I., Monte,
  E. (2012). Plant-beneficial effects of \textit{Trichoderma} and of
  its genes. \textit{Microbiology}. \textit{38},
  17-25. 

\bibitem{Secretome} Ramirez-Valdespino, C., Casas-Flores, S.,
  Olmedo-Monfil, V. (2019). \textit{Trichoderma} as a Model to Study
  Effector-Like Molecules. \textit{Frontiers in Microbiology}, \textit{15}.

\bibitem{Drought} Repas, T.S., Gillis, D.M., Boubakir, Z., Bao,
  Xioahui., Samuels, G.J., Kaminskyj, S.G.W. (2017). Growing plants on
  oily, nutrient-poor soil using a native symbitoic
  fungus. \textit{PLOS ONE}, \textit{12}(10).

\bibitem{Kaminskyj} Kaminskyj et al. (2012). Method for increasing
  plant growth using the fungus \textit{Trichoderma harzianum} (United
  States PCT/CA2O1O/OO1454). https://patentimages.storage.googleapis.com/57/4f/7f/5ff60c47045f01/US20120178624A1.pdf
  
\bibitem{Braker2} Hoff, K. J., Lomsadze, A., Borodovsky, M., Stanke,
  M. (2019). Whole-Genome Annotation with BRAKER. \textit{Methods Mol
    Biol.}, \textit{1962}, 65-95.

\bibitem{GeneMarkES} Borodovsky, M., Lomsadze, A. (2011). Eukaryotic
  Gene Prodeiction using GeneMark.hmm-E and
  GeneMark-ES. \textit{Current Protocols in
    Bioinformatics}. \textit{4}.

\bibitem{Glimmer} Majoros, W.H., Pertea, M., Salzberg, S.L. (2004) TigrScan and
  GlimmerHMM: two open-source \textit{ab initio} eukaryotic
  gene-finders. \textit{Bioinformatics}. \textit{20}(9), 2878-2879.
  
\bibitem{AUGUSTUS} Stanke, M., Morgenstern, B. (2005). AUGUSTUS: a web
  server for gene prediction in eukaryotes that allows user-defined
  constraints. \textit{Nucleic Acids Research}. \textit{33}, 465-467.
  
\bibitem{InterProScan} Jones, P., Binns, D., Chang, H. Y., Fraser,
  W., Li, W., Hunter, S. (2014). InterProScan 5: genome-scale
  protein function
  classification. \textit{Bioinformatics}. \textit{30}(9),
  1236-1240.

\bibitem{AnnotationSummary} Yandell, M., Ence, D. (2012). A beginner's
  guide to eukaryotic genome annotation. \textit{Nature Reviews
    Genetics}. \textit{13}, 329-342.
  
\bibitem{assembly} Sohn, J., Nam, J. (2016) The present and future of
  \textit{de novo} whole-genome assembly. \textit{Briefings in
    Bioinformatics}. \textit{19}1, 23-40.

\bibitem{spades} Bankevich, A., Nurk, S., Antipov, D., et
  al. (2012). SPAdes: a new genome assembly algorithm and its
  applications to single-cell sequencing. \textit{Joural of
    Computational Biology}. \textit{19}(5), 455-477. 
  
\bibitem{masurca} Zimin, A.V., Marcias, G., Puiu, D., Roberts, M.,
  Salzberg, S.L., Yorke, J.A. (2013). The MaSuRCA genome
  assembler. \textit{Bioinformatics}. \textit{29}(21), 2669-2677.
  
\bibitem{Repeats} Lerat., E. (2009). Identifying repeats and
  transposable elements in sequences genomes: how to find your way
  through the dense forest of programs. \textit{Nature
    Heredity}. \textit{104}, 520-533.

\bibitem{fungalrepeats} Li, WC., Huang, CH., Chen, CL. et
  al. (2017). Trichoderma reesei complete genome sequence,
  repeat-induced point mutation, and partitioning of CAZyme gene
  clusters. \textit{Biotechnol Biofuels}. \textit{10}, 170.

\bibitem{dontMask} Slotkin, R.K. (2018). The case for not masking away
  repetitive DNA. \textit{Mobile DNA}. \textit{9}(1), 15.
  
\bibitem{GeneFinding} Wang, Z., Chen, Yazhu., Li, Y. (2016). A Brief
  Review of Computational Gene Prediction Methods. \textit{Genomics
    Proteomics Bioinformatics}. \textit{2}4, 216-221.

\bibitem{RepeatMasker} Smit, A., Hubley, R., Green, P. (2013-2015)
  RepeatMasker Open-4.0. https://www.repeeatmasker.org

\bibitem{Infernal} Nawrocki, E.P., Eddy, S.R. (2013). Infernal 1.1:
  100-fold faster RNA homology
  searches. \textit{Bioinformatics}. \textit{29}, 2933-2935.

\bibitem{GenomeThreader} Gremme, G., Brendel, V., Sparks, M.E., Kurtz,
  S. (2005) Engineering a software tool for gene structure prediction
  in higher organisms. \textit{Information and Software
    Technology}. \textit{47}, 965-978.

\bibitem{Validation} Adav, S.S., Sze, S.K. (2014). Chapter 8 -
  \textit{TRichoderma} Secretome: An Overview. \textit{Biotechnology
    and Biology of Trichoderma}. 103-104. 
  
\end{thebibliography}

\end{document}
