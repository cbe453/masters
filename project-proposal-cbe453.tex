\documentclass[12pt]{article}
\usepackage[noblocks]{authblk}

\title{Project Proposal for Analysis of Novel \textit{Trichoderma} Genomes}
\author{Connor Burbridge}
\affil{USask NSID: cbe453 \\
  USask ID no.\ 11162928 \\
  Supervisors: Dave Schneider \& Tony Kusalik\\}

\begin{document}
\parindent=14pt
\maketitle

\clearpage
\tableofcontents
\clearpage

\section{Background}

\subsection{Trichoderma}

Crop resistance to environmental stressors is a necessity for crop
health and overall crop yields. Current popular methods for crop
protection involve the use of pesticides and genetically modified
organisms, which can be expensive and potentially politically dividing
in the case of GMOs (citation needed). In addition, crops will suffer
when soils are not sufficient for crop growth and health. These soil
insufficiencies can include drought stress, nutrient stress and can
also include solis that have been contaminated with hydrocarbons,
making it difficult to grow crops in those regions and provides an
opportunity for new bioremediation processes. Recently, two strains of
\textit{Trichoderma} have been identified in the prairie regions of
Alberta and Saskatchewan. These two strains, named Tsth20 and DC1,
have been found to have unique properties when incoluated in plants in
the soils mentioned before.

\textit{Trichoderma} is a type of fungi that can colonize the roots of
plants in a non-toxic, non-lethal, opportunistic symbiotic
relationship.\cite{Trichoderma} Many strains of \textit{Trichoderma}
have been shown to provide resistance to bacteria and other fungi in
soils through the use of polyketides, non-ribosomal peptide
synthetases and other antibiotic
products\cite{Trichoderma}\cite{Secretome}. In addition to these
beneficial properties, the two strains mentioned above provide even
further protection for plants in dry, salty soils and may also be
considered as a bioremediation tool in soils contaminated with
hydrocarbon content. However, little is known about how these
mechanisms work in these new strains, so DC1 and Tsth20 were sequenced
in an attempt to better understand the details of their genomes and
secretomes.

\subsection{Genome Assembly}

Sequence assembly has been a long-standing issue in the field of
computer science\cite{assembly}. Determining the correct order and
combination of smaller subsequences into an accurate sequence assembly
is also computationally difficult in terms of compute resources such
as memory, CPU count and storage required for input
sequences\cite{assembly}. In addition to these difficulties, there can
be difficulties encountered during asssembly due to the nature of the
data or genomes themselves. Insufficient data used in an assembly may
result in short, fragmented assemblies, depending on the size of the
genomes while sequence data that is not long enough can fail to fully
capture repetitive regions in an assembly. To solve this problem, a
wide range of assembly tools have been developed with their own unique
approaches to genome assembly problems, so it is important to use an
appropriate assembler for the task at hand, and important to evaluate
the assembly thoroughly. One approach to aid in the issue of genome
coverage during assembly, is to use a combination of long and short
reads in what is known as a hybrid assembly. Combining both highly
accurate short reads with deep coverage along with less accurate but
much longer reads can produce high quality genome assemblies that
capture long repetitive regions. Assemblies must also be evaluated
with measures such as N50, L50, coverage, average contig length and
total assembled length to ensure that the genomes assemble well at
least based on those metrics\cite{assembly}. Following appropriate
assembly protocols is essential to the further success of a project as
downstream processing such as annotation depends on a high-quality
assembly.

\subsection{Genome Annotation}
With the explosion of sequence data and genomes assemblies made
available in recent years, genome annotation has become a crucial part
of the sequence analyiss pipeline. Genome annotation can involve the
annotation of genes as well as the annotation of structures within a
genome. These annotations can be performed using either evidence
oriented homology-based annotation programs or \textit{ab-initio}
statistical methods which do not consider existing
evidence\cite{AnnotationSummary}. It may also be possible to use a
combination of both in some circumstances. Furthermore, the downstream
products of these genes can then be annotated for functional
properties to determine what functions these genes and gene products
could potentially perform.

\subsection{Repeat Identification/Masking and Identification of AT-rich Genomic regions}
In addition to gene finding, repeat identification within assembled
genomes is another problem that needs to be considered. Regions with
long repeats can have a significant impact on genome assembly as well
as gene finding due to the limitation of short reads used in some
assemblies\cite{Repeats}. Short reads may be unable to bridge or cover
entire repeat regions within a genome, so it is important to consider
the use of long reads such as Nanopore or PacBio when pursuing a new
genome assembly project. It is also possible for these repetitive
regions to contain genes as well, making for an interesting
investigation in regards to \textit{Trichoderma}, as fungal genomes
have been shown to contain many repeat regions with a high
concentration of A and T nucleotides\cite{fungalrepeats}. Once these
repetitive regions have been identified (by any number of existing
programs), the genome can be masked to exlude these regions in
downstream processing if desired. This may affect the gene finding
process described later and may be an interesting topic to look into
considering the large number of available gene finding programs
available.

\subsection{Gene Prediction}
Gene finding is another long standing computational problem in
bioinformatics, which concerns itself with identifying potential genes
within genomes based on patterns or evidence considered by the gene
finding program. These two methods are called \textit{ab initio} for
programs that search for patterns and gene structure, while similarity
or evidence-based searches use prior information such as RNAseq data,
expressed sequence tags and protein sequences to identify gene within
a new genome\cite{GeneFinding}. Complicating the process more are the
introduction of introns and alternative splicing in Eukaryotes, making
it possible for one 'gene' to have several differing versions at the
same locus. Examples of \textit{ab initio} methods include tools such
as GeneMark-ES\cite{GeneMarkES} and AUGUSTUS\cite{AUGUSTUS}, while
evidence based methods include tools such as Braker2\cite{Braker2},
which can incorporate existing data such as RNAseq, protein sequences
etc. in gene prediction models. As mentioned in the repeat finding
section of this proposal, long repeats and transpoable elements can
cause issues for gene finding programs, making this an interesting
area of research, at least for fungal genomes such as
\textit{Trichoderma}. Fungal genomes are also interesting in the topic
of gene finding as there are few programs targeted directly at gene
finding in fungal genomes while organisms such as human, mouse, and
\textit{Arabidopsis} benefit from having many tools tested on them as
they are model organisms in the field, at least according to table 1
in \cite{GeneFinding}. How these different methods and tools perform
when applied to fungal genomes is an important consideration as fungi
have features that can benefit plant growth as mentioned earlier.

\section{Research problem}

Genome annotation, in the case of gene finding in this work, has been
a popular computational problem for decades. The identification of
possible genes provides other researchers with a valuable resource for
future research avenues. Due to the importance and popularity of this
topic, there have been a variety of gene finding tools developed, each
with their own implementation for certain applications or use
cases. The abundance of gene finding tools raises questions such as
which tool is better? Do different tools provide drastically different
results? How much overlap exists between gene finding programs? Are
there stark differences between \textit{ab initio} methods and
similarity-based methods?  Comparative studies of gene finding tools
have been performed before, however most of these studies are in
reference to popular eukaryotic subjects of study such as humans, mice
and \textit{Arabidopsis}. Few if any comparative studies have been
performed in fungal genomes, and even fewer in the case of
\textit{Trichoderma}.

This lack of comparative analysis in fungal genomes provides a
research avenue to compare results from the application of different
gene finding tools to new and existing \textit{Trichoderma}
assemblies. Results from using different annotation tools will allow
us to identify a 'core' genome along with genes that are not predicted
by all tools in all genomes, or outliers. This analysis will also
allow us to compare results from both evidence-based gene finding
methods as well as \textit{ab-initio} gene finding methods. One
interesting area to investigate between these methods would be in
regions of low nuclotide diversity, or the AT-rich regions common to
\textit{Trichoderma} and other fungi. It is possible that gene finding
programs may have difficulty with these regions of genomes, as they
may contain repeat regions and as mentioned before, have low
nucleotide diversity, making it unlikely but not impossible for genes
to be found in these regions. Results from this work will provide
insight into the use of different gene finding tools in the context of
\textit{Trichoderma}. This work will also provide a large set of
called genes spanning at minimum two \textit{Trichoderma} genomes,
along with two new assemblies which have been analyzed for repeat and
AT-rich regions.

\section{Deliverables}
\begin{itemize}
\item Lists of genes for each \textit{Trichoderma} assembly considered
\item A consenus or 'core' gene set shared by each genome considered
  based on different gene finding tools
\item Repeat regions identified in assembled \textit{Trichoderma}
  genomes.
\item A potential list of true positive based on genes which have
  supporting RNAseq eveidence
\end{itemize}

\section{Timeline}
\begin{itemize}
\item Initial genome assemblies: 
\item Gene finding and annotation results: 
\item Analysis of gene finding results: 
\end{itemize}

\section{References}

\begin{thebibliography}{99}
\bibitem{Trichoderma} Hermosa, R., Viterbo, A., Chet, I., Monte,
  E. (2012). Plant-beneficial effects of \textit{Trichoderma} and of
  its genes. \textit{Microbiology}. \textit{38},
  17-25. 

\bibitem{Secretome} Ramirez-Valdespino, C., Casas-Flores, S.,
  Olmedo-Monfil, V. (2019). \textit{Trichoderma} as a Model to Study
  Effector-Like Molecules. \textit{Frontiers in Microbiology} \textit{15}.

\bibitem{Braker2} Hoff, K. J., Lomsadze, A., Borodovsky, M., 
  Stanke, M. (2019). Whole-Genome Annotation with
  BRAKER. \textit{Methods Mol Biol.}, \textit{1962}, 65-95.

\bibitem{GeneMarkES} Borodovsky, M., Lomsadze, A. (2011). Eukaryotic
  Gene Prodeiction using GeneMark.hmm-E and
  GeneMark-ES. \textit{Current Protocols in
    Bioinformatics}. \textit{4}.

\bibitem{AUGUSTUS} Stanke, M., Morgenstern, B. (2005). AUGUSTUS: a web
  server for gene prediction in eukaryotes that allows user-defined
  constraints. \textit{Nucleic Acids Research}. \textit{33}, 465-467.
  
\bibitem{InterProScan} Jones, P., Binns, D., Chang, H. Y., Fraser,
  W., Li, W., ... Hunter, S. (2014). InterProScan 5: genome-scale
  protein function
  classification. \textit{Bioinformatics}. \textit{30}(9),
  1236-1240.

\bibitem{AnnotationSummary} Yandell, M., Ence, D. (2012). A beginner's
  guide to eukaryotic genome annotation. \textit{Nature Reviews
    Genetics}. \textit{13}, 329-342.
  
\bibitem{assembly} Sohn, J., Nam, J. (2016) The present and future of
  \textit{de novo} whole-genome assembly. \textit{Briefings in
    Bioinformatics}. \textit{19}1, 23-40.

\bibitem{Repeats} Lerat., E. (2009). Identifying repeats and
  transposable elements in sequences genomes: how to find your way
  through the dense forest of programs. \textit{Nature
    Heredity}. \textit{104}, 520-533.

\bibitem{fungalrepeats} Li, WC., Huang, CH., Chen, CL. et
  al. (2017). Trichoderma reesei complete genome sequence,
  repeat-induced point mutation, and partitioning of CAZyme gene
  clusters. \textit{Biotechnol Biofuels}. \textit{10}, 170.
  
\bibitem{GeneFinding} Wang, Z., Chen, Yazhu., Li, Y. (2016). A Brief
  Review of Computational Gene Prediction Methods. \textit{Genomics
    Proteomics Bioinformatics}. \textit{2}4, 216-221.
  
\end{thebibliography}

\end{document}
