\documentclass[12pt]{article}
\usepackage[noblocks]{authblk}

\title{Project Proposal for Analysis of Novel \textit{Trichoderma} Genomes}
\author{Connor Burbridge}
\affil{USask NSID: cbe453 \\
  USask ID no.\ 11162928 \\
  Supervisors: Dave Schneider \& Tony Kusalik\\}

\begin{document}
\parindent=14pt
\maketitle

\clearpage
\tableofcontents
\clearpage

\section{Background}

\subsection{Trichoderma}

Crop resistance to environmental stressors is a necessity for crop
health and overall crop yields. Current popular methods for crop
protection involve the use of pesticides and genetically modified
organisms, which can be expensive and potentially politically dividing
in the case of GMOs (citation needed). In addition, crops will suffer
when soils are not sufficient for crop growth and health. These soil
insufficiencies can include drought stress, nutrient stress and can
also include solis that have been contaminated with hydrocarbons,
making it difficult to grow crops in those regions and provides an
opportunity for new bioremediation processes. Recently, two strains of
\textit{Trichoderma} have been identified in the prairie regions of
Alberta and Saskatchewan. These two strains, named Tsth20 and DC1,
have been found to have unique properties when incoluated in plants in
the soils mentioned before.

\textit{Trichoderma} is a type of fungi that can colonize the roots of
plants in a non-toxic, non-lethal, opportunistic symbiotic
relationship.\cite{Trichoderma} Many strains of \textit{Trichoderma}
have been shown to provide resistance to bacteria and other fungi in
soils through the use of polyketides, non-ribosomal peptide
synthetases and other antibiotic
products\cite{Trichoderma}\cite{Secretome}. In addition to these
beneficial properties, the two strains mentioned above provide even
further protection for plants in dry, salty soils and may also be
considered as a bioremediation tool in soils contaminated with
hydrocarbon content. However, little is known about how these
mechanisms work in these new strains, so DC1 and Tsth20 were sequenced
in an attempt to better understand the details of their genomes and
secretomes.

\subsection{Genome Assembly}

Sequence assembly has been a long-standing issue in the field of
computer science\cite{assembly}. Determining the correct order and
combination of smaller subsequences into an accurate sequence assembly
is also computationally difficult in terms of compute resources such
as memory, CPU count and storage required for input
sequences\cite{assembly}. In addition to these difficulties, there can
be difficulties encountered during asssembly due to the nature of the
data or genomes themselves. Insufficient data used in an assembly may
result in short, fragmented assemblies, depending on the size of the
genomes while sequence data that is not long enough can fail to fully
capture repetitive regions in an assembly. To solve this problem, a
wide range of assembly tools have been developed with their own unique
approaches to genome assembly problems, so it is important to use an
appropriate assembler for the task at hand, and important to evaluate
the assembly thoroughly. One approach to aid in the issue of genome
coverage during assembly, is to use a combination of long and short
reads in what is known as a hybrid assembly. Combining both highly
accurate short reads with deep coverage along with less accurate but
much longer reads can produce high quality genome assemblies that
capture long repetitive regions. Assemblies must also be evaluated
with measures such as N50, L50, coverage, average contig length and
total assembled length to ensure that the genomes assemble well at
least based on those metrics\cite{assembly}. Following appropriate
assembly protocols is essential to the further success of a project as
downstream processing such as annotation depends on a high-quality
assembly.

\subsection{Genome Annotation}
With the explosion of sequence data and genomes assemblies made
available in recent years, genome annotation has become a crucial part
of the sequence analyiss pipeline. Genome annotation can involve the
annotation of genes as well as the annotation of structures within a
genome. These annotations can be performed using either evidence
oriented homology-based annotation programs or \textit{ab-initio}
statistical methods which do not consider existing
evidence\cite{AnnotationSummary}. It may also be possible to use a
combination of both in some circumstances. Furthermore, the downstream
products of these genes can then be annotated for functional
properties to determine what functions these genes and gene products
could potentially perform.

\subsection{Gene Prediction}
In progress... Braker2, HMMs, evidence-based vs. \textit{ab initio}
methods

\subsection{Repeat Identification and Masking}
In progress...

\subsection{Identification of AT-Rich Regions}
In progress...

\section{Research problem}

Genome annotation, in the case of gene finding in this work, has been
a popular computational problem for decades. The identification of
possible genes provides other researchers with a valuable resource for
future research avenues. Due to the importance and popularity of this
topic, there have been a variety of gene finding tools that have been
developed. This raises questions such as which tool is better? Do
different tools provide drastically different results? How much
overlap is there? Comparative studies of gene finding tools have been
performed before, however most of these studies are in reference to
popular eukaryotic subjects of study such as humans, mice and in the
case of plants, \textit{Arabidopsis}. Few if any comparative studies
have been performed in fungal genomes, and even fewer in the case of
\textit{Trichoderma}.

This lack of comparative analysis in fungal genomes provides a
research avenue to compare results from the application of different
gene finding tools to new and existing \textit{Trichoderma}
assemblies. Results from using different annotation tools will allow
us to identify a 'core' genome along with genes that are not predicted
by all tools in all genomes (outliers). This analysis will also allow
us to compare results from both evidence-based gene finding methods as
well as \textit{ab-initio} gene finding methods. One interesting area
to investigate between these methods would be in regions of low
nuclotide diversity, or the AT-rich regions common to
\textit{Trichoderma} and other fungi. It is possible that gene finding
programs may have difficulty with these regions of genomes, as they
may contain repeat regions and as mentioned before, have low
nucleotide diversity, making it unlikely for genes to be found in
these regions.

\section{Deliverables}
\begin{itemize}
\item Lists of genes for each \textit{Trichoderma} assembly considered
\item Consenus or 'core' genome for each genome considered based on
  different gene finding tools
\item Comparison of gene finding tools in AT-rich and repetitive regions
\item A potential list of true positive based on genes which have
  supporting RNAseq eveidence
\end{itemize}

\section{Timeline}
\begin{itemize}
\item Initial genome assemblies: September 30th
\item Gene finding and annotation results: November 30th
\item Analysis of gene finding results: December 31st - January 31st
\end{itemize}

\section{References}

\begin{thebibliography}{99}
\bibitem{Trichoderma} Hermosa, R., Viterbo, A., Chet, I., Monte,
  E. (2012). Plant-beneficial effects of \textit{Trichoderma} and of
  its genes. \textit{Microbiology}. \textit{38},
  17-25. 

\bibitem{AntiSmash} Medema, M. H., Blin, K., Cimermancic, P., de
  Jager, V., Zakrzewski, P., Fischbach, M. A., Weber, T., Takano, E.,
  Breitling, R. (2011). antiSMASH: rapid identification, annotation
  and analysis of secondary metabolite biosynthesis gene clusters in
  bacterial and fungal genome sequences. \textit{Nucleic acids
    research}. \textit{39}, 339–346.

\bibitem{Secretome} Ramirez-Valdespino, C., Casas-Flores, S.,
  Olmedo-Monfil, V. (2019). \textit{Trichoderma} as a Model to Study
  Effector-Like Molecules. \textit{Frontiers in Microbiology} \textit{15}.

\bibitem{Braker2} Hoff, K. J., Lomsadze, A., Borodovsky, M., 
  Stanke, M. (2019). Whole-Genome Annotation with
  BRAKER. \textit{Methods Mol Biol.}, \textit{1962}, 65-95.

\bibitem{InterProScan} Jones, P., Binns, D., Chang, H. Y., Fraser,
  W., Li, W., ... Hunter, S. (2014). InterProScan 5: genome-scale
  protein function
  classification. \textit{Bioinformatics}. \textit{30}(9),
  1236-1240.

\bibitem{AnnotationSummary} Yandell, M., Ence, D. (2012). A beginner's
  guide to eukaryotic genome annotation. \textit{Nature Reviews
    Genetics}. \textit{13}, 329-342.
  
\bibitem{assembly} Sohn, J., Nam, J. (2016) The present and future of
  \textit{de novo} whole-genome assembly. \textit{Briefings in
    Bioinformatics}. \textit{19}1, 23-40.

\bibitem{EffectorP} Sperschneider, J., et al. (2016) EffectorP:
  predicting fungal effector proteins from secretomes using machine
  learning. \textit{New Phytologist}. \textit{210}2, 743-761.

\bibitem{SignalP} Teufel, F., et al. (2022). SignalP 6.0 predicts all
  five types of signal peptides using protein language
  models. \textit{Nature Biotechnology}. \textit{40}, 1023-1025.
  
\end{thebibliography}

\end{document}
