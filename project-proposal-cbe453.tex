\documentclass[12pt]{article}
\usepackage[noblocks]{authblk}

\title{Project Proposal for Analysis of Novel \textit{Trichoderma} Genomes}
\author{Connor Burbridge}
\affil{USask NSID: cbe453 \\
  USask ID no.\ 11162928 \\
  Supervisors: Dave Schneider \& Tony Kusalik\\}

\begin{document}
\parindent=14pt
\maketitle

\clearpage
\tableofcontents
\clearpage

\section{Introduction}

test
Crop resistance to environmental stressors is a necessity for crop
health and overall crop yields. Current popular methods for crop
protection involve the use of pesticides and genetically modified
organisms, which can be expensive and potentially politically dividing
in the case of GMOs (citation needed). In addition, crops will suffer
when soils are not sufficient for crop growth and health. These soil
insufficiencies can include drought stress, nutrient stress and can
also include solis that have been contaminated with hydrocarbons,
making it difficult to grow crops in those regions and provides an
opportunity for new bioremediation processes. Recently, two strains of
\textit{Trichoderma} have been identified in the prairie regions of
Alberta and Saskatchewan. These two strains, named Tsth20 and DC1,
have been found to have unique properties when incoluated in plants in
the soils mentioned before.

\textit{Trichoderma} is a type of fungi that can colonize the roots of
plants in a non-toxic, non-lethal, opportunistic symbiotic
relationship.\cite{Trichoderma} Many strains of \textit{Trichoderma}
have been shown to provide resistance to bacteria and other fungi in
soils through the use of polyketides, non-ribosomal peptide
synthetases and other antibiotic
products\cite{Trichoderma}\cite{Secretome}. In addition to these
beneficial properties, the two strains mentioned above provide even
further protection for plants in dry, salty soils and may also be
considered as a bioremediation tool in soils contaminated with
hydrocarbon contents. However, little is known about how these
mechanisms work in these new strains, so DC1 and Tsth20 were sequenced
in an attempt to better understand the details of their genomes and
secretomes.

\section{Research Avenues}

With this data, there are several possible research angles that could
be approached. However, the first step is to assemble and annotate
these genomes, which can act as deliverables for this project. There
are currently two initial assemblies for both DC1 and Tsth20. One set
of assemblies using only Illumina sequence data, and one hybrid
assembly with Nanopore and Illumina reads. As expected, the
hybrid asssemblies are more contiguous than assemblies with only
Illumina data, although more testing and assembiles using other
programs is required. One interesting observation noted when
processing the input sequence data is a number of sections of low
variability in nucelotide content, indicating potential repeat regions
or simply just areas of interest for further research. One article
mentions that avirulence and effector-like proteins in close proximity
to AT-rich regions may promote mutations of said genes and proteins as
repeat-induced-mutations may alter these genes frequently, although
this is research targeted at pathogenic fungi.

Following assembly, the genomes will be annotated using publicly
available RNAseq data from several \textit{Trichoderma} strains and a
gene calling tool such as Braker2\cite{Braker2} to train and call
possible genes. Other gene calling tools may be considered as
well. Annotated genes will then be processed using tools such as
EffectorP\cite{EffectorP} and SignalP\cite{SignalP} in order to
determine if gene products are involved in processes associated with
host plants. Another tool has recently come to our attention
AntiSmash\cite{AntiSmash}, which aids in determing membership of gene
products in signalling pathways so we may also incorporate that. The
combined use of these tools should provide a better understanding of
the secretomes used by these new strains. This annotation process
presents a possible research avenue which may contribute to the
computer science aspect of this project. Currently, these tools
(SignalP and EffectorP) determine membership of each group using
different machine learning, statistical methods, and cutoffs, making
the results difficult to compare and coalesce. An attempt could be
made to produce a more unified approach to determing whether or not a
gene product belongs to one of these categories. Should other
opportunities for computational contributions present themselves
during this process, they may also be pursued.

Once suitable features of interest have been identified, a comparative
analysis could be performed, including DC1, Tsth20 as well as other
well established \textit{Trichoderma} assemblies. This analysis could
also include the AT-rich regions discussed ealier for a comprehensive
comparison and understanding of AT-rich regions as well as effector an
signalling molecules in the context of multiple genomes. However, this
may be beyond the scope of the project.

\section{Genome Assembly}

Sequence assembly has been a long-standing issue in the field of
computer science\cite{assembly}. Determining the correct order and
combination of smaller subsequences into an accurate sequence assembly
is also computationally difficult in terms of compute resources such
as memory, CPU count and storage required for input
sequences\cite{assembly}. In addition to these difficulties, there can
be difficulties encountered during asssembly due to the nature of the
data or genomes themselves. Insufficient data used in an assembly may
result in short, fragmented assemblies, depending on the size of the
genomes while sequence data that is not long enough can fail to fully
capture repetitive regions in an assembly. To solve this problem, a
wide range of assembly tools have been developed with their own unique
approaches to genome assembly problems, so it is important to use an
appropriate assembler for the task at hand, and important to evaluate
the assembly thoroughly. One approach to aid in the issue of genome
coverage during assembly, is to use a combination of long and short
reads in what is known as a hybrid assembly. Combining both highly
accurate short reads with deep coverage along with less accurate but
much longer reads can produce high quality genome assemblies that
capture long repetitive regions. Assemblies must also be evaluated
with measures such as N50, L50, coverage, average contig length and
total assembled length to ensure that the genomes assemble well at
least based on those metrics\cite{assembly}. Following appropriate
assembly protocols is essential to the further success of a project as
downstream processing such as annotation depends on a high-quality
assembly.

\section{The Secretome}
In progress...

\section{Genome Annotation}
With the explosion of sequence data and genomes assemblies made
available in recent years, genome annotation has become a crucial part
of the sequence analyiss pipeline. Genome annotation can involve the
annotation of genes as well as the annotation of structures within a
genome. These annotations can be performed using either evidence
oriented homology-based annotation programs or \textit{ab-initio}
statistical methods which do not consider existing
evidence\cite{AnnotationSummary}. It may also be possible to use a
combination of both in some circumstances. Furthermore, the downstream
products of these genes can then be annotated for functional
properties to determine what functions these genes and gene products
could potentially perform.

\subsection{Gene Prediction}
In progress... Braker2

\subsection{Functional Annotation}
In progress... InterProScan, EffectorP, SignalP and AntiSmash(?)

\subsection{Repeat Identification and Masking}
In progress...

\subsection{Identification of AT-Rich Regions}
In progress...

\section{Deliverables}

\begin{itemize}
  \item Committee meeting: August 1st - 5th or August 8th - 12th
  \item Genome assemblies of Tsth20 and DC1: August 31st
  \item Gene annotation of assemblies: September 30th
  \item Identify potential effectors and siganlling proteins along with AT-rich regions of assemblies: October 31st
  \item Identify computational research problem and solve: November - January
  \item Start thesis: January - February 2023
  \item Finish thesis: April - May 2023
\end{itemize}

\section{References}

\begin{thebibliography}{99}
\bibitem{Trichoderma} Hermosa, R., Viterbo, A., Chet, I. \& Monte,
  E. (2012). Plant-beneficial effects of \textit{Trichoderma} and of
  its genes. \textit{Microbiology}. \textit{38},
  17-25. 

\bibitem{AntiSmash} Medema, M. H., Blin, K., Cimermancic, P., de
  Jager, V., Zakrzewski, P., Fischbach, M. A., Weber, T., Takano,
  E., \& Breitling, R. (2011). antiSMASH: rapid identification,
  annotation and analysis of secondary metabolite biosynthesis
  gene clusters in bacterial and fungal genome sequences. Nucleic
  acids research, 39(Web Server issue),
  W339–W346.

\bibitem{Secretome} Ramirez-Valdespino, C., Casas-Flores, S.,
  Olmedo-Monfil, V. (2019). \textit{Trichoderma} as a Model to Study
  Effector-Like Molecules. \textit{Frontiers in Microbiology} \textit{15}.

\bibitem{Braker2} Hoff, K. J., Lomsadze, A., Borodovsky, M., \&
  Stanke, M. (2019). Whole-Genome Annotation with
  BRAKER. \textit{Methods Mol Biol.}, \textit{1962}, 65-95.

\bibitem{InterProScan} Jones, P., Binns, D., Chang, H. Y., Fraser,
  W., Li, W., ... Hunter, S. (2014). InterProScan 5: genome-scale
  protein function
  classification. \textit{Bioinformatics}. \textit{30}(9),
  1236-1240.

\bibitem{AnnotationSummary} Yandell, M., Ence, D. (2012). A beginner's
  guide to eukaryotic genome annotation. \textit{Nature Reviews
    Genetics}. \textit{13}, 329-342.
  
\bibitem{assembly} Sohn, J., Nam, J. (2016) The present and future of
  \textit{de novo} whole-genome assembly. \textit{Briefings in
    Bioinformatics}. \textit{19}1, 23-40.

\bibitem{EffectorP} Sperschneider, J., et al. (2016) EffectorP:
  predicting fungal effector proteins from secretomes using machine
  learning. \textit{New Phytologist}. \textit{210}2, 743-761.

\bibitem{SignalP} Teufel, F., et al. (2022). SignalP 6.0 predicts all
  five types of signal peptides using protein language
  models. \textit{Nature Biotechnology}. \textit{40}, 1023-1025.
  
\end{thebibliography}

\end{document}
