\documentclass[12pt]{article}

\usepackage{amsfonts}
\usepackage{amsmath}

\renewcommand{\Pr}[1]{\ensuremath{\mathbb{P}\left[#1\right]}}

\title{Binomial test of random placement\\
  of gene calls}
\author{Dave Schneider}

\begin{document}

\maketitle

Let $q$ be the length fraction of the genome assigned the low GC
content class and $N$ be the total number of gene calls.  If the gene
calls are ``random distributed'' between regions of normal and low GC
content then one would expect that the proportions of gene calls in
the two classes would more or less follow the proportions of total
lengths of the two classes.  Let $k$ be the number of gene calls in
regions of low GC content.

Consider a single gene call (i.e., $N=1$).  If it is randomly placed
on the genome then it will be assigned to a low GC content region with
probability $q$ and, correspondingly, probability $1-q$ for occurrence in a region of normal GC content.  In other words
\begin{align}
  \Pr{k=0\mid N=1} & = 1 - q \\
  \Pr{k=1\mid N=1} & = 1  
\end{align}
Now consider the outcomes for $N=2$ \emph{independent} gene calls
sampled from the same distribution.  There is one way to get either
zero or two ``hits''
\begin{align}
  \Pr{k=0\mid N=2} & = \left(1 - q\right)^2 \\
  \Pr{k=2\mid N=2} & = q^2
\end{align}
but two ways to get exactly one hit
\begin{equation}
  \Pr{k=1\mid N=2} = 2\left(1-q\right)q 
\end{equation}
the first hit could be to a low GC content region and the second to a
normal region, or \emph{vice versa}.

In general, the number of ways that the outcomes can be partitioned
into two classes is given by the well known binomial coefficients.
So, the general formula is for finding $k=n$ out of $N$ hits in
regions of low GC content is
\begin{equation}
  \Pr{k=n \mid N} = \binom{n}{N} q^{n}\left(1-q\right)^{N-n}
  \label{eq:binomial_mass}
\end{equation}

The binomial test (e.g., \texttt{scipy.stats.binomtest}) can be used
to determine whether the observation of $k=n$ out of $N$ hits
occurring regions of low GC content is consistent with the null
hypothesis that the calls are random assigned to regions of low GC
content with probability $q$.  In other words, the null hypothesis
implies that true probability mass function for $k$ is given by
Equation~\ref{eq:binomial_mass}). This function is peaked in the
vicinity of $n=Nq$.  Note, you may want to use the
\texttt{scipy.stats.binom} package to graph this function.  The peaked
nature of this function suggests that if $k$ is very different than
$Nq$ then it is unlikely that the true probability mass function is
given by Equation~\ref{eq:binomial_mass}.

The question is how close do $k$ and $Nq$ have to be in order for
Equation~\ref{eq:binomial_mass} to provide a plausible explanation of the
observed data.  The call
\begin{verbatim}
result = scipy.stats.binomtest(k, N, q, alternative='two-sided')
\end{verbatim}
provides the answer.  The \texttt{alternative=two-sided} argument
implies that the test should consider the possibility that the true
proportion is either greater or less than \texttt{q}.  You can
retrieve the so-called $p$-value
\begin{verbatim}
print('p-value: {}'.format(result.pvalue)
\end{verbatim}
as well as the confidence interval (see documentation for details).

%------------------------------

\end{document}
